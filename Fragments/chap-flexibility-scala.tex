There are a number of ways to improve the flexibility of the
basic reference immutability system.

\section{Locality-dependent Result Qualifers}

It is useful to be able to say that a method's result-type qualifier
depends only on the receiver's qualifier,
only on the qualifier of a parameter of that method,
or on neither the receiver's qualifier nor a parameter's qualifier.


For example, if a method returns one of its receiver's fields,
and the method is called with a {\em mutable} receiver reference,
then the result should also be considered {\em mutable} even though
the method itself treats the receiver as {\em readonly}.
Listing~\ref{lst-loc} shows

\begin{lstlisting}[caption={Locality-dependent Results},label={lst-loc},float=htp]
class C {
	val f = ...
	def m1(this: @readonly, c: @readonly): @readonly = {
		this.f
	}
	def m2(this: @readonly, c: @readonly): @readonly = {
		c.f
	}
}
val c = new C
val cr: @readonly = new C
c.m1(c)   // result is @readonly
c.m1(cr)  // result is @readonly
cr.m1(c)  // result is @readonly
c.m2(c)   // result is @readonly
c.m2(cr)  // result is @readonly
cr.m2(c)  // result is @readonly
\end{lstlisting}





It is useful to be able to differentiate the following cases:



It is useful to be able to say that the result of a method was reached through
a path containing one of that method's parameters.




It is also useful to be able to say that the result of a method
was not reached through any method parameter.

Listing~\ref{lst-loc} shows three versions of method~{\cd m}, each with a
different result locality.


\begin{lstlisting}[caption={Locality-dependent Results (2)},label={lst-loc-2},float=htp]
class C {
	def m1(this: @rothis, c: @readonly): @rothis = {
		this.f
	}
	def m2(this: @readonly, c: @roparam): @roparam = {
		c.f
	}
}
val c = new C
val cr: @readonly = new C
c.m1(c)   // result is @mutable
c.m1(cr)  // result is @mutable
cr.m1(c)  // result is @readonly
c.m2(c)   // result is @mutable
c.m2(cr)  // result is @readonly
cr.m2(c)  // result is @mutable
\end{lstlisting}

\begin{lstlisting}[caption={Locality-dependent Results},label={lst-loc},float=htp]
class C {
	def m1(this: @readonly, p: @readonly): @readonly = {
		this.f
	}
	def m2(this: @readonly, p: @readonly): @readonly = {
		p.f
	}
	def m3(this: @mutable, p: @mutable): @mutable = {
		new D()
	}
}
val c = new C
val cr: @readonly = new C
val p: @mutable = ...
val pr: @readonly = ...
c.m1(p)
\end{lstlisting}



\begin{lstlisting}[caption={Polymorphism with {\em rothis}},label={lst-rothis},float=htp]
class C {
  private var f = ...
  def getF(this: @rothis) = { this.f }
}
val mc: C @mutable
val mf = mc.getF     // returns a mutable reference
val rc: C @readonly
val rf = rc.getF     // returns a readonly reference
\end{lstlisting}





\section{}

\begin{lstlisting}
class C {
	def m1(this: @mutable)
}
def m2(c: C @readonly) {
	c.m1()     // Error: Cannot call m1 due to incompatible receiver
}
val c: @mutable = new C
m2(c)        // OK: mutable <: readonly
\end{lstlisting}

\begin{lstlisting}
class C {
	def m1(this: @mutable)
}
def m2(c: C @readonly @*@calls[C]*@) {
	c.m1()     // OK: allowed by @calls[C]
}
val c: @mutable = new C
val cr: @readonly = new C
m2(c)        // OK: calls to c.m1 are allowed here
m2(cr)       // Error: calls to cr.m1 are not allowed here
\end{lstlisting}

What restrictions exist on copying references with @calls qualifiers?
1. calls[D] <: calls[C] where D <: C.
		calls[D] is less restrictive than calls[C],
		because the set of methods in D is a superset of the set of methods in C.
2. The default calls qualifer is @calls[Any] (or $calls_\top$).
		@calls[Any] entails no call-site restrictions, but neither does it allow
		any in-callee exceptions.
3. Therefore,
	var c: @calls[C]
	var d: @calls[Any]
	d = c   // OK
	c = d   // Error
4. The key to this system is that a calls[Any] reference r may be
		passed to a method expecting a calls[C] reference,
		provided r is of type C and qual(r) <: rcv(m), where
		m is any method of class C.

\section{Qualifier Polymorphism on Receivers}
\label{sec-poly-qual}

I introduce a polymorphic qualifier {\em rothis},\footnote{
	The reason for the name {\em rothis} is that {\em rothis}-qualified
	references are those which are expected to be reached through an access path
	rooted at a {\em readonly this.}
}
which is similar in purpose and function to
Javari's {\em romaybe}~\cite{javari} and ReIm's {\em polyread}~\cite{reim}.
The purpose of {\em rothis} is to improve the flexibility of getter methods.
Like {\em mutable}-returning methods, {\em rothis}-returning methods are adapated to
the receiver qualifier at call sites~--
an {\em rothis}-returning method called with a {\em mutable}
receiver will yield a {\em mutable} result, and the same method called
with a {\em readonly} receiver will yield a {\em readonly} result.

% methods with an rothis this will not mutate the locality of this
However, an {\em rothis}-qualified reference, like a {\em readonly}-qualified reference,
does not allow mutation of its referent object.
Also like {\em readonly}, the non-mutation guarantee offered by {\em rothis}
is transitive.
Therefore, a method where the receiver parameter {\cd this} is qualified
by {\em rothis} can be safely called with a {\em readonly} receiver reference~--
a method with a {\em mutable}-qualified {\cd this} cannot.

There is no particular requirement that an {\em rothis}-returning method
must have an {\em rothis} {\cd this}, nor that a method with an {\em rothis} {\cd this}
must have an {\em rothis} result.
However, it is likely to be the case that the type inference mechanism
will automatically infer an {\em rothis} result for methods that have an {\em rothis} {\cd this}.

% avoidance of code duplication (const)
Code duplication is prevented in cases where a method has both an {\em rothis} {\cd this}
and an {\em rothis} result. Without {\em rothis}, many getter methods would need at least two versions:
one that returns {\em mutable} results, and another that accepts {\em readonly} receivers.


% lattice: between mutable and readonly
The {\em rothis} qualifier is positioned below {\em readonly} on the qualifier lattice.
The positioning below {\em readonly} guarantees that an {\em rothis}-returning method
will never return a reference that was reached through a path that contains
a {\em readonly} element, unless that path element happens to be {\cd this}.

For example, listing~\ref{lst-rothis} shows a getter method {\cd getF} that
returns a private field {\cd f}. Calling {\cd getF} with a {\em mutable} receiver
returns a {\em mutable} reference, and calling {\cd getF} with a {\em readonly}
receiver returns a {\em readonly} reference.
The receiver parameter {\cd this} is shown
explicitly, qualified by {\em rothis}. (Scala does not support explicit {\cd this} parameters~--
the example here does not compile as shown. I will discuss syntactially-valid
alternative annotation schemes in a later chapter.)
I assume here that the field {\cd f} is {\em mutable}, and that the result of
method {\cd getF} is inferred.

\begin{lstlisting}[caption={Polymorphism with {\em rothis}},label={lst-rothis},float=htp]
class C {
  private var f = ...
  def getF(this: @rothis) = { this.f }
}
val mc: C @mutable
val mf = mc.getF     // returns a mutable reference
val rc: C @readonly
val rf = rc.getF     // returns a readonly reference
\end{lstlisting}


% other approaches: polyread/romaybe

% other approaches: generalization over parameters (prev) - handling of multiple params
Initially, I tried a multi-parameter interpretation of {\em rothis}
where any number of method parameters (in addition to {\cd this}) participated
in the viewpoint adaptation operation.
In the multi-parameter interpretation, the result of an {\em rothis}-returning
method is the least upper bound of the qualifiers on all arguments passed by the caller
to {\em rothis}-qualified method parameters.
However, I have yet to find a practical or idiomatic Scala example that
would benefit from a multi-parameter interpretation.

\begin{comment}
% The reason built-in polymorphism doesn't work is that what we usually
%  want is to relate the result qualifier to the qualifiers of one or more parameters,
%  even though the underlying types of the parameters and result could be incompatible.
Scala has built-in support for type polymorphism.
Listing~\ref{lst-polymorph-via-types} shows 
I conjecture that, for any practical cases where qualifier polymorphism beyond {\em rothis}
is required, Scala's standard facilities for type polymorphism will be adequate.
\begin{lstlisting}[caption={Qualifier Polymorphism Via Type Polymorphism},label={lst-polymorph-via-types},float=htp]
def m(t: T): R = ...                                      // monomorphic
def m[U <: T @readonly, V <: R @readonly](t: U): V = ...  // polymorphic 
\end{lstlisting}
\end{comment}

When read through an environment reference, a local variable must be observed to
have a type with a field-admissible qualifier.
Specifically, variables with {\em fresh} or {\em rothis} qualifiers are observed
to have {\em mutable} or {\em readonly} qualifiers (respectively) when read from within nested classes.
The qualifier adjustment occurs before viewpoint adaptation.

%The reason for not allowing a nested-class variable read to yield an {\em rothis}
%result is that {\em rothis} is a polymorphic qualifier that does not have the
%same meaning when observed from inside a nested class.


%The reason for not allowing a nested-class variable read to yield a {\em fresh}
%result is that an object of the type of the nested class could survive past
%the return of the enclosing method.
%After the enclosing method returns, the {\em fresh}-qualified reference is

When a variable is written from within a nested class, its qualifier is not adjusted.
If variable type qualifiers were adjusted on writes, then {\em mutable} references could
get written to {\em fresh} variables, or {\em readonly} references could get written to
{\em rothis} variables.

\section{Qualifier Polymorphism on Formal Parameters}
\label{sec-poly-qual-param}

TBD.

\section{Returning Fresh References} \label{sec-fresh}

TBD.

\section{Viewpoint Adaptation on New Object Construction} \label{sec-vp-new}

TBD.
% Basic idea: have "new" return a qualifier equal to the path-dependent qualifier used to select the type

\section{The Package Hierarchy} \label{sec-packages}

Environment reference qualifiers can be extended to express the ability to mutate
within a particular package.
Listing~\ref{lst-package-env} shows a class {\cd C} that updates a static counter
during construction. Class {\cd C}'s environment reference allows mutation
within the inner package~{\cd B}, but not the outer package {\cd A}.\footnote{
	Packages in production code are usually arranged in a file system hierarchy
	rather than a scoped hierarchy as shown in listing~\ref{lst-package-env}.
	Listing~\ref{lst-package-env} shows a scoped hierarchy for clarity of presentation.
}
The environment qualifier~\mbox{{\em mutable}$_B$} allows mutation within
object~{\cd D}, which is within package {\cd B}.

%Like nested classes, packages are arranged in a hierarchy.
%Unlike ordinary classes, package contents are always instantiated in the static environment.

\begin{lstlisting}[caption={Packages and Environment References},float={htp},label={lst-package-env}]
package A {
	package B {
		class C(env: @*@mutEnv[B]*@) {
			D.initCount += 1           // update a static counter
		}
		object D {
			var initCount = 0
		}
	}
}
\end{lstlisting}

In Scala, hierarchies of packages are intentially similar to hierarchies of objects.
Listing~\ref{lst-object-env} shows the replacement of the {\cd package}
keyword with the {\cd object} keyword, with no change to code meaning.

\begin{lstlisting}[caption={Objects and Environment References},float={htp},label={lst-object-env}]
object A {
	object B {
		class C(env: @*@mutEnv[B]*@) {
			D.initCount += 1           // update a static counter
		}
		object D {
			var initCount = 0
		}
	}
}
\end{lstlisting}

At first glance, supporting mutation permissions at the package level
may seem less than useful in a purity system~-- packages are globally-visible
entities, which means that side effects within packages are potentially
observable from anywhere.
However, limitation of observational exposure is a separate problem.
It may be possible to show in future work that some methods can
observe side effects only within certain packages, making them independent from
methods that mutate only within different packages.
In any case, knowing that a method can mutate only within certain packages
could aid the programmer in understanding what side effects a method can produce.

\section{Side Effects in the External World} \label{sec-external-world}

TBD.
