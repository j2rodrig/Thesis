\documentclass[letterpaper,11pt]{article}

\usepackage{alltt}
\usepackage{amsmath}
\usepackage[greek,english]{babel}
%\usepackage{booktabs}
%\usepackage{cite}
%\usepackage{color}
\usepackage{fixltx2e}
\usepackage[T1]{fontenc}
%\usepackage{fullpage}
\usepackage[top=1in, bottom=1in, left=1in, right=1in]{geometry}
\usepackage{graphicx}
\usepackage[utf8]{inputenc}
\usepackage{listings}
%\usepackage{mathtools}
\usepackage{mdwlist}
%\usepackage{needspace}
\usepackage{url}
\usepackage{verbatim}
\usepackage{xspace}

% Fonts and symbols
%\usepackage{amsfonts}
%\usepackage{amssymb}
%\usepackage[scaled=.65]{beramono}
\usepackage{courier}
\usepackage{times}

\newenvironment{absolutelynopagebreak}
  {\par\nobreak\vfill\penalty0\vfilneg
   \vtop\bgroup}
  {\par\xdef\tpd{\the\prevdepth}\egroup
   \prevdepth=\tpd}

\newcommand{\greekmu}{\greektext m\latintext}

\newcommand{\tab}[0]{\hspace{0.4cm}}

\renewcommand{\ttdefault}{txtt}

\lstset{
	basicstyle=\footnotesize\ttfamily,
	tabsize=4,
	extendedchars=true,
	breaklines=true,
	escapeinside={(*}{*)}
}

\newcommand{\code}[1]{\lstinline$#1$}



\begin{document}

\title{
	An Alternative Effect-Checking System for Scala
}
\author{Jonathan Rodriguez\\University of Waterloo}
\maketitle

The objective of the proposed work is to develop a side-effect
annotation system based on reference immutability for the Scala~\cite{scala-book} language
(specifically, the Dotty compiler~\cite{dotty}).
The system is inspired by ReIm~\cite{reim}, a reference-immutability checker
for Java.

% What reference immutability is.
In a reference immutability system, a \emph{read-only reference}
has a \emph{deep} or \emph{transitive} guarantee that no writes will occur
to any object that has been reached through that reference
(in constrast to C++'s \emph{const} or Java's \emph{final}, which provide shallow
or non-transitive guarantees only).
% What reference immutability can be used for: purity? concurrency? compiler optimizations?
% Is it just purity that's important? What are examples of useful guarantees that we might have?
% Reference immutability is a contributor guarantee (can be used in combination with other information to prove stuff)
Reference immutability guarantees have several uses -- in conjunction with other
readily-available information, reference immutability can be used to
express that certain objects cannot be modified after construction, and to
modularly prove absence of side effects (functional \emph{purity}).
In addition to aiding human reasoning,
information about immutability and purity could support compiler
optimizations such as automatic memoization and parallelization.
	% relate to side-effects
	% talk about purity
	% compiler optimizations - 
	% automatic parallelization of pure functions


% Unanswered questions:
% Scala language more complicated than Java:
% Dependent types (what does this have to do with anything?),
% First-class functions (closures),
% Arbitrary nesting of functions and classes.
% Existing work: efftp (cite problems), Insane (no checking), others (not for Scala).
% Our work is also for dotty.
% We hope that the findings will be transferrable to other languages with Scala-like features
ReIm, unfortunately, is built for Java, not Scala.
Scala supports (and seems to encourage the ubiquitous use of)
first-class functions, nested function definitions, and lambda expressions.
Existing approaches for Scala either do not support modularly checkable annotions
(e.g., Insane~\cite{effect-analysis-callbacks}) or have not been systematically evaluated
on real programs (e.g., Efftp~\cite{efftp}).
We attempted to evaluate Efftp on real programs, but we were unable to
perform a sound evaluation due to the errors we encountered
in Efftp's implementation.


% Support for performing inference in addition to checking: Scala seems to encourage a programming style with lots of defintions, and also automatically generates definitions.

% reference imm vs object imm

\begin{comment}
Reference immutability is a form of ??deep ownership
where mutations of shared state cannot be performed through any access path
that contains a \emph{read-only reference}.
\emph{ReIm~\cite{reim}} implements a reference immutability system for Java.
A ReIm-like system may also work well for Scala code,
but Scala introduces a variety of language constructs that make it difficult to
see clearly whether a ReIm-like system would work well for Scala.
\end{comment}

Our expected contributions include:
\begin{itemize}
\item The first effect annotation system for Dotty~\cite{dotty},
	a new compiler front-end for Scala that
	implements the Dependent Object Types~(DOT) calculus~\cite{dot-calculus}.

\item Demonstration of an approach to side-effect analysis
	that, unlike Efftp, makes a clearly-visible distinction between datatype annotations
	(which are propagated like ordinary data types) and
	effect annotations (which are summaries of entire function bodies).
	For example, it makes sense to have readonly references and pure functions,
	but not readonly functions or pure references.
	Efftp attempts to deal with both kinds of annotations as annotations
	on function return types, an approach which we believe adds unnecessary complexity
	to the implementation.

\item A discussion of the surprising correspondence between ReIm and JPure~\cite{jpure}.
	We find that JPure's notions of freshness and staleness are nearly identical to
	ReIm's notions of mutability and immutability, respectively.
	We furthermore find that ReIm's \emph{viewpoint adaptation} makes
	it strictly more precise than JPure.

\item An evaluation on real programs, e.g., the PSP collections library~\cite{psp-page},
	the Dotty compiler, and others.
	%A systematic evaluation of Efftp on
	%real programs was never performed -- we speculate that the errors in Efftp's implementation
	%would make a systematic evaluation quite difficult.
	If time permits, we would like to build an inference tool that can suggest annotations
	of source files in large projects.

\item We also hope to answer the question of whether Efftp's generalized notion
	of freshness~\cite{efftp} can provide improvements in precision
	over a ReIm-like viewpoint adaptation in real programs.
	%Efftp allows a function to be annotated with the specific localities
	%it may return. In contrast, freshness in JPure is a simple all-or-nothing condition.
	
\end{itemize}

%\bibliographystyle{alpha}
%\bibliographystyle{plain}
\bibliographystyle{abbrv}
\bibliography{../references}

\end{document}
