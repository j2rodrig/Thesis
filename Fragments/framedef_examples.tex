\documentclass[letterpaper,11pt]{article}

\usepackage{alltt}
\usepackage{amsmath}
\usepackage[greek,english]{babel}
%\usepackage{booktabs}
%\usepackage{cite}
%\usepackage{color}
\usepackage{fixltx2e}
\usepackage[T1]{fontenc}
%\usepackage{fullpage}
\usepackage[top=1in, bottom=1in, left=1in, right=1in]{geometry}
\usepackage{graphicx}
\usepackage[utf8]{inputenc}
\usepackage{listings}
%\usepackage{mathtools}
\usepackage{mdwlist}
%\usepackage{needspace}
\usepackage{url}
\usepackage{verbatim}
\usepackage{xspace}

% Fonts and symbols
%\usepackage{amsfonts}
%\usepackage{amssymb}
%\usepackage[scaled=.65]{beramono}
\usepackage{courier}
\usepackage{times}

\newenvironment{absolutelynopagebreak}
  {\par\nobreak\vfill\penalty0\vfilneg
   \vtop\bgroup}
  {\par\xdef\tpd{\the\prevdepth}\egroup
   \prevdepth=\tpd}

\newcommand{\greekmu}{\greektext m\latintext}

\newcommand{\tab}[0]{\hspace{0.4cm}}

\renewcommand{\ttdefault}{txtt}

\lstset{
	basicstyle=\footnotesize\ttfamily,
	tabsize=4,
	extendedchars=true,
	breaklines=true,
	escapeinside={(*}{*)},
	frame=l,  % trbl for single-line frames, TRBL for double-line frames
	xleftmargin=\parindent
}

\newcommand{\code}[1]{\lstinline$#1$}



\begin{document}

\section{The Lifting Transformation}

\label{framedef_examples}

Nested definitions are lifted to the top level to simplify the analysis.
All references to enclosing scopes are passed explicitly after the lifting operation,
allowing us to retain a simple definition of purity: if all parameters of
a function are immutable, then the function is pure.

The required elements of the transformation are the following:

\begin{itemize}
	
	\item [\mbox{\small DEFINITIONS}] For each class and function definition, add one parameter for each enclosing scope,
		up to and including the nearest enclosing class or (if there is no enclosing class) the package root.
		For definitions immediately enclosed by a class, the only extra parameter is the receiver (\code{this}).
		
	\item [\mbox{\small APPLICATIONS}] For each function application or constructor invocation,
		add additional parameters as necessary
		to match the class or function definition.
		If the applied function or constructor is defined inside the local scope, the special object \code{myframe}
		is passed. The \code{myframe} object represents the stack frame of the current function.
		Through the \code{myframe} reference, inner definitions are able to observe and modify
		local variables and parameters.
		
	\item [\mbox{\small IDENTIFIERS}] For the first identifier on each access path, substitute an equivalent
		access path prefix that starts with a local parameter or local variable.
	
	\item [\mbox{\small CLOSURES}] Partial evaluations produce closure objects. Each closure object must contain
		a field for every applied parameter, including extra parameters.
	
\end{itemize}

\subsection{Examples}

\subsubsection{Definition transformation}

To illustrate the lifting transformation, we provide some examples.
The following Scala code contains nested functions, classes, and packages:

\begin{lstlisting}
package P {
	package Q {
		class C () {
			def F () = {
				class D () {
					def G () = {
						def H () = {
						}
					}
				}
			}
		}
	}
}
\end{lstlisting}

The liftings of \code{F} and \code{G} are straightforward.
The receiver object \code{this} is passed as a parameter,
giving the following new signatures:

\begin{lstlisting}
def F (this : C)
def G (this : D)
\end{lstlisting}


The lifting of \code{H} requires two parameters, one for \code{this}
and one for the frame of the enclosing method \code{G}.
The frame parameter \code{enclG} does not have a programmer-accessible type,
but since the runtime system must have some memory-based
representation of the frame of \code{G}, the type nevertheless exists.
For the purpose of example, we give \code{enclG}'s type the name \code{FrameG}:

\begin{lstlisting}
def H (enclG : FrameG, this : D)
\end{lstlisting}

Classes \code{D} and \code{C} must also be given parameters for enclosing scopes.
Class \code{D} requires parameters for the frame of \code{F} and
the enclosing class \code{C}.
We use the name \code{FrameF} to denote the type of the frames of \code{F}.
Class \code{C} does not have any enclosing classes, but it does have enclosing
packages.
We give \code{C} a parameter for each enclosing package: for package \code{P},
\code{enclP}, and for \code{Q}, \code{enclQ}.
We must also include a parameter for the root package.
The \code{packageRoot} parameter provides an access path to packages other than \code{P} or \code{Q}.

\begin{lstlisting}
class D (enclF : FrameF, enclC : C)
class C (enclQ : Q, enclP : P, packageRoot : Root)
\end{lstlisting}


\subsubsection{Application and identifier transformation}

Calls to \code{F} and \code{G} (defined above) must include the receiver object as a parameter.
For example, \code{c.F()} becomes \code{c.F(c)}.
Note that the transformed expression is still qualified by \code{c}
because the correct function must be called even if \code{F} is overridden in a subclass of C.

A call to \code{H} must additionally include a parameter of type \code{FrameG}.
Since \code{H} is only visible inside of \code{G}, there is always an object of
type \code{FrameG} available;
from inside an inner definition, the parameter \code{enclG}, and from
the body of \code{G} itself, the special object \code{myframe}.

Similarly, the expression \code{new D()} is expanded to \code{new D(myself, this)}
from within the body of \code{F} and \code{new D(enclF, enclC)}
from within the body of \code{D}.

If a function call or object construction requires references to objects
that are not immediately available, an identifier transformation must take place to
find an access path leading to the required object.

Identifier transformations are required wherever an identifier refers to an object
in an outer scope.
%Identifier transformations involve searching through parameters in enclosing
%definitions to find 
For example, to transform the expression \code{new C()} from within the body of
\code{D}, we must find access paths to the package references required by \code{C}.
%we must perform a compile-time search through the enclosing frame references
%to find the necessary package references.
The shortest paths to the necessary references result in the transformed
expression:
\begin{lstlisting}
new C(enclC.enclQ, enclC.enclP, enclC.packageRoot)
\end{lstlisting}
Other longer paths are possible from within \code{D}, e.g., \code{enclF.this.enclQ}.
It is not clear whether there could be any benefit from using a non-shortest access path.

\subsubsection{Closure transformation}

Let us redefine function \code{G} to create and call a closure of \code{H}:

\begin{lstlisting}
def G (): (() => Unit) = {   (*// return type is a 0-parameter function*)
	def H () = {
	}
	val h = H _      (*// underscore means partial evaluation*)
	h ()
}
\end{lstlisting}

The transformed signature of \code{H} includes the parameters
\code{enclG : FrameG} and \code{this : D}.
These parameters must be captured by the closure object at the site
where the closure is created:

\begin{lstlisting}
class ClosureH (enclG : FrameG, enclD : D) {
	def apply (this : ClosureH) = H (this.enclG, this.enclD)
}
def G (this : D): ClosureH = {
	val h = new ClosureH (myself, this)
	h.apply (h)
}
\end{lstlisting}

Unfortunately, there seems to be an inherent loss of information
where closures are involved.
%In the above example, \code{h} must be mutable if any of the partially-applied
%parameters are mutable, otherwise we may not be able to determine the purity
%of \code{h.apply}.
The type \code{ClosureH} can be passed anywhere
the type \code{()=>Unit} is required, making it impossible in general
to determine at the call site what \code{h.apply} is actually calling,
or with what partially-applied parameters.
We do, however, know whether \code{h} is mutable, which provides us with
enough information to determine whether any of the partially-applied parameters
may be mutable (although not precisely which of the parameters are mutable).

\subsection{Alternatives}

\subsubsection{Precision options in package parameters}

Instead of passing a \code{packageRoot} parameter,
we may pass a parameter for every possible package.
This approach may allow us to be more precise about which packages
a function may mutate.

Another alternative is to only provide the \code{packageRoot} parameter, and no
parameters for locally enclosing packages (\code{enclP} and \code{enclQ} in the above examples).

It is unknown whether changing the precision of package parameters
will provide us with any useful information in the analysis.

\subsubsection{Single-enclosing paramater option}

Instead of adding a parameter for each enclosing scope up to the nearest class,
we may add a parameter for the immediately-enclosing scope only.
In the above example, the transformed signature of \code{H} would merely
be:
\begin{lstlisting}
def H (enclG : FrameG)
\end{lstlisting}
During identifier transformation, a usage of \code{this} inside \code{H}
would become \code{enclG.this}, which involves a precision loss.

The precision loss occurs where \code{H} is mutable in \code{this} but
immutable in \code{enclG}. Because we must reach \code{this} through
\code{enclG}, we must force \code{enclG} to also be mutable.
It is unknown whether the precision loss significantly affects actual precision in practice.

\subsubsection{Package-level parameters for all definitions}

Reaching a package-level identifier requires finding an access path through all enclosing scopes.
For example, a mutable access to an aribitrary package from within function \code{H}
goes through the path \mbox{\code{this.this.packageRoot}}.
We are therefore unable to say that \code{H} mutates through a package-level variable,
but is immutable in \code{C} and \code{D}.

If we include package-level parameters, the transformed signature of \code{H} could be:
\begin{lstlisting}
def H (enclG : FrameG, this : D, enclQ : Q, enclP : P, packageRoot : Root)
\end{lstlisting}
It is unknown whether this modification provides any improvement
in actual precision.


% alternatives: single encl parameter only, single encl + root parameter, all scopes have parameters (don't stop the parameter list at classes)


%\bibliographystyle{alpha}
%\bibliographystyle{plain}
%\bibliographystyle{abbrv}
%\bibliography{../references}

\end{document}
