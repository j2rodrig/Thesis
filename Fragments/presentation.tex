\documentclass{beamer}

\usepackage{comment}
\usepackage[utf8]{inputenc}
\usepackage{listings}
\usepackage{multirow}
\usepackage{url}
\usepackage{xcolor}

\newcommand{\tab}[0]{\hspace{0.4cm}}

\renewcommand{\ttdefault}{txtt}

\lstset{
	basicstyle=\footnotesize\ttfamily,
	tabsize=4,
	extendedchars=true,
	breaklines=true,
	escapeinside={(*}{*)},
	%frame=l,  % trbl for single-line frames, TRBL for double-line frames
	%xleftmargin=1.5\parindent
}

\newcommand{\code}[1]{\lstinline$#1$}

\usetheme{CambridgeUS}

\begin{document}
	
	\title{Reference Immutability for Scala}
	\author{Jonathan Rodriguez}
	\institute[Univ. of Waterloo]{David R. Cheriton School of Computer Science\\University of Waterloo}
	%\institute{University of Waterloo}
	\date{July 18, 2014}
	\frame{\titlepage}
	
	\AtBeginSection[]
	{
	\begin{frame}
		\frametitle{Outline}
		\tableofcontents[currentsection]
	\end{frame}
	}

%introducing reference immutability
	%motivational example (in C and Java): constraints on a list
	%reference immutability as a constraint system: mutable <: polyread <: readonly ...
		%other binary constraints are possible...
	%RI as an ownership system: owners-as-modifiers
	%RI as a restriction on aliasing
	%RI and side effects. purity/concurrency: not quite strong enough, but a good start
	%Related work: ReIm (no generics), Efftp, Javari, JPure
%
	%viewpoint adaptation examples (Java? C?)
	%exactly what I'm doing and not doing (Javari features, but I'm only doing readonly)
	%generic types: (also to support arrays.) list example, example where const is appropriate.
	%local type inference
%challenges presented by Scala
	%principle: an RI annotation is allowed on any type expression: maximal re-use of Scala's existing type inference and checking
		% data types vs. RI types: A <: B if A's data type <: B's data type AND ...
	%principle: don't break legacy code (if it's not annotated, RI system has no effect)
	%principle: incremental annotation (perhaps like C++ const)
	%nested methods
		% extra parameters
	%closures
		% hidden extra parameters
%evaluation

\section{Understanding Reference Immutability}

	\subsection{Motivational Example}

\def\redcolor{\color{red}}
\def\greencolor{\color{green!60!black}}
\def\bluecolor{\color{blue}}
\def\blackcolor{\color{black}}

\begin{frame}[containsverbatim]
\frametitle{A List of Integers (in Java)}
Without annotations, arbitrary mutations of \code{list} are permitted:
\begin{lstlisting}
	class List {
		public int elem;
		public List tail;
	}
	List list = ...;
	list.elem = ...;          // Allowed
	list.tail = ...;          // Allowed
	list.tail.elem = ...;     // Allowed
	list.tail.tail = ...;     // Allowed
\end{lstlisting}
\end{frame}

\begin{frame}[containsverbatim]
\frametitle{A List of Integers (in C)}
Without annotations, arbitrary mutations of \code{list} are permitted:
\begin{lstlisting}
	struct List {
		int elem;
		struct List * tail;
	};
	struct List * list = ...;
	list->elem = ...;         // Allowed
	list->tail = ...;         // Allowed
	list->tail->elem = ...;   // Allowed
	list->tail->tail = ...;   // Allowed
\end{lstlisting}
\end{frame}

\begin{frame}[containsverbatim]
\frametitle{A List of Integers (in C)}
Adding ``\code{const}'' prohibits top-level mutations:
\begin{lstlisting}[escapechar=`]
	struct List {
		int elem;
		struct List * tail;
	};
	`\aftergroup\redcolor`const`\aftergroup\blackcolor` struct List * list = ...;
	list->elem = ...;         `\aftergroup\redcolor`// Not allowed`\aftergroup\blackcolor`
	list->tail = ...;         `\aftergroup\redcolor`// Not allowed`\aftergroup\blackcolor`
	list->tail->elem = ...;   // Allowed
	list->tail->tail = ...;   // Allowed
\end{lstlisting}
\end{frame}

\begin{frame}[containsverbatim]
\frametitle{A List of Integers (in C)}
A hypothetical ``\code{readonly}'' would prohibit mutations transitively:
\begin{lstlisting}[escapechar=`]
	struct List {
		int elem;
		struct List * tail;
	};
	`\aftergroup\redcolor`readonly`\aftergroup\blackcolor` struct List * list = ...;
	list->elem = ...;         `\aftergroup\redcolor`// Not allowed`\aftergroup\blackcolor`
	list->tail = ...;         `\aftergroup\redcolor`// Not allowed`\aftergroup\blackcolor`
	list->tail->elem = ...;   `\aftergroup\redcolor`// Not allowed`\aftergroup\blackcolor`
	list->tail->tail = ...;   `\aftergroup\redcolor`// Not allowed`\aftergroup\blackcolor`
\end{lstlisting}
\end{frame}

\begin{frame}[containsverbatim]
\frametitle{A List of Integers (in C)}
The ``\code{readonly}'' constraint cannot be violated by assignment:
\begin{lstlisting}[escapechar=`]
	struct List {
		int elem;
		struct List * tail;
	};
	readonly struct List * list = ...;
	struct List * alias_tail = list->tail;  `\aftergroup\redcolor`// Not allowed`\aftergroup\blackcolor`
	alias_tail->elem = ...;
	alias_tail->tail = ...;
\end{lstlisting}
\end{frame}

\begin{frame}[containsverbatim]
\frametitle{A List of Integers (in C)}
Or by a function argument:
\begin{lstlisting}[escapechar=`]
	struct List {
		int elem;
		struct List * tail;
	};
	void setElement(struct List * list, int value) {
		list->elem = value;
	}
	readonly struct List * list = ...;
	setElement(list, 0);        `\aftergroup\redcolor`// Not allowed`\aftergroup\blackcolor`
	setElement(list->tail, 0);  `\aftergroup\redcolor`// Not allowed`\aftergroup\blackcolor`
\end{lstlisting}
\end{frame}

\begin{frame}[containsverbatim]
\frametitle{A List of Integers (in C)}
Or by return value:
\begin{lstlisting}[escapechar=`]
	struct List {
		int elem;
		struct List * tail;
	};
	struct List * getTail(readonly struct List * list) {
		return list->tail;     `\aftergroup\redcolor`// Not allowed`\aftergroup\blackcolor`
	}
\end{lstlisting}
\end{frame}

	\subsection{Reference Immutability as a Type System}

\begin{frame}[containsverbatim]
\frametitle{Reference Immutability as a Type System}
	Enforcing \code{readonly} restricts the set of valid programs --\\
	\code{readonly} is therefore a \emph{type} or a \emph{constraint}.\\
\hfill\\
	Reference immutability types form a simple hierarchy:
\begin{center}
	\code{mutable <: polyread <: readonly}
\end{center}
%In addition to \code{readonly}, the following types are defined:
\begin{itemize}
\item \code{mutable} is the absence of \code{readonly}.
\item \code{polyread} is an \emph{unadapted type}.\\
	It must behave correctly in \code{mutable} or \code{readonly} contexts.\\
	(More on \code{polyread} later.)   % in viewpoint adapation section.
	%that may or may not be
	%treated as \code{readonly}, depending on context.
%Reference immutability types
\end{itemize}
\end{frame}

%\begin{frame}[containsverbatim]
%\frametitle{Reference Immutability as a Type System}
%\end{frame}


	\subsection{Other Perspectives}
	% ownership, owners-as-modifiers
		% (response to owners-as-dominators: heavy annotations, inflexible, requiring substantial refactoring)
	% aliasing restrictions
	% memory side-effect restrictions

\begin{comment}
\begin{frame}[containsverbatim]
\frametitle{Reference Immutability as an Ownership Type Discipline}
\emph{Ownership types} express patterns
of object \emph{ownership} or \emph{containment}.
\begin{itemize}
\item E.g., a list object may \emph{own} its elements.
\end{itemize}
\hfill\\
\hfill\\
Reference immutability belongs to the \emph{owners-as-modifiers}
discipline.
\begin{itemize}
\item Each object \emph{owns} objects reachable through non-\code{readonly} fields.
\item An object's methods may modify owned objects.
\item Unowned objects may be read, but not modified.
\item Ownership is not confined to a tree structure -- ownership may be cyclical.
\end{itemize}
\end{frame}
\end{comment}

\begin{frame}[containsverbatim]
\frametitle{Reference Immutability as Aliasing Control}
Reference immutability controls the effects of object aliasing.
\begin{itemize}
\item Aliases may be propagated anywhere, but not all aliases can be used to mutate objects.
\item Pure functional programming is one extreme -- all references are \code{readonly}.
\end{itemize}
Reference immutability is not object immutability.
\begin{itemize}
\item \emph{Object immutability} is a guarantee that an object cannot be modified after creation.
\item Reference immutability only guarantees that modifications cannot occur through specific references --
	object immutability is guaranteed where \emph{all} references to an object are \code{readonly}.
\end{itemize}
\end{frame}

\begin{frame}[containsverbatim]
\frametitle{Reference Immutability as Side Effect Control}
In-memory side effects may only occur through \code{mutable} references.
\begin{itemize}
\item If all of a function's parameters are \code{readonly}, then
	that function is \emph{pure} or \emph{side-effect free}.
\item Pure functions may still mutate freshly-created objects,
	but those objects are not visible to external observers.
\end{itemize}
\end{frame}


%--------- 8 MINUTES ---------%


\section{Features of Reference Immutability Systems}

	\subsection{Local Inference, Viewpoint Adaptation, Generic Types}
	%local inference for lightening the annotation burden
	%viewpoint adaptation for preventing code duplication (otherwise, would need readonly and non-readonly methods - code example)
	%generic types: to support, e.g., mutable arrays of readonly elements
	
	
\begin{frame}[containsverbatim]   %-- 1.5 min.
\frametitle{Local Inference}
Local type inference reduces the annotation burden.
\begin{itemize}
\item Particularly important for complex type systems like Scala's.
\end{itemize}
For example:
\begin{lstlisting}[escapechar=`]
	val readonly_list: List[Int @readonly] @readonly = ...
	val alias_list = readonly_list
\end{lstlisting}
\code{alias_list} should be given the same type as \code{readonly_list}.
\end{frame}

\begin{frame}[containsverbatim]   %-- 1 min.
\frametitle{Viewpoint Adaptation}
\emph{Viewpoint adapation} prevents code duplication when adding \code{readonly}.
\\\hfill\\
For example (in Java):
\begin{lstlisting}[escapechar=`]
	class List {
		public int elem;
		public List tail;
	}
	List getTail(List list) {
		return list.tail;
	}
\end{lstlisting}
\redcolor Problem: Can't call \code{getTail} with a \code{readonly} argument. \blackcolor
\end{frame}

\begin{frame}[containsverbatim]   %-- 1 min.
\frametitle{Viewpoint Adaptation}
\greencolor Option 1. Overload \code{getTail} \blackcolor
\begin{lstlisting}[escapechar=`]
	List getTail(List list) {
		return list.tail;
	}
	readonly List getTail(readonly List list) {
		return list.tail;
	}
\end{lstlisting}
\greencolor Option 2. Use generics \blackcolor
\begin{lstlisting}[escapechar=`]
	<T extends readonly> List<T> getTail(List<T> list) {
		return list.tail;
	}
\end{lstlisting}
\greencolor Option 3. Use \code{polyread} \blackcolor
\begin{lstlisting}[escapechar=`]
	polyread List getTail(polyread List list) {
		return list.tail;
	}
\end{lstlisting}
\end{frame}

\begin{frame}[containsverbatim]   %-- 1.5 min.
\frametitle{Viewpoint Adaptation}
\begin{lstlisting}[escapechar=`]
	polyread List getTail(polyread List list) {
		return list.tail;
	}
\end{lstlisting}
\begin{itemize}
\item \code{polyread} is replaced by a specific RI type at each call site.
\item The specific RI type is the least upper bound of all arguments applied
	to \code{polyread} parameters. For example:
\end{itemize}
\begin{lstlisting}[basicstyle=\scriptsize\ttfamily,escapechar=`]
	polyread List getOne(polyread List first, polyread List second) {
		if(first) return first.tail;
		else return second.tail;
	}
	getOne(readonly_list,mutable_list);  // returns readonly
	getOne(mutable_list,polyread_list);  // returns polyread
	getOne(mutable_list,mutable_list);   // returns mutable
\end{lstlisting}
\end{frame}

\begin{frame}[containsverbatim]   %-- 2 min.
\frametitle{Viewpoint Adaptation}
Viewpoint adaptation is also performed on fields.
Given a reference \code{list} to an object with field \code{tail}:
\begin{center}
\begin{tabular}{cccccc}
	& {\small Type of }\code{list} & $\triangleright$ & {\small Type of }\code{tail} & $\rightarrow$ & {\small Type of }\code{list.tail} \\
	\hline
	%\code{readonly} & $\triangleright$ & -- & $\rightarrow$ & \code{readonly} \\
	1. & (any) & $\triangleright$ & \code{readonly} & $\rightarrow$ & \code{readonly} \\
	2. & $q$ & $\triangleright$ & \code{mutable} & $\rightarrow$ & $q$ \\
	3. & $q$ & $\triangleright$ & \code{polyread} & $\rightarrow$ & $q$ \\
\end{tabular}
\end{center}
Rules 1 and 2 find the least-restrictive type that protects immutability guarantees.\\
Rule 3 performs viewpoint adaptation of fields.
\end{frame}

\begin{frame}[containsverbatim]   %-- 1 min.
\frametitle{Generic Types}
Reference immutability types may be used in generics:
\begin{lstlisting}[escapechar=`]
	class List[E] {
		var elem: E = ...
		var tail: List[E] = ...
	}
	val mm_list: List[Int] = ...
	val mr_list: List[Int @readonly] = ...
	val rm_list: List[Int] @readonly = ...
	val rr_list: List[Int @readonly] @readonly = ...
\end{lstlisting}
Some open questions, e.g.:
\begin{lstlisting}[escapechar=`]
	class List[E @readonly] {         `\aftergroup\redcolor`// Allow @readonly here?`\aftergroup\blackcolor`
		...
	}
	val mm_list: List[Int] = ...      `\aftergroup\redcolor`// Default @readonly?`\aftergroup\blackcolor`
	val mm_list: List[Int @mutable] = ... `\aftergroup\redcolor`// Allow @mutable?`\aftergroup\blackcolor`
\end{lstlisting}
\end{frame}


	\subsection{Other Features}
\begin{frame}[containsverbatim]   %-- 1 min.
\frametitle{Other Features (Not Implemented Here)}
\begin{itemize}
\item Whole-program inference (e.g., ReImInfer, JPure, Kneuss et al.).
\item Using the \code{mutable} keyword to selectively break transitive guarantees (Javari).
\item Dynamic checking of potentially-unsafe casts (Javari).
\item Serialization and reflection (Javari).
\end{itemize}
\end{frame}
	% NOT IMPLEMENTED:
	% whole-program inference (e.g., ReIm, JPure, Kneuss et al.)
	% removal of readonly via casting (Javari)


\section{Reference Immutability for Scala}

	\subsection{New Challenges: Nested Methods, Partially-applied Functions}

\begin{frame}[containsverbatim]
\frametitle{Nested Methods}
Scala allows nested methods. For example:
\begin{lstlisting}[escapechar=`]
	def f() = {
		var v = ...     // v is defined in the body of f
		def g() = {
			v = ...     // f.v is reassigned in g
		}
		g()
	}
\end{lstlisting}
\end{frame}

\begin{frame}[containsverbatim]
\frametitle{Nested Methods}
Effect annotations make the mutability of enclosing scopes explicit:
\begin{lstlisting}[escapechar=`]
	`\aftergroup\redcolor`@mutable(none)`\aftergroup\blackcolor` def f() = {
		var v = ...     // v is defined in the body of f
		`\aftergroup\redcolor`@mutable(f)`\aftergroup\blackcolor` def g() = {
			v = ...     // f.v is reassigned in g
		}
		g()
	}
\end{lstlisting}
\hfill\\
\hfill\\
\redcolor \code{@mutable(none)} \blackcolor means all outer scopes are \code{readonly}.\\
\redcolor \code{@mutable(f)} \blackcolor means the scope named \code{f} must be \code{mutable}.\\
Default: All outer scopes are mutable.
\end{frame}

\begin{frame}[containsverbatim]
\frametitle{Nested Methods}
Effect annotations are handled like extra parameters.\\
For illustration (does not compile):
\begin{lstlisting}[escapechar=`]
	def f(`\aftergroup\redcolor`package: @readonly`\aftergroup\blackcolor`) = {
		var v = ...     // v is defined in the body of f
		def g(`\aftergroup\redcolor`f: @mutable, package: @readonly`\aftergroup\blackcolor`) = {
			v = ...     // f.v is reassigned in g
		}
		g()
	}
\end{lstlisting}
\end{frame}

\begin{frame}[containsverbatim]
\frametitle{Partially-applied Functions}
Scala allows partial evaluation:
\begin{lstlisting}[escapechar=`]
	def f(): `\aftergroup\redcolor`(() => Unit)`\aftergroup\blackcolor` = {  // f returns a closure
		var v = ...
		def g() = {
			v = ...
		}
		return `\aftergroup\redcolor`g _`\aftergroup\blackcolor`      // _ means partial evaluation in Scala
	}
\end{lstlisting}
\end{frame}

\begin{frame}[containsverbatim]
\frametitle{Partially-applied Functions}
Partial evaluation creates an anonymous closure object
(with extra parameters shown):
\begin{lstlisting}[escapechar=`]
def f(package: @readonly): ((`\aftergroup\redcolor`this: @mutable`\aftergroup\blackcolor`) => Unit) = {
	var v = ...
	def g(f: @mutable, package: @readonly) = {
		v = ...
	}
	`\aftergroup\redcolor`class anonymousClosure(f: @mutable, package: @readonly) {
		def apply(this: @mutable) = g(this.f, this.package)
	}`\aftergroup\blackcolor`
	return new anonymousClosure()
}
\end{lstlisting}
Problem to solve:
\begin{itemize}
\item Loss of precision: Can't express that the closure only modifies \code{f}.
\end{itemize}
\end{frame}

\begin{frame}[containsverbatim]
\frametitle{Partially-applied Functions}
Possible annotations:
\begin{lstlisting}[escapechar=`]
	(() => Unit) @mutable(all)    // may mutate anything
	(() => Unit) @mutable(f)      // may mutate scope f
	(() => Unit) @mutable(none)   // mutates nothing (pure)
\end{lstlisting}
\code{@mutable(all)} is the top type, \code{@mutable(none)} is bottom.\\
Outside of \code{f}, \code{@mutable(f)} has no meaning. For example:
\begin{lstlisting}[escapechar=`]
	def f(): (() => Unit) @mutable(all) = {
		var v = ...
		@mutable(f) def g() = {
			v = ...
		}
		val closure_g: (() => Unit) @mutable(f) = g _
		return closure_g
	}
\end{lstlisting}
\end{frame}

\begin{frame}[containsverbatim]
\frametitle{Nested Classes}
Nested classes can be treated similarly to nested methods:
\begin{lstlisting}[escapechar=`]
	`\aftergroup\redcolor`@mutable(none)`\aftergroup\blackcolor` def f() = {
		var v = ...
		`\aftergroup\redcolor`@mutable(f)`\aftergroup\blackcolor` class G {
			v = ...
		}
		val g = new G()
	}
\end{lstlisting}
\code{@mutable(f)} is both a field of \code{G} and a parameter to \code{G}'s default constructor.
\end{frame}

\begin{frame}[containsverbatim]
\frametitle{Nested Classes}
Instances of nested classes can be returned:
\begin{lstlisting}[escapechar=`]
	@mutable(none) def f(): E = {
		var v = ...
		@mutable(f) class G extends E {
			v = ...
		}
		val g = new G()
		return g
	}
\end{lstlisting}
\code{G} is not visible outside \code{f}, but an instance can be returned if
it extends a class \code{E} visible from outside \code{f}.
\end{frame}

\begin{comment}
\begin{frame}[containsverbatim]
\frametitle{Partially-applied Functions}
\begin{lstlisting}[escapechar=`]
def f(package: @readonly):
		((`\aftergroup\redcolor`closure: @mutable, hidden: @mutable`\aftergroup\blackcolor`) => Unit) = {
	var v = ...
	def g(f: @mutable, package: @readonly) = {
		v = ...
	}
	`\aftergroup\redcolor`class anonymousClosure(f: @mutable, package: @readonly) {
		def apply(this: @mutable, f: @mutable) =
			g(f or this.f, this.package)
	}`\aftergroup\blackcolor`
	return `\aftergroup\redcolor`new anonymousClosure()`\aftergroup\blackcolor`
}
\end{lstlisting}
\end{frame}
\end{comment}

\begin{comment}
\begin{frame}[containsverbatim]
\frametitle{Nested Classes}
Nested classes are annotated much like nested methods:
\begin{lstlisting}[escapechar=`]
	`\aftergroup\redcolor`@mutable(none)`\aftergroup\blackcolor` def f() = {
		var v = ...     // v is defined in the body of f
		`\aftergroup\redcolor`@mutable(f)`\aftergroup\blackcolor` class G {
			`\aftergroup\redcolor`@mutable(this)`\aftergroup\blackcolor` def g() = {
				v = ...     // this.f.v is reassigned in g
			}
		}
	}
\end{lstlisting}
\redcolor\code{@mutable(this)} \blackcolor means that the innermost enclosing class (\code{G}) is mutable.
\end{frame}

\begin{frame}[containsverbatim]
\frametitle{Nested Classes}
Alternatively, \code{@mutable(f)} can be placed directly on \code{g}:
\begin{lstlisting}[escapechar=`]
	`\aftergroup\redcolor`@mutable(none)`\aftergroup\blackcolor` def f() = {
		var v = ...     // v is defined in the body of f
		class G {
			`\aftergroup\redcolor`@mutable(f)`\aftergroup\blackcolor` def g() = {
				v = ...     // this.f.v is reassigned in g
			}
		}
	}
\end{lstlisting}
A reference to \code{f} is implicitly available as a field of \code{G}.
\end{frame}
\end{comment}

\begin{comment}
\begin{frame}[containsverbatim]
\frametitle{Nested Classes}
%What do I want to say here?
Classes and methods can be arbitrarily nested:
\begin{lstlisting}[escapechar=`]
	class C() {
		def f(): E = {
			class D() extends E {
				def g() = {
				}
			}
			return new D()
		}
	}
\end{lstlisting}
\begin{lstlisting}[escapechar=`]
	def f(`\aftergroup\redcolor`package: @readonly`\aftergroup\blackcolor`) = {
		var v = ...     // v is defined in the body of f
		class G(`\aftergroup\redcolor`f: @mutable, package: @readonly`\aftergroup\blackcolor`) {
			def g(`\aftergroup\redcolor`this: @mutable`\aftergroup\blackcolor`) = {
				`\aftergroup\redcolor`this.f.`\aftergroup\blackcolor`v = ...     // f.v is reassigned in g
			}
		}
		
	}
\end{lstlisting}
\end{frame}
\end{comment}





\begin{comment}
\begin{frame}[containsverbatim]
\frametitle{Partially-applied Functions}
Partially-applied functions may mutate variables that are no longer in scope.
For example, let \code{f} return a partial application of \code{g}:
\begin{lstlisting}[escapechar=`]
	`\aftergroup\redcolor`@mutable(none)`\aftergroup\blackcolor` def f(): `\aftergroup\redcolor`(() => Unit)`\aftergroup\blackcolor` = {
		var v = ...     // v is defined in the body of f
		`\aftergroup\redcolor`@mutable(f)`\aftergroup\blackcolor` def g() = {
			v = ...     // f.v is reassigned in g
		}
		`\aftergroup\redcolor`return g _`\aftergroup\blackcolor`      // _ means partial application in Scala
	}
	val partial_g = f()
	partial_g()         // g() is executed here
\end{lstlisting}
\end{frame}

\begin{frame}[containsverbatim]
\frametitle{Partially-applied Functions}
Inside the compiler, partial application creates a closure object:
\begin{lstlisting}[escapechar=`]
	@mutable(none) def f(): (() => Unit) = {
		var v = ...     // v is defined in the body of f
		@mutable(f) def g() = {
			v = ...     // f.v is reassigned in g
		}
		`\aftergroup\redcolor`class anonymousClosure {
			def apply() = g()
		}
		return new anonymousClosure()`\aftergroup\blackcolor`
	}
	val partial_g = f()
	partial_g`\aftergroup\redcolor`.apply`\aftergroup\blackcolor`()   // g() is executed here
\end{lstlisting}
\end{frame}

\begin{frame}[containsverbatim]
\frametitle{Partially-applied Functions}
The \code{apply} method takes the same enclosing-scope parameters as \code{g}.

The \code{apply} method requires \code{@mutable(f)} because it calls \code{g}, and
the closure class requires \code{@mutable(f)} because
\begin{lstlisting}[escapechar=`]
	@mutable(none) def f(): (() => Unit) `\aftergroup\redcolor`@mutable(inner)`\aftergroup\blackcolor` = {
		var v = ...     // v is defined in the body of f
		@mutable(f) def g() = {
			v = ...     // f.v is reassigned in g
		}
		class anonymousClosure {
			`\aftergroup\redcolor`@mutable(f)`\aftergroup\blackcolor` def apply() = g()
		}
		val g_:(()=>Unit)@mutable(f) = new anonymousClosure()
		g_()
		return new anonymousClosure()
	}
	val partial_g = f()
	partial_g.apply()   // g() is executed here
\end{lstlisting}
\code{@mutable(f)} is replaced by \code{@mutable(inner)} when \code{f} returns.\\
\code{@mutable(inner)} means that the function mutates \emph{something},
	but information about \emph{what} has been lost\\
\end{frame}



\begin{frame}[containsverbatim]
\frametitle{Nested Methods}
\begin{lstlisting}[escapechar=`]
	def f(`\aftergroup\redcolor`package: @readonly`\aftergroup\blackcolor`) = {
		var v = ...     // v is defined in the body of f
		class G(`\aftergroup\redcolor`f: @mutable, package: @readonly`\aftergroup\blackcolor`) = {
			v = ...     // f.v is reassigned in g
		}
		new G()
	}
\end{lstlisting}
\end{frame}

\begin{frame}[containsverbatim]
\frametitle{Nested Methods}
\begin{lstlisting}[escapechar=`]
	class H(`\aftergroup\redcolor`package: @readonly`\aftergroup\blackcolor`) {
	}
	def f(`\aftergroup\redcolor`package: @readonly`\aftergroup\blackcolor`): H = {
		var v = ...     // v is defined in the body of f
		class G(`\aftergroup\redcolor`f: @mutable, package: @readonly`\aftergroup\blackcolor`) extends H = {
			v = ...     // f.v is reassigned in g
		}
		return new G()
	}
\end{lstlisting}
\end{frame}

\begin{frame}[containsverbatim]
\frametitle{Nested Methods}
\begin{lstlisting}[escapechar=`]
	class Closure(`\aftergroup\redcolor`package: @readonly`\aftergroup\blackcolor`) {
		def apply(`\aftergroup\redcolor`this: @readonly`\aftergroup\blackcolor`)
	}
	def f(`\aftergroup\redcolor`package: @readonly`\aftergroup\blackcolor`): Closure = {
		var v = ...     // v is defined in the body of f
		def g(`\aftergroup\redcolor`f: @mutable, package: @readonly`\aftergroup\blackcolor`) = {
			v = ...     // f.v is reassigned in g
		}
		class anonymousClosure(`\aftergroup\redcolor`f: @mutable, package: @readonly`\aftergroup\blackcolor`)
				extends Closure(`\aftergroup\redcolor`package: @readonly`\aftergroup\blackcolor`) {
			def apply(`\aftergroup\redcolor`this: @mutable`\aftergroup\blackcolor`) = g(this.f, this.package)
		}
		return new G()
	}
\end{lstlisting}
\end{frame}


%\begin{comment}

class C(package:@mutable) {
	def f(this:@mutable) = {
		class D(f:@mutable,this_C:@mutable) {
			val this_D:@mutable = ...;
			def g(this_D:@mutable) = {
			}
		}
	}
}

def f(package:@mutable,stdout:@mutable) = {
	def g(f:@mutable,package:@mutable,stdout:@mutable) = {
		def h(g:@readonly,f:@mutable,package:@mutable,stdout:@mutable) = {
			def apply(h:@readonly,g:@readonly,f:@mutable,package:@mutable,stdout:@mutable) = {}
		}
		h(g,f,package,stdout)
	}
}

def f(package:@mutable,stdout:@mutable) = {
	def g(f:@mutable,package:@mutable,stdout:@mutable) = {
		class H(g:@readonly,f:@mutable,package:@mutable,stdout:@mutable) {
			def apply(H:@mutable) = {}  // H.g, H.f, package, stdout
		}
		new H(g,f,package,stdout)
	}
}

def f(package:@mutable,stdout:@mutable) = {
	def g(f:@mutable,package:@mutable,stdout:@mutable) = {
		def h(g:@readonly,f:@mutable,package:@mutable,stdout:@mutable) = {
		}
		class anonymousClass
		h(g,f,package,stdout)
	}
}

@mutable(f,package)/*ctor*/ class C() @mutable(hidden)/*apply*/ {
	@mutable(f,D,package) def apply()
}

@mutable(hidden) <: @mutable(f) <: @mutable(none)

%\end{comment}



\begin{frame}[containsverbatim]
\frametitle{Nested Methods}
Scala permits arbitrary nesting of classes and methods. For example:
\begin{lstlisting}[escapechar=`]
	class C {
		def f() = {
			@mutable(f,C) def g() = {
				class D(g,f,C) extends E {
					@mutable(D)  means this
					@mutable(g)  means this.g
					@mutable(f)  means this.f
					@mutable(C)  means this.C
					
					@mutable(C,f,g,this) def apply() = ...
				}
				
				val d: E = new D(g,f,this)
				d.apply(d.C, d.f, d.g, d)  // we don't actually know what scopes are being closed over here.
				  // b/c the type E doesn't know about concrete extension D, or D's environment.
				
				@mutable(g)   means g
				@mutable(f)   means f
				@mutable(C)   means this
			}
			g(f, this)
		}
		@mutable(C)   means this
	}
\end{lstlisting}
\end{frame}

\begin{frame}[containsverbatim]
\frametitle{Nested Methods}
Scala permits arbitrary nesting of classes and methods. For example:
\begin{lstlisting}[escapechar=`]
	class C {
		def f() = {
			def g() = {
				class D {
				}
			}
		}
	}
\end{lstlisting}
\end{frame}
\end{comment}

	\subsection{Playing Nicely with Scala's Type System}

\begin{frame}
\frametitle{Playing Nicely with Scala's Type System}
Required changes to the type system:
\begin{itemize}
	%\item Setting a default RI type where annotations do not exist.
	%\item Modifying the subtyping relationship: \code{A} can only be a subtype of \code{B}
	%	 if the RI type of \code{A} is a subtype of the RI type of \code{B}.
	\item Modifying the \emph{subtype-of} relationship to include RI types.
	\item Modifying the RI type at dereference/select sites. (Transitive Typing)
	\item Modifying the returned RI type at call sites. (Viewpoint Adaptation)  % Also: interrupt default type checking for polyread parameters so that readonly arguments are compatible with polyread params. Overload/implicit resolution will also need to match polyread params.
	\item Setting default RI types on unannotated type expressions.
\end{itemize}
Changes that are \emph{not} required:
\begin{itemize}
\item Transforming the abstract syntax tree.
\item Changing the default order in which types are inferred by Scala.
\item Changing the inferred type of any AST node (except for RI types).
\end{itemize}
\end{frame}

\section{Evaluation and Future Work}
	%sanity check
	%code example?
	
	%safe constraint: a stronger (more restrictive) constraint on aliasing, useful for concurrency
	%I/O effects: ???

\begin{frame}
\frametitle{Evaluation Questions}
Does this Reference Immutability system for Scala:
\begin{itemize}
	\item allow for easy incremental annotation
	of existing code, or are there cases where large numbers of annotations must
	be added at the same time?
	\item achieve near-maximal RI typings with few annotations,
	or does most code have to be heavily annotated to achieve near-maximal typings?
	\item precisely express functional purity where purity exists,
	or is it too imprecise to express purity in some cases?
\end{itemize}
Answers to use examples from the standard library, compiler,
and other programs.
\end{frame}

\begin{frame}
\frametitle{Future Work}
Reference Immutability is an instance of a general constraint typing:
\begin{center}
	\code{unconstrained <: maybe_constrained <: constrained}
\end{center}
Sălcianu and Rinard (2005) infer \code{readonly} and a stronger constraint \code{safe}.
\begin{itemize}
\item \code{safe} merely adds a few extra aliasing restrictions.
\end{itemize}
\hfill\\
\hfill\\
Are there other useful transitive constraints?
\end{frame}

\begin{frame}
\frametitle{Future Work}
More questions:
\begin{itemize}
\item How to protect functions from \emph{observational exposure}?\\
	Pure functions can still observe side effects produced by impure functions,
	unless aliasing is further constrained.
\item Some mutations of an object's concrete state do not affect observable state (e.g., memoization).
	Javari allows programmers to declare that certain fields do not belong to observable state,
	but it does not offer any way of automatically checking this.
	Is there a way to \emph{guarantee} that the abstract state remains unchanged under certain mutations?

\end{itemize}
\end{frame}

\begin{comment}
\section{(Temporary)}

\defverbatim\ListingOne{%
\begin{lstlisting}
	class List {
		public int head = ...;
		public List tail = ...;
	}
	List list = ...;
	list.head = ...;
	list.tail = ...;
	list.tail.head = ...;
	list.tail.tail = ...;
\end{lstlisting}}

\begin{frame}[allowframebreaks]{MyListing}
	\ListingOne
\end{frame}

\begin{frame}[containsverbatim]
\begin{center}
\begin{tabular}{c|c}
\textbf{Java:} & \textbf{C:} \\
\hline
\begin{lstlisting}
class List {
	public int head;
	public List tail;
}
List list = ...;
list.head = ...;
list.tail = ...;
list.tail.head = ...;
list.tail.tail = ...;
\end{lstlisting}
&
\begin{lstlisting}
struct List {
	int elem;
	struct List * tail;
};
struct List * list = ...;
list->elem = ...;
list->tail = ...;
list->tail->elem = ...;
list->tail->tail = ...;
\end{lstlisting}
\end{tabular}
\end{center}
\end{frame}
\end{comment}

\begin{comment}
\begin{frame}[containsverbatim]
\begin{lstlisting}
struct List {
	int elem;
	struct List * tail;
};

readonly struct List * list = ...;

list->elem = ...;         // Not allowed
list->tail = ...;         // Not allowed
list->tail->elem = ...;   // Not allowed
list->tail->tail = ...;   // Not allowed
\end{lstlisting}
\end{frame}

	\defverbatim\ListingOne{%
	\begin{lstlisting}
	class List {
		public int head = ...;
		public List tail = ...;
	}
	List list = ...;
	list.head = ...;
	list.tail = ...;
	list.tail.head = ...;
	list.tail.tail = ...;
	\end{lstlisting}
	}
	
	\begin{frame}[allowframebreaks]{MyListing}
	\title{List Example}

	\begin{center}
	\begin{tabular}{c|c}
	%\ListingOne
	&
	Hi!
	\\
	\end{tabular}
	\end{center}
	
	\end{frame}

	% mutable <:   polyread   <: readonly
	%   no    <: undetermined <:  yes
	
	
	% side effects & purity
	
	% limitations: concurrency
	
	\begin{frame}[fragile]
	\end{frame}
\end{comment}
	
\end{document}