\documentclass[letterpaper,11pt]{article}

\usepackage{alltt}
\usepackage{amsmath}
\usepackage[greek,english]{babel}
%\usepackage{booktabs}
%\usepackage{cite}
%\usepackage{color}
\usepackage{fixltx2e}
\usepackage[T1]{fontenc}
%\usepackage{fullpage}
\usepackage[top=1in, bottom=1in, left=1in, right=1in]{geometry}
\usepackage{graphicx}
\usepackage[utf8]{inputenc}
\usepackage{listings}
%\usepackage{mathtools}
\usepackage{mdwlist}
%\usepackage{needspace}
\usepackage{url}
\usepackage{verbatim}
\usepackage{xspace}

% Fonts and symbols
%\usepackage{amsfonts}
%\usepackage{amssymb}
%\usepackage[scaled=.65]{beramono}
\usepackage{courier}
\usepackage{times}

\newenvironment{absolutelynopagebreak}
  {\par\nobreak\vfill\penalty0\vfilneg
   \vtop\bgroup}
  {\par\xdef\tpd{\the\prevdepth}\egroup
   \prevdepth=\tpd}

\newcommand{\greekmu}{\greektext m\latintext}

\newcommand{\tab}[0]{\hspace{0.4cm}}

\renewcommand{\ttdefault}{txtt}

\lstset{
	basicstyle=\footnotesize\ttfamily,
	tabsize=4,
	extendedchars=true,
	breaklines=true,
	escapeinside={(*}{*)},
	frame=l,  % trbl for single-line frames, TRBL for double-line frames
	xleftmargin=\parindent
}

\newcommand{\code}[1]{\lstinline$#1$}



\begin{document}

The purpose of this document is to propose an approach to transforming nested class and function
definitions into flat definitions that pass their environments by explicit parameters.
This transformation is a kind of lambda lifting or closure conversion.
The advantage we hope to gain from this transformation is a simplification of the analysis
that removes the distinctions between fields and local variables.
Describing the analysis in terms of the transformed code may also make the results
more easily generalizable to other languages.

The remainder of this section describes the transformation by example.
A formalization of the transformation may be performed at a later time.

\begin{comment}
Questions that may be worth discussing include:
\begin{itemize}
\item Are the proposed tranformation examples correct and preserving of all necessary semantics?
\item Do we care about the distinction between \code{val} and \code{var} in the analysis?
\item How can we prove what we need to prove about immutability?
\end{itemize}
\end{comment}

Class defintions are converted to frame defintions by
adding parameters that refer to outer scopes (including the package scope),
lifting the constructor(s) to top-level defintions,
and explicity allocating memory before calling the constructor.
We define the polymorphic function \code{alloc_frame[T]} to return a newly-allocated,
uninitialized space for an object of type \code{T}.
We introduce the \code{framedef} keyword to refer to lifted defintions,
reserving the use of \code{def} for unlifted defintions.

\begin{lstlisting}
class C(p: P) {
}
val c = new C(some_p)
\end{lstlisting}

becomes:

\begin{lstlisting}
framedef C(myself: C, package: Package, p: P) = {
	myself.p = p
}
package.c = alloc_frame[C]()
C(package.c, package, some_p)
\end{lstlisting}

%Note that constructor definitions return references to their own frames.
Parameters are assigned explicitly into the object frame, regardless of visibility
or mutability -- at a minimum, these parameters must be visible to inner definitions.

Function definitions are similarly lifted. Since there is no explicit type
for a function frame, we must create one. For example, for function \code{F}
we create a frame type \code{FrameF} that is large enough to hold all local variables
declared in \code{F}.

\begin{lstlisting}
def F(p: P): Q = {
	...
}
val q = F(some_p)
\end{lstlisting}

becomes:

\begin{lstlisting}
framedef F(myself: FrameF, package: Package, p: P): Q = {
	myself.p = p
	...
}
fframe = alloc_frame[FrameF]()
package.q = F(fframe, package, some_p)
\end{lstlisting}

\pagebreak

To handle nested classes and functions, we only add one additional parameter for
each enclosing scope.

\begin{lstlisting}
class C(p: P) {
	def F(p: P): Q = {
	}
	val q = F(p)
}
\end{lstlisting}

becomes:

\begin{lstlisting}
framedef F(myself: FrameF, outerC: C, package: Package, p: P): Q = {
	myself.p = p
}
framedef C(myself: C, package: Package, p: P) = {
	myself.p = p
	fframe = alloc_frame[FrameF]()
	myself.q = F(fframe, myself, package, myself.p)
}
\end{lstlisting}

It should be possible to determine that F is side-effect free when called from C,
because \code{fframe} is fresh.

%nested definitions
%closures g _
Functions can return closures.
Closures are represented as objects with an \code{apply} member,
which holds a reference to a function.
In the lifted code, the reference is unadorned -- its environment is
the closure itself, which is passed as an explicit parameter.
To make things extra complicated, we will allow the returned
closure to modify an enclosing scope's parameter.

\begin{lstlisting}
def F(var p: P): (Q => R) = {
	def G(q: Q): R = {
		p = something
	}
	G _   // the underscore denotes partial evaluation
}
val closure = F(some_p)
val r = closure(some_q)
\end{lstlisting}

becomes:

\begin{lstlisting}
framedef G(myself: FrameG, fframe: FrameF, package: Package, q: Q): R = {
	myself.q = q           // must be ro
	fframe.p = something   //
}
framedef ClosureG(myself: ClosureG, fframe: FrameF, package: Package) = {
	myself.fframe = fframe      // mu?
	myself.package = package
	myself.apply = ApplyClosureG
}
framedef ApplyClosureG(closure: ClosureG, q: Q): R = {
	gframe = alloc_frame[FrameG]()
	return G(gframe, closure.fframe, closure.package, q)
}
framedef F(myself: FrameF, package: Package, p: P): ClosureG = {
	myself.p = p
	closure = alloc_frame[ClosureG]()
	ClosureG(closure, myself, package)
	return closure
}
fframe = alloc_frame[FrameF]()
package.closure = F(fframe, package, some_p)
package.r = package.closure.apply(package.closure, some_q)
\end{lstlisting}

%Since a closure's \code{apply} method never does anything except call another
%method, we shouldn't need to allocate another local frame when we call it.

\begin{comment}
An example of a pure constructor that calls the closure returned by~\code{F}:

\begin{lstlisting}
class D {
	val some_p = new P
	val callsG = F(some_p)
	val r = callsG(some_q)      // modifies some_p
}
\end{lstlisting}

becomes:

\begin{lstlisting}
framedef D(myself: D, package: Package): D = {
	myself.some_p = P(alloc_frame[FrameP](), ...)
	myself.callsG = F(alloc_frame[FrameF](), package, myself.some_p)
	myself.r = callsG(some_q)   // modifies some_p
	return myself
}
\end{lstlisting}

The above example may look pathological, but at the present time,
I have no reason to think that something like it is unusual.
\end{comment}

%\bibliographystyle{alpha}
%\bibliographystyle{plain}
%\bibliographystyle{abbrv}
%\bibliography{../references}

\end{document}
