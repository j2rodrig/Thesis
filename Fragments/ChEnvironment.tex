\section{Accessing Enclosing Environments}

The determination of method purity depends on more than just the types of explicit parameters. In Dotty, Java, and other object-oriented languages, every method has an implicit parameter (usually {\cd this}) that refers to the method's receiver object. For a method to be pure, none of its parameters (including the receiver parameter) can be mutable.

Methods and classes can be arbitrarily nested in Dotty, and methods can read from and write to fields and local variables declared in any enclosing scope.

In contrast, Java has much tighter restrictions on what nested classes are allowed to do. Although previous work on reference immutability offers satisfactory solutions given the restrictions of Java, there is not (to my knowledge) any prior work that adequately addresses the difficulties that arise from the relatively unrestricted nesting rules of Dotty.



\begin{comment}
2. traits have uniquely-named outer-reference fields.
3. methods are defined as classes (logic goes in the constructor, locals become fields,)
4. calls to methods become pairs of construct / field read.
5. the `this' read becomes a read of an outer field
6. construction of classes involves passing the outer reference as a param
Order of transformation: Innermost methods first?
\end{comment}

\subsection{Representation of Method Definitions as Class Definitions} \label{sec:methdefs-as-classdefs}

The first step I take is to represent method definitions as class definitions. This change in representation helps define what happens semantically when (e.g.) a nested class refers to local variables or parameters of an enclosing method. Furthermore, by representing method definitions as class definitions, the number of distinct cases that must be addressed later in the discussion is reduced. The latter discussion only needs a treatment of the class-inside-class case---rather than discussing method-inside-method, method-inside-class, class-inside-method, and class-inside-class as separate cases.

A key insight here is that a method's local variables need not be stored on the stack; the high-level semantics of the language are preserved regardless of whether local variables exist on the stack or the heap. In fact, in some cases it it necessary to store local variables on the heap---in particular, heap storage is necessary when those local variables remain visible to (and assignable by!) an inner class instance after the method returns.

A second key insight is that object construction is more powerful than ordinary method invocation. Object construction involves both an allocation of heap memory and an invocation of an initializer method. Like ordinary methods, initializer methods can execute arbitrary computations. The key semantic distinction is that ...

Ordinary method invocation also allocates memory, typically on the stack. The key semantic distinction is that an object survives the completion of its initializer, whereas the memory allocated for an ordinary method call is typically lost immediately upon completion. It is therefore reasonably straightforward to use object construction to simulate ordinary method invocation, but not vice versa.

Consider the general method form shown in listing~\ref{lst:meth-1}. Method~{\cd m} inside class~{\cd C} takes a series of parameters~{\cd \ovr{x}} and returns a result of type~{\cd T}. (The overline notation~{\cd \ovr{x}} means that~{\cd x} can represent any of a series of a series of names, similar to the overline notation in Featherweight Java.) Method~{\cd m} initializes a set of local variables~{\cd \ovr{y}} to the results of a set of expressions~{\cd \ovr{e}}. The result of~{\cd m} is the final expression~{\cd e}. (In practice, {\cd m} may also contain some number of statements, which I omit here.)

Also shown in listing~\ref{lst:meth-1} is a call to~{\cd m}. A series of expressions~{\cd \ovr{e}} are evaluted, and their results bound to the parameters~{\cd \ovr{x}}. The result is assigned to the field~{\cd y}.

\begin{lstlisting}[float=htbp, caption={Method Transformation 1}, label={lst:meth-1}]
class C {
	def m((*\ovr{x}*): (*\ovr{S}*)): T = {
		var (*\ovr{y}*) = (*\ovr{e}*)
		e
	}

	var y = m((*\ovr{e}*))
}
\end{lstlisting}

Adding an explicit {\cd this} reference, the definition and call of~{\cd m} are as shown in listing~\ref{lst:meth-2}. The type of {\cd this} in~{\cd m} is~{\cd C}, and \mbox{{\cd C}'s} {\cd this} is passed explicitly in the call to~{\cd m}.
%Also shown is an explicit declaration of {\cd this} as a self reference to the current object of class~{\cd C}.
\footnote{Using {\cdf this} as the name of a method parameter is not allowed in Dotty. The code in listing~\ref{lst:meth-2} is for explanatory purposes, and won't compile as-is.}

\begin{lstlisting}[float=htbp, caption={Method Transformation 2 (Explicit This)}, label={lst:meth-2}]
class C {
	def m(this: C, (*\ovr{x}*): (*\ovr{S}*)): T = {  // explicit ``this'' parameter
		var (*\ovr{y}*) = (*\ovr{e}*)
		e
	}

	var y = m(this, (*\ovr{e}*))           // explicit pass of ``this'' to method
}
\end{lstlisting}

The key transformation is transformation of the method definition into a class definition. See listing~\ref{lst:meth-3}.
The parameters of method~{\cd m} become parameters of the constructor of class {\cd \und M}. (I use the underscore \mbox{({\cd \und})} to distingush synthetic names from programmer-specified names. Names beginning with the underscore are assumed to be non-conflicting with any other names.)
The parameters and local variables of~{\cd m} become fields of~{\cd \und M}, which makes these parameters and local variables accessible to inner methods and classes.
All computations performed by~{\cd m} are performed by the constructor of~{\cd \und M}, and reads and writes of local variables become reads and writes of fields.
The result of~{\cd m} is stored in the field {\cd \und result}.

\begin{lstlisting}[float=htbp, caption={Method Transformation 3 (Closure)}, label={lst:meth-3}]
class C {
	class _M(_this: C, (*\ovr{\und x}*): (*\ovr{S}*)) {  // method parameters become class parameters
		val _outer = _this         // outer reference stored as field
		val (*\ovr{x}*) = (*\ovr{\und x}*)                 // parameters stored as fields
		var (*\ovr{y}*) = (*\ovr{e}*)                  // local variables stored as fields
		val _result: T = e         // result is stored for post-construction retrieval
	}

	var y = (new _M(this, (*\ovr{e}*)))._result  // method call becomes constructor call
}
\end{lstlisting}

The method call in listing~\ref{lst:meth-2} becomes a constructor call in listing~\ref{lst:meth-3}. 

% 

\subsection{``This'' References Translate To Outer-Accessor Paths}

In Dotty, the keyword {\cd this} is always understood to have a classname prefix. For example, in listing~\ref{lst:this-classname-prefix}, the {\cd this} keyword by itself refers to an instance of the innermost enclosing class~{\cd D}. So {\cd this} by itself is equivalent to \mbox{\cd D.this}.

By giving a different classname as a prefix, {\cd this} can be used to access an enclosing environment. Again in listing~\ref{lst:this-classname-prefix}, \mbox{\cd C.this} refers to an instance of the enclosing class~{\cd C}.

\begin{lstlisting}[float=htbp, caption={``This'' with Classname Prefix}, label={lst:this-classname-prefix}]
class C {
	class D {
		def m() = {
			this    // same as D.this
			D.this  // refers to an instance of enclosing class D
			C.this  // refers to an instance of enclosing class C
		}
	}
}
\end{lstlisting}

Access to instances of outer classes is mediated by fields containing references to those instances. Since classes may be nested to an arbitrary depth, a single use of {\cd this} may translate to an arbitrarily-long access path. The Dotty compiler internally handles the translation of {\cd this} into outer-accessor paths. The translation of listing~\ref{lst:this-classname-prefix} may produce something like listing~\ref{lst:this-outer-accessor} (shown with explicit receiver references).

\begin{lstlisting}[float=htbp, caption={``This'' to Outer-Accessor Path Translation}, label={lst:this-outer-accessor}]
class C {
	class D(_enclosing: C) {
		val _outer = _enclosing

		def m(_this: D) = {
			_this         // D.this
			_this._outer  // C.this
		}
	}
}
\end{lstlisting}

The {\cd \und enclosing} reference of class~{\cd C} is a synthetic parameter of the initializer of class~{\cd D}, which is stored in the synthetic {\cd \und outer} member of class~{\cd D}. (Synthetic parameters and members are named such that they do not conflict with other synthetic or non-synthetic names.) The method~{\cd m} is able to access the enclosing instance by reading the {\cd \und outer} member.

Note that when an instance of class~{\cd D} is created, it must be passed an implicit reference to the enclosing class~{\cd C}. The reference denoted by the prefix used to select~{\cd D} is the same reference passed to the initializer of~{\cd D}. See listing~\ref{lst:new-path-dep}, which shows the construction of an object of class~{\cd D} from within the initializer of~{\cd C}. The reference to the enclosing environment of~{\cd D} from within the initializer of~{\cd C} is merely \mbox{\cd C.this}.

\begin{lstlisting}[float=htbp, caption={Path Dependence and Enclosing References}, label={lst:new-path-dep}]
class C {
	class D(_enclosing: C) {
	}
	// The following is equivalent to new D().
	// D is reached through path C.this.
	val _enclosing = C.this
	new _enclosing.D(_enclosing)
}
\end{lstlisting}

In general, all constructible types must be reducible to the form \mbox{\cd x.D} where~{\cd x} is a valid term reference and {\cd D} is a class. The same~{\cd x} is the enclosing-object reference passed to the initializer of~{\cd D}.


%\subsection{Viewpoint Adaptation and Environment Access Paths}

%The rationale for the foregoing discussion of environment references is to 

% extends ...

\subsection{Environment Access and Inheritance}

Listing~\ref{lst:single-base-env-ref} shows a class~{\cd D} that inherits from another class~{\cd C}. Classes {\cd C}~and {\cd D}~in this listing share a common enclosing environment {\cd Env}. The constructor of~{\cd D} requires a reference to an object of class {\cd Env}, which it passes directly to the constructor of~{\cd C}. If~{\cd C} had a different enclosing class than~{\cd D}, then the path from {\cd \und enclosing} to the environment of~{\cd C} would need to be passed to the constructor of~{\cd C}. 



\begin{lstlisting}[float=htbp, caption={Single Base Class with Environment Reference}, label={lst:single-base-env-ref}]
class Env {
	class C(_enclosing: Env) {
		val _outer_C = _enclosing
	}
	class D(_enclosing: Env) extends _enclosing.C(_enclosing) {
		val _outer_D = _enclosing
	}
}
\end{lstlisting}

Inheritance is not limited to classes.
Listing~\ref{lst:base-class-trait-env-ref} shows a class~{\cd D} that inherits from another class~{\cd C} and a trait~{\cd L}. What is distinctive about traits (versus classes) here is that trait constructors cannot take arguments. Trait members are {\em linearized}---the inheriting class~{\cd D} contains all members of~{\cd L}, including \mbox{{\cd L}'s} environment reference. It is the job of~\mbox{{\cd D}'s} constructor to initialize {\cd \und outer\und L} to the same reference used to reach the definition of~{\cd L} (which, in this case, is {\cd \und enclosing}).

\begin{lstlisting}[float=htbp, caption={Base Class and Trait with Environment References}, label={lst:base-class-trait-env-ref}]
class Env {
	trait L {
		val _outer_L: Env
	}
	class C(_enclosing: Env) {
		val _outer_C = _enclosing
	}
	class D(_enclosing: Env) extends _enclosing.C(_enclosing) with _enclosing.L {
		val _outer_L = _enclosing
		val _outer_D = _enclosing
	}
}
\end{lstlisting}

Environment references in derived classes must be checked for compatibility with the environment references in base classes/traits. Unlike ordinary type checking, the default ``checking'' of environment reference types in the Dotty compiler is not really type checking at all. Rather, the Dotty compiler performs a path dependent lookup of each base class/trait name. Since a successful lookup requires reachability of the base class/trait's environment, a successful lookup entails environment-reference compatibility.

However, the addition of mutability permissions means that derived-class environment-reference types may now be incompatible with base-class/trait environment-reference types, even if the lookups of those bases are successful. Two extra steps are required. First, viewpoint adaptation must be performed along the path used to reach each base class/trait. Second, the mutability of each viewpoint-adapted path must be checked for compatibility with the base class/trait's environment-reference type.

Consider the definition of class~{\cd D} in listing~\ref{lst:base-traits-env-ref}. Class~{\cd D} extends a sequence of traits {\cd L1} ... {\cd Ln}, which are reached through the respective paths {\cd x1} ... {\cd xn}. In general, the type given for any {\cd extends} clause may be represented as the intersection of some sequence of base traits or classes.

\begin{lstlisting}[float=htbp, caption={Base Traits with Environment References}, label={lst:base-traits-env-ref}]
class Env {
	class D(_enclosing: M & ReadonlyNothing | Env)
		extends x1.L1 with x2.L2 ... with xn.Ln
	{
		val _outer_L1 = x1
		...
		val _outer_Ln = xn
		val _outer_D = _enclosing
	}
}
\end{lstlisting}

In listing~\ref{lst:base-traits-env-ref}, \mbox{{\cd D}'s} {\cd \und enclosing} parameter has a type with mutability~{\cd M}.
Each path {\cd x1} ... {\cd xn} is rooted at {\cd Env.this}, which is equivalent to {\cd \und enclosing} (as seen from within the definition of {\cd D}).
Viewpoint adaptation must be performed along each path {\cd xi}, and the resulting type must be compatible with the type of the corresponding environment reference {\cd\und outer\und Li}.



\subsection{Polymorphic Environment References}

% What about the constraint solver? Consider:
% 	def m[M](x: M): M
%  When callling m, the fact that M is part of the result type puts a constraint on M.
%	var y: A
%	y = m(...)   // constraint: M <: A
%

% For the below, the "result" field x has type U | M & ReadonlyNothing, yielding the constraint (although not solvable as-is):
%	U | EnvRef & ReadonlyNothing <: A

% Reasoning about envrionment-reference bounds:
% If no bounds are declared, then bounds are maximal and overriding of definitions becomes a non-issue. The construction EnvRef & ReadonlyNothing | C means that any type can be substituted for EnvRef, and the type remains valid.

Listing~\ref{lst:poly-env-ref} shows a class~{\cd D} with a type parameter {\cd EnvRef}. The type {\cd EnvRef} is abstract, with a lower bound of {\cd Nothing} and an upper bound of (readonly)~{\cd Any}.
The type of the parameter {\cd \und enclosing} and the type of field~{\cd x} are both viewpoint-adapted with {\cd EnvRef}. The viewpoint adaptation allows field~{\cd x} to flexibly store the result of a read from the enclosing environment; the type of {\cd x} is mutable where the environment is mutable, and readonly where the environment is readonly.

\begin{lstlisting}[float=htbp, caption={Polymorphic Environment Reference}, label={lst:poly-env-ref}]
class C {
	class D[EnvRef](_enclosing: EnvRef & ReadonlyNothing | C) {
		val x: EnvRef & ReadonlyNothing | T = ...  // T is an arbitrary type
	}
}
\end{lstlisting}
% 

The instantiation of {\cd D} involves setting {\cd EnvRef} to the type of the reference used to reach~{\cd D}.

% Type {\cd EnvRef} can be used within~{\cd D} to support typing of arbitrary viewpoint adaptations. The field~{\cd f} in figure~\ref{lst:poly-env-ref} has a type {\cd U} that is viewpoint-adapted with the prefix type {\cd EnvRef}. The viewpoint adaptation with {\cd EnvRef} allows field~{\cd f} to store the result of a read from the enclosing environment.

The transformation of polymorphic method definitions into class definitions (sec.~\ref{sec:methdefs-as-classdefs}) may result in a class definition like that shown in listing~\ref{lst:poly-env-ref}. Under such a transformation, the result of the method is stored as a field, so it is important that fields be able to store viewpoint-adapted results.

In general, consider the transformation of polymorphic methods with the form shown in listing~\ref{lst:poly-meth-1}. Method~{\cd m} has type parameters~{\cd \ovr{R}} in addition to ordinary parameters~{\cd \ovr{x}}. Each type parameter has a lower bound~{\cd L} and an upper bound~{\cd U}. The result type~{\cd T} may depend on any or all type parameters~{\cd \ovr{R}} (or type members of any~{\cd \ovr{x}}).

\begin{lstlisting}[float=htbp, caption={Polymorphic Method Transformation 1}, label={lst:poly-meth-1}]
class C {
	def m[(*\ovr{R}*) >: (*\ovr{L}*) <: (*\ovr{U}*)]((*\ovr{x}*): (*\ovr{S}*)): T = {
		var (*\ovr{y}*) = (*\ovr{e}*)
		e
	}
}
\end{lstlisting}

Listing~\ref{lst:poly-meth-2} adds an explicit receiver parameter {\cd this}. In contrast to the monomorphic methods discussed in section~\ref{sec:methdefs-as-classdefs}, the receiver type is the first type parameter {\cd EnvRef}.

\begin{lstlisting}[float=htbp, caption={Polymorphic Method Transformation 2 (Explicit This)}, label={lst:poly-meth-2}]
class C {
	def m[EnvRef >: Nothing <: Any, (*\ovr{R}*) >: (*\ovr{L}*) <: (*\ovr{U}*)](
		this: EnvRef & ReadonlyNothing | C, (*\ovr{x}*): (*\ovr{S}*)): T =
	{
		var (*\ovr{y}*) = (*\ovr{e}*)
		e
	}
}
\end{lstlisting}

Listing~\ref{lst:poly-meth-3} converts the method definition~{\cd m} into the class definition~{\cd \und M}. Type parameters of the method become type parameters of the class. The field {\cd \und result} has type~{\cd T}, an arbitrary type that may depend on any type parameter (including {\cd EnvRef}).

\begin{lstlisting}[float=htbp, caption={Polymorphic Method Transformation 3 (Closure)}, label={lst:poly-meth-3}]
class C {
	class _M[EnvRef >: Nothing <: Any, (*\ovr{R}*) >: (*\ovr{L}*) <: (*\ovr{U}*)](
		_this: EnvRef & ReadonlyNothing | C, (*\ovr{\und x}*): (*\ovr{S}*))
	{
		val _outer = _this         // outer reference stored as field
		val (*\ovr{x}*) = (*\ovr{\und x}*)                 // parameters stored as fields
		var (*\ovr{y}*) = (*\ovr{e}*)                  // local variables stored as fields
		val _result: T = e         // result is stored for post-construction retrieval
	}
}
\end{lstlisting}



\subsection{Practical Viewpoint Adaptation and Environment References}

% show how adaptation of vars/fields/this reduces to a simple scope-stepping problem.

\subsection{Covariance of Environment-Reference Type Parameters}

Environment-reference type parameters may be covariant.
Covariance allows (for example) a class~{\cd D} with a mutable environment reference to be a subtype of~{\cd D} with a readonly environment reference.

\begin{figure}[htbp]
\begin{lstlisting}
Given:
------
class D[+EnvRef](_enclosing: EnvRef & ReadonlyNothing | C) { ... }

We have:
--------
D[MutableAny]  <:  D[M]  <:  D[Any]

(*(For arbitrary mutability type M. Path to D omitted.)*)
\end{lstlisting}
\caption{Covariant Environment-Reference Type Example}
\label{fig:cov-env-ref}
\end{figure}

Covariance of environment-reference types prevents those types from being used in contravariant positions.
Contravariant positions include method parameter types and assignable ({\cd var}) field types.
% Logically, preventing use in contravariant positions is reasonable because environment references 

TODO: Determine if preventing EnvRef use in assignable vars is a practical problem or not.

Note: The EnvRef param of a method is invariant (so is not subject to variance restrictions). Methods cannot inherit like classes, so any variance issues occur only where the original definition was a class.
Note 2: Furthermore, traits do not take type parameters, and arbitrary type members are always invariant. Variance issues belong only to classes, and (realistically) only to assignable fields of classes.

Method results, non-assignable ({\cd val}) field types, and constructor parameters may use environment-reference types.


% \subsection{Package Hierarchies and Environment References}

\begin{comment}

\subsection{Viewpoint Adaptation of Parameterless Methods}

% Problem: Fields are viewpoint-adapted. What needs to be done with overriding parameterless methods?

% Q: In general, is there a problem with overriding class definitions with environment-reference type parameters? A: Yes, if their lower bounds are not in the range MutableAny ... Any. The previous solution of (e.g.) C .. C | ReadonlyNothing would cause problems if the envref C is different between overriding and overriden classes.

Dotty allows field definitions to override 

\begin{lstlisting}[float=htbp, caption={Field Definition Overriding Parameterless Method Definition}, label={lst:field-overriding-method}]
class Env {
	class C {
		def x: T = ...
	}
	class D extends C {
		override val x: T = ...
	}
}
\end{lstlisting}

\begin{lstlisting}[float=htbp, caption={Synthetic Accessor Overriding Parameterless Method Definition}, label={lst:synthetic-accessor-overriding-method}]
class Env {
	class C {
		def x[EnvRef_x](this: C): EnvRef_x & ReadonlyNothing | T = ...
	}
	class D extends C {
		private val _x: T = ...
		override def x[EnvRef_x](this: EnvRef_x & ReadonlyNothing | D)
			: EnvRef_x & ReadonlyNothing | T =
		{
			this._x
		}
	}
}
\end{lstlisting}

% override contravariance of this: C <: EnvRef_x .. OK
% override covariance of result: EnvRef_x & ReadonlyNothing | T <: T .. Not OK


\subsection{Viewpoint Adaptation of Environment References in Practice}

% locals as fields (examples)
% super

\subsection{Inheritance and Environment References in Practice}

% Base classes must have envrionment-reference mutabilities that are compatible with the derived class's env-ref mutability.

The mutabilities of 

If a base class definition is reachable from the derived class,

The addition of mutability permissions to types means that 

The types of environment references are not computed during the typer phase of the compiler. 

\subsection{}

% packages as outer classes
% accessor methods and overriding





Some translation is performed on constructor calls to handle synthetic outer references, and this translation is related to path-dependent types. In Dotty, types are {\em path-dependent}. The class name {\cd D} by itself is not a type; where~{\cd D} is used without a prefix path, an appropriate prefix path is assumed. (typically \mbox{\cd C.this}, where~{\cd C} is )

In general, for any class~{\cd D} that 

\begin{lstlisting}[float=htbp, caption={}, label={lst:}]
class C {
	class D(_enclosing: C) {
	}
	new D()
	new C.this.D()  // with explicit path prefix
	new C.this.D(C.this)  // with explicit outer reference
}
\end{lstlisting}




\begin{lstlisting}[float=htbp, caption={Method Transformation 4 (Flattening)}, label={lst:meth-4}]
class C { this =>
	val y = (new _M(this, x))._result
}
class _M(this: C, _x: S) = {
	val _outer = this
	val x = _x
	...
	val _result = resultExpr
}
\end{lstlisting}

\end{comment}
