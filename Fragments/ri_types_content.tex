
\section{Reference Immutability Types}

Reference Immutability (RI) Types are qualifiers on ordinary data types.
The basic RI types we use here are \emph{readonly}, \emph{mutable}, and \emph{polyread}.
RI types specify deep (or transitive) write permissions.
A \emph{readonly} type disallows writes to the referenced object, and transitively
all objects referenced by fields of that object.
A \emph{mutable} type does not place any restrictions on the use of referenced objects --
for programs with no RI types, all references are assumed to be mutable.
A \emph{polyread} type indicates that readonly restrictions may or may not apply,
depending on context. (\emph{Polyread} is discussed in detail below.)

RI types form a hierarchy.
\emph{Mutable} types can be safely cast to \emph{polyread},
and \emph{polyread} types can be safely cast to \emph{readonly}.
Briefly, the subtyping relationship is:

\begin{center}
	mutable <: polyread <: readonly
\end{center}

The transitive nature of RI types means that readonly restrictions must be
propagated along access paths.
In particular, any field accessed through a \emph{readonly} reference must
be treated as \emph{readonly}, and a \emph{mutable} field accessed through
a \emph{polyread} reference is treated as \emph{polyread}.
If \emph{polyread} fields are accessed through \emph{mutable} references,
they are treated as (or adapted to) \emph{mutable}.
Definition~\ref{accesspath-ri-def} summarizes the RI type rule for access paths.

%The transitive nature of RI types means that readonly restrictions must be
%propagated along access paths -- a reference that is readonly along one
%access path may be mutable when observed along a different access path.
%The following table lists all possible combinations of RI types for
%a given access path~\code{h.k}:
%We define the RI type of an access path to be at least as restrictive 
%as any element of that path.
%(Unlike Javari~\cite{javari}, we do not allow the \emph{mutable} RI type to
%violate readonly restrictions.)

\begin{definition}
	\label{accesspath-ri-def}
	Let \code{h} be an access path to some object~$O$, and let~$r_h$
	be the Reference Immutability (RI) type of~\code{h}.
	Furthermore, let~\code{k} be a field of $O$ with an RI type of~$r_k$.
	The RI type of the access path~\code{h.k}
	%Given an identifier~\code{h} and an access path~\code{k}
	%with Reference Immutability (RI) Types~$r_h$ and~$r_k$, respectively,
	%the RI type of the combined access path~\code{h.k}
	is $r_h$ if $r_k$
	is \emph{polyread}, else the RI type of~\code{h.k} is
	the least upper bound of $r_h$ and $r_k$.
\end{definition}

The complete list of RI type combinations for access paths is:

\begin{center}
	\begin{tabular}{cc|c}
	RI Type of \code{h} & RI Type of \code{k} & RI Type of \code{h.k} \\
	\hline
	mutable & mutable & mutable \\
	mutable & polyread & mutable \\
	mutable & readonly & readonly \\
	polyread & mutable & polyread \\
	polyread & polyread & polyread \\
	polyread & readonly & readonly \\
	readonly & mutable & readonly \\
	readonly & polyread & readonly \\
	readonly & readonly & readonly \\
	\end{tabular}
\end{center}



RI types can be used in generic contexts. For example,
an Iterator class may be declared covariant in its element type,
so the IR type of the element is also covariant.
The programmer can declare that such an iterator is \emph{mutable}, but
contains \emph{readonly} elements~-- the iterator object itself
can be modified, but elements returned by the iterator cannot be modified.
%Elements can be added or removed from this iterator, but elements retrieved
%from the list cannot be modified.
The iterator is a subtype of a \emph{polyread} or \emph{readonly} iterator
with \emph{readonly} elements, but not a subtype of any iterator
with non-\emph{readonly} elements.

%Covariance, contravariance, and invariance are already supported by Scala.
%To the extent possible, RI types adhere to 

The purpose of the \emph{polyread} type is for \emph{viewpoint adaptation}.
Viewpoint adaptation allows a method's return type to vary (or be adapted) depending on
the types of its parameters. For example, consider a basic getter function
(shown with an explicit receiver \code{this}):
\begin{lstlisting}
class C {
	private[this] var x: X = ...
	def get (this: C @polyread): X @polyread = this.x
}
val ro_c: C @readonly = new C
val mu_c: C = new C
val ro_x: X @readonly = ro_c.get()   (*// readonly this yields readonly x*)
val mu_x = mu_c.get()                (*// mutable this yields mutable x*)
\end{lstlisting}
The return type of \code{C.get} must be \code{@polyread} to avoid code duplication.
Otherwise, we would require the programmer to create two versions of \code{C.get}: one returning
\code{@readonly} and the other returning \code{@mutable}.

Viewpoint adaptation of \emph{polyread}-returning methods
occurs at the point of method application.
If any \emph{polyread} parameter is given a \emph{readonly} argument,
then the return type is \emph{readonly}.
We must conservatively assume that any \emph{polyread} parameter may be
returned or assigned into (the locality of) the returned reference,
so we must return \emph{readonly} to protect the \emph{readonly} guarantee
of the \emph{readonly} argument.
If none of the arguments are \emph{readonly}, but at least one is \emph{mutable},
then we assume that some reference in the locality of the mutable argument may
be returned, and that the caller may need to mutate that reference after
the viewpoint-adapted function returns.
The remaining case is where all of the \emph{polyread} parameters are given
\emph{polyread} arguments.
In this case, the returned reference remains in an unadapted state (\emph{polyread}),
allowing unadapted methods to be composed.

\begin{table}
	\begin{center}
	\begin{tabular}{ccc|c}
	Mutable    & Polyread   & Readonly   & Adapted \\
	arguments? & arguments? & arguments? & RI Type \\
	\hline
	   &   &   & polyread \\
	 Y &   &   & mutable \\
	   & Y &   & polyread \\
	 Y & Y &   & mutable \\
	   &   & Y & readonly \\
	 Y &   & Y & readonly \\
	   & Y & Y & readonly \\
	 Y & Y & Y & readonly \\
	\end{tabular}
	\end{center}
	\caption{Viewpoint adaption for \emph{polyread}-returning methods.}
	\label{viewadapt-methods}
\end{table}

Table~\ref{viewadapt-methods} lists the possible combinations of argument RI types
and their resulting return RI types.
The resulting viewpoint adaptation is summarized in definition~\ref{viewadapt-def}.

\begin{definition}
	\label{viewadapt-def}
	Let \code{m} be a \emph{polyread}-returning method, and $P$ be the set of all
	parameters of \code{m} that have the RI type \emph{polyread}.
	For a given an application of \code{m}, let $A$ be the set of arguments that
	map to the elements of $P$.
	If any elements of $A$ have the RI type \emph{readonly}, then
	the adapated type of the application is \emph{readonly}.
	Otherwise, if any elements of $A$ have the RI type \emph{mutable}, then
	the adapated type of the application is \emph{mutable}.
	Otherwise, adapated type of the application is \emph{polyread}.
\end{definition}

The rules for viewpoint adaptation only need to account for \emph{polyread} parameters.
If a \emph{polyread}-returning method has \emph{readonly} parameters,
then we know that those parameters cannot be returned or assigned into the return reference,
so they cannot affect viewpoint adaptation.
\emph{Mutable} parameters may be returned or assigned into the returned reference,
but since \emph{mutable} references can be safely used in \emph{mutable}, \emph{polyread},
or \emph{readonly} contexts, they do not place any constraints on
the semantics of viewpoint adaptation.

Our viewpoint adaptation differs from ReIm~\cite{reim} in that
we determine RI return types based only on argument types
(argument types include the receiver type), while
ReIm adapts return types based on the assignment context or call-site context.
We made this decision primarily for compatibility with Scala's local type inference,
which determines types bottom-up -- the types of the left-hand sides
of assignments are not always available before the right-hand side types
must be determined.
The decision of ReIm's authors to consider of the assignment context or call-site context
may permit their inference algorithm to produce more ideal types
in some cases.
Their global inference system, ReImInfer, appears to rely
on call-site context to produce a maximal typing.
Unfortunately, the maximal typing produced by ReImInfer may become inconsistent
if generic types are introduced.
We restrict our focus to the problem of local inference,
leaving global inference of RI types as a separate problem
for future work.
%The authors of ReIm claim that context sensitivity permits a better
%typing.

\begin{comment}
The difference is due to the nature of the inference used.
ReImInfer is a global inference system, but Scala uses local type inference.
Under local type inference, the type of the right-hand side of an assignment
is determined without taking the type of the left-hand side into account.
Therefore, we can expect the parameter types to be available at a function
application site, but not necessarily the type that the return value must
be adpated to.
Because of the restrictions of local type inference,
we cannot guarantee a ``best'' typing like ReImInfer.
However, we should note that ReImInfer does not support generic types,
and it is questionable whether the approach it takes can
find best typings that will type-check correctly in the presence of generics.
We consider global inference and local inference to be separate problems.
Where global inference addresses the issue of generating or suggesting annotations
for non-annotated code, local inference is normally used to reduce the programmer's
annotation burden when working with annotated code.
Our work focuses on the integration of RI types with local type inference,
leaving global inference as a possibility for future work.
\end{comment}

%These are roughly the same types as used by ReIm~\cite{reim} and Javari~\cite{javari}
%(although Javari uses the term \emph{romaybe} instead of \emph{polyread}).
%However, our interpretation of the distinction between \emph{mutable} and \emph{polyread} differs slightly
%from ReIm and Javari -- we detail our interpretation below.

\subsection{Defaults}

One of the objectives of the RI type system implementation is to permit
RI annotations to be incrementally added to existing code.
Another objective is to allow type annotations to be locally inferred.
Meeting these objectives requires choosing appropriate defaults
where RI types are not annotated and cannot be inferred.

For code with no RI type information (as would be the case when using a
pre-compiled library that was compiled without RI types),
we assume that every reference is \emph{mutable}.
Parameters must be \emph{mutable} because we must conservatively assume that any called
method may mutate the locality of any parameter.
Return values can safely be assumed \emph{mutable} because in the absence of RI type annotations,
there is no way to apply \emph{polyread} or \emph{readonly} constraints.
Fields may be typed as \emph{mutable} or \emph{polyread}, since in both cases
access through a \emph{mutable} reference will cause the field to behave as \emph{mutable}.
Local variables may similarly be typed as \emph{mutable} or \emph{polyread} by default.
If \emph{polyread}, then access through a mutable frame reference %(see section~\ref{framedef_examples})
will cause the variable to behave as \emph{mutable}.
These defaults should allow unannotated code to type-check cleanly.

One implication of a \emph{mutable}-everywhere default is that code with
RI type annotations cannot pass \emph{polyread} or \emph{readonly} arguments
to unannotated methods.
While this implication may seem restrictive, it is necessary to
preserve reference immutability constraints.
Because we do not know whether the called method will perform mutations,
we cannot guarantee that a call to that method will preserve the
\emph{readonly} or \emph{polyread} constraints.
The ideal way to address this issue is to annotate the called method
(and as necessary, methods it depends on) and re-compile.
If it becomes necessary in practice to allow such calls without
annotating called methods, we may need to introduce an annotation that
can be placed at the call site (perhaps an intentionally ugly annotation)
that makes an exception to normal type-checking rules.

For code with annotations, we still use \emph{mutable} defaults
where the RI type is not declared or inferred.
\emph{Mutable} references are assignable to \emph{polyread} or \emph{readonly}
variables, hopefully relieving programmers from the need to declare RI types
except on method signatures and certain key variables.

The RI types of fields can be inferred from the constructor, as is usual
for Scala code.
The \emph{mutable} default is acceptable here also, since according to
definition~\ref{accesspath-ri-def}, the resulting RI type of a field access
is the same regardless of whether the field's type is \emph{mutable} or \emph{polyread}.
For \emph{readonly}, the programmer still has the option to annotate it explicitly
or allow it to be inferred.

\section{Integrating RI with Scala type inference}

One good reason to perform inference on Reference Immutability (RI) types is that inference
reduces the annotation burden -- the programmer does not need to
explicitly declare RI types where they can be inferred from the context.
One approach to RI inference is to build an inference algorithm that runs
separately from the type checker -- this is the approach taken by ReIm~\cite{reim}.
ReIm's inference algorithm is guaranteed to find a maximal typing.
The disadvantage of this approach is that certain complexities
in the Scala type system (such as covariance, contravariance, and trait linearization)
will increase the complexity of type inference and checking.
Another approach is to rely on Scala's built-in type inference mechanism,
which is the approach attempted for Efftp~\cite{efftp}.
The potential advantage of this second approach is that many complexities
in Scala can be handled automatically.
One disadvantage of this approach is that more annotations may be required to
achieve a maximal typing -- but on the other hand, RI types would behave
similarly to other types.\footnote{We should note here that Efftp's difficulties are
likely due to the attempt to use Scala's type inference mechanisms to
propagate arbitrary effect types, which do not necessarily
behave like standard data types. RI types, on the other hand, are merely constraints
on references, and so can be propagated very much like standard data types.}

A few modifications of Scala's type inference system are required to implement
RI types.
First, the inference must be aware of the RI type hierarchy,
specifically that \code{mutable} is a subtype of \code{polyread} and
\code{polyread} is a subtype of \code{readonly}.
Each standard data type also has an RI type.
If no RI type has been declared or can be inferred, the RI type defaults to \code{mutable}.
Whenever the type inferencer performs a least-upper-bound (LUB) or
greatest-lower-bound (GLB) operation on a data type, a LUB or GLB is also performed
on the RI type.

A type is considered equal to another type if and only if
the data types are equal \emph{and} the RI types are equal
(and similarly for subtype relationships).
We initially planned on ignoring the RI type when computing type equality
and subtyping relationships to ensure that adding RI annotations would never change
the program semantics. However, ignoring the RI type also means that we forfeit
use of Scala's built-in mechanisms for type-checking.
(Type checking for a language like Scala is non-trivial due to covariance,
contravariance, abstract types, structural types, and other features.)
Our current approach is to allow the Scala type-checker to consider
RI types. In the case of unannotated code, all RI types are defaulted to
\code{mutable}, resulting in no change to program semantics.
If annotations are added, then those annotations will be taken into account
when subtyping relationships are evaluated.
Changed subtyping relationships could cause Scala's inference engine
to make different decisions about which overloaded methods or which
implicit functions most closely match a given context.
However, we do not consider these changes to be problematic in general.
One of the following situations must apply for any given decision about overloading choices:
1) adding annotations results in no change to the choice of overload;
2) adding annotations results in the previous best choice to become non-best,
	resulting in a different selection;
3) adding annotations results in the previous best choice to be rejected due
to incompatible RI types, resulting in a different selection; or
4) adding annotations results in all overloads to be rejected due to incompatible RI types,
	resulting in an error.
We assert that all of these situations are correct in theory,
although in practice the process of adding annotations to existing software
may break the code in unexpected ways if the code does not carefully manage overloads or implicits.
We expect that the \emph{mutable} default will help in practice,
since unannotated overloads (all \emph{mutable} parameters) will tend not to match annotated code
(which is likely to use at least a few \emph{readonly} or \emph{polyread} variables).



%then one of the following situations
%will apply whenever Scala's inference engine makes a decision
%If annotations are present, it is possible that the compiler may make different
%selections of implicit functions or overloaded methods if the preferred
%selection (before annotation) is disallowed due to incompatible RI types.
%Disallowing method selections with incompatible RI types is the behaviour
%we want; the only question is whether we allow alternative methods to be
%selected silently, or do we force an error?

Other required modifications include handling of viewpoint adaptation
at call sites and propagation of RI types along access paths.

% Closures?
