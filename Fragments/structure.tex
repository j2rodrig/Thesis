\documentclass[letterpaper,11pt]{article}

\usepackage{alltt}
\usepackage{amsmath}
\usepackage{amsthm}
\usepackage[greek,english]{babel}
%\usepackage{booktabs}
%\usepackage{cite}
%\usepackage{color}
\usepackage{fixltx2e}
\usepackage[T1]{fontenc}
%\usepackage{fullpage}
\usepackage[top=1in, bottom=1in, left=1in, right=1in]{geometry}
\usepackage{graphicx}
\usepackage[utf8]{inputenc}
\usepackage{listings}
%\usepackage{mathtools}
\usepackage{mdwlist}
\usepackage{multirow}
%\usepackage{needspace}
\usepackage{url}
\usepackage{verbatim}
\usepackage{xspace}

% Fonts and symbols
%\usepackage{amsfonts}
%\usepackage{amssymb}
%\usepackage[scaled=.65]{beramono}
\usepackage{courier}
\usepackage{times}

\newenvironment{absolutelynopagebreak}
  {\par\nobreak\vfill\penalty0\vfilneg
   \vtop\bgroup}
  {\par\xdef\tpd{\the\prevdepth}\egroup
   \prevdepth=\tpd}

\newcommand{\greekmu}{\greektext m\latintext}

\newcommand{\tab}[0]{\hspace{0.4cm}}

\renewcommand{\ttdefault}{txtt}

\lstset{
	basicstyle=\footnotesize\ttfamily,
	tabsize=4,
	extendedchars=true,
	breaklines=true,
	escapeinside={(*}{*)},
	%frame=l,  % trbl for single-line frames, TRBL for double-line frames
	xleftmargin=1.5\parindent
}

\newcommand{\code}[1]{\lstinline$#1$}

% Theorem Styles
\newtheorem{theorem}{Theorem}[section]
\newtheorem{lemma}[theorem]{Lemma}
\newtheorem{proposition}[theorem]{Proposition}
\newtheorem{corollary}[theorem]{Corollary}
% Definition Styles
\theoremstyle{definition}
\newtheorem{definition}{Definition}[section]
\newtheorem{example}{Example}[section]
\theoremstyle{remark}
\newtheorem{remark}{Remark}



\begin{document}
	
	% TODO
	% method viewpoint adaptation: least upper bound on polyread params.
	% access paths RI type for x.f where x is e.g. @ro List[@mu T] -- what are the implications of defining w.f to be @mu?
	% related works!

\title{Reference Immutability for Scala}
\author{Jonathan Rodriguez\\University of Waterloo}
\maketitle

%The foundation of good programming is modular reasoning.
%When I, as a programmer, look at a piece of code in isolation,
%how much do I know about its invariants?

%Critical to modular reasoning is the ability to clearly see what
%a piece of code will \emph{always do} or \emph{never do},
%or what effects will \emph{always occur} or \emph{never occur}.

%In the absence of good invariants, you always run the risk of
%your code being used in ways you don't expect and for
%purposes you never predicted.

%reference immutability types for modularly checkable documentation of programmer intent
%incremental annotation
%where practical, minimize the number of annotations that must be added

Critical to creating large and reliable programs
is the programmer's ability to reason locally about
all effects a piece of code may have.
Compilers can help with this task by automatically enforcing
certain guarantees or invariants.
These guarantees are often expressed as data types, which delimit the
ways in which any given piece of data can be used or operated on.
Classical data types, however, only provide \emph{shallow} or \emph{path-independent} guarantees --
the classical type of a field, for example, is independent of
the chain of references or pointers traversed
to reach that field.
As a result, local reasoning about \emph{deep}, \emph{transitive}, or \emph{path-dependent}
guarantees is not well-supported by classical type systems.
Many objects logically contain or link to other objects.
Path-dependent guarantees allow the programmer to express (for example) that a method
will never modify contained or linked objects.
The restrictions on the original object's usage are transitively applied to all contained or linked objects.
In this work, I
develop an extension of the Scala language's type system to support
\emph{deep types}.
Initially, I plan to implement \emph{reference immutability} as an instance of a deep type system,
but other deep types may also be useful (such as the \emph{safe} guarantee
inferred by Sălcianu and Rinard's analysis~\cite{purity-effect-java}).

%Oh yeah: using subclassing of annotations (e.g. polyread inherits from readonly) to get the subtyping relationship.
%  so we may still need to modify the subtyping functions to take the annotations into account.



%alternative positioning:
%implementing transitive constraints
%what kinds of constraints can we represent? Obviously, write constraints... read constraints, I/O constraints...

%expected contrinbutions include: 


\section{Introduction}
\label{introduction}

%Reference immutability
\emph{Reference immutability} is a constraint on a reference that guarantees that
no object reached through that reference can be modified.
The guarantee is \emph{deep} or \emph{transitive} --
unlike a pointer-to-\emph{const} in C++, a \emph{readonly} reference
requires that all references contained within the referred-to object
are treated as \emph{readonly}.

Reference immutability is a weaker guarantee than \emph{object immutability}.
An object immutability guarantee ensures that immutable objects cannot be modified
by anyone, but a \emph{readonly} reference only guarantees that objects cannot be modified
through that specific reference.
Other, non-\emph{readonly}, references to the same objects are not prohibited.

%Motivation
Reference immutability is useful as a support for programmer reasoning.
Reference immutability annotations can be used to express that
a method will never modify objects reached through certain parameters,
helping provide the callers of the method with strong guarantees about which
objects may be modified.
One example of a situation where readonly parameters are useful is where
the caller would otherwise have to make a copy of an object
to guarantee that the original object is not modified.
Reference immutability annotations can also indicate that the object returned
to a caller should not be modified.
For example, a method that performs queries on an immutable graph can safely
return elements of that graph if its return type is annotated \emph{readonly}.
Reference immutability also helps programmers express that an object
can be modified during initialization, but is immutable afterward.

Reference immutability is useful to support stronger guarantees,
such as object immutability and method purity.
Object immutability is useful when creating immutable collections or other immutable
structures.
Object immutability is present where it is known that all references to an
object are \emph{readonly}.
Reference immutability can also express \emph{temporary object immutability},
which is a guarantee that a method call will not modify an object, although
the caller can modify it before and after the call.

Method \emph{purity} or \emph{side-effect freedom}
is present where all references passed to a method (including \code{this}
and references to the method's enclosing scopes) are \emph{readonly}.
\emph{Purity} is defined here as the inability to mutate
objects that exist in a method's \emph{prestate}.
(A method's \emph{prestate} is the set of all objects that may exist in memory
immediately prior to invoking the method.)

Work on reference immutability will provide some insight into the implementation
of other deep guarantees, which may be explored in future work.
For example, the \emph{safe} guarantee (see~\cite{purity-effect-java})
is a guarantee that no new aliases will be created that make the safe reference
(or its transitively reachable references) visible from a method's prestate.


%Another possibly-useful guarantee is a guarantee that no \code{volatile}
%reads are performed on any object reachable through a given reference.


\begin{comment}
Optimizations that depend on referential transparency may also benefit from
reference immutability.
If it can be established that a method is side-effect-free, does not throw exceptions,
and does not perform any synchronization operations
(such as \code{volatile} reads, interlocked operations, or calls to \code{synchronized} methods),
then the compiler should be able to treat that method as referentially transparent.
Extensions to the basic reference immutability system may be able
to represent \code{volatile} reads, synchronization operations, and I/O,
but I leave these extensions as challenges for future work.
%It should be noted that calls to \code{synchronized} methods
%and many interlocked operations involve writes to memory, which 
\end{comment}

In this work, I explore how a reference immutability system may be integrated with
the Scala programming language~\cite{scala-book}.
Other reference immutability proposals have been implemented for Java (e.g., ReIm~\cite{reim},
Javari~\cite{javari}), but
Scala offers different challenges and opportunities.
Specifically, Scala has built-in support for nested methods,
generic types, and local type inference.
I attempt to implement reference immutability in a way that makes sense
with respect to Scala's design.
I conjecture that the best way to integrate reference immutability is to
handle reference immutability constraints like ordinary data types,
subject as much as possible to the same inference and subtyping rules as other types.

%The Efftp~\cite{efftp} system also attempted to integrate side-effect annotations into
%the Scala type system, but it ran into difficulties which we conjecture were caused
%by an attempt to propagate effect types differently than ordinary data types.
%Our intent is the same as Javari's~\cite{javari} was for Java, except that we leave certain
%aspects such as dynamic downcasting and assignability for future work.


% RI types
% Implementation in Scala: compiler issues, augmentation of nested methods/classes with extra parameters. Implementation details, timeline.
% Related work: papers on immutability, side effect analysis, pointer analysis, ownership.

Reference immutability types are explained in detail in section~\ref{ri_types}.
Integrating reference immutability types with Scala is discussed in section~\ref{scala-integration}.
Evaluation is discussed in section~\ref{evaluation},
related work in section~\ref{related-work}, and
I conclude in section~\ref{conclusion}.

%Scala language
%Nested methods
%Local type inference
%Generics

%Previous proposals
%ReIm
%JPure -> later section
%Javari (is this proposal about 'Scalari'?)
%JoeE


\section{Reference Immutability Types}
\label{ri_types}

%Problems to fix:
% checking RI types of first-class functions?
% least upper bound for parameters?
% 

% class S[T] { val f: T }
% val s: @ro = new S[U @mu]
% s.f = @ro S[T] :> ???

Reference Immutability (RI) Types are qualifiers on ordinary data types.
In this section, I discuss how basic RI types work, followed by discussions
of RI type defaults and how RI types could interact with generic types.

The basic RI types I use here are \emph{readonly}, \emph{mutable}, and \emph{polyread}.
RI types specify deep (or transitive) write permissions.
A \emph{readonly} type disallows writes to the referenced object, and transitively
all objects referenced by fields of that object.
A \emph{mutable} type does not place any restrictions on the use of referenced objects --
for programs with no RI types, all references are assumed to be mutable.
A \emph{polyread} (``polymorphic read'') type indicates that readonly restrictions may or may not apply,
depending on context. (\emph{Polyread} is discussed in detail below.)

RI types form a hierarchy.
The \emph{mutable} type is a subtype of \emph{polyread},
and \emph{polyread} is a subtype of \emph{readonly}.
Briefly, the subtyping relationship is:

\begin{center}
	mutable <: polyread <: readonly
\end{center}

The transitive nature of RI types means that readonly restrictions must be
propagated along access paths.
In particular, any field accessed through a \emph{readonly} reference must
be treated as \emph{readonly}, and a \emph{mutable} field accessed through
a \emph{polyread} reference is treated as \emph{polyread}.
If \emph{polyread} fields are accessed through \emph{mutable} references,
they are treated as (or adapted to) \emph{mutable}.
This rule differs from ReIm and Javari in the case where the field is \emph{mutable}.
For example, I treat a \emph{mutable} field access through a \emph{readonly}
reference as \emph{readonly}, whereas Javari treats such an access as \emph{mutable}.
Javari does this to allow programmers to declare that certain fields are are not
part of an object's abstract state by explicitly giving them the type \emph{mutable}.
ReIm does not allow fields to be declared \emph{mutable}.
My rules are designed to allow the same expressive power as ReIm, but also to
treat fields and local variables more uniformly than ReIm, which is important for defining
how RI types interact with nested methods and closures (see section~\ref{scala-integration}).
Definition~\ref{accesspath-ri-def} summarizes the RI type rule for access paths.

%The transitive nature of RI types means that readonly restrictions must be
%propagated along access paths -- a reference that is readonly along one
%access path may be mutable when observed along a different access path.
%The following table lists all possible combinations of RI types for
%a given access path~\code{h.k}:
%We define the RI type of an access path to be at least as restrictive 
%as any element of that path.
%(Unlike Javari~\cite{javari}, we do not allow the \emph{mutable} RI type to
%violate readonly restrictions.)

\begin{definition}
	\label{accesspath-ri-def}
	Let \code{h} be an access path to some object~$o$, and let~$r_h$
	be the Reference Immutability (RI) type of~\code{h}.
	Furthermore, let~\code{k} be a field of $o$ with an RI type of~$r_k$.
	The RI type of the access path~\code{h.k}
	%Given an identifier~\code{h} and an access path~\code{k}
	%with Reference Immutability (RI) Types~$r_h$ and~$r_k$, respectively,
	%the RI type of the combined access path~\code{h.k}
	is $r_h$ if $r_k$
	is \emph{polyread}, else the RI type of~\code{h.k} is
	the least upper bound of $r_h$ and $r_k$.
\end{definition}

The complete list of RI type combinations defined in definition~\ref{accesspath-ri-def} is:

\begin{center}
	\begin{tabular}{cc|c}
	RI Type of \code{h} & RI Type of \code{k} & RI Type of \code{h.k} \\
	\hline
	mutable & mutable & mutable \\
	mutable & polyread & mutable \\
	mutable & readonly & readonly \\
	polyread & mutable & polyread \\
	polyread & polyread & polyread \\
	polyread & readonly & readonly \\
	readonly & mutable & readonly \\
	readonly & polyread & readonly \\
	readonly & readonly & readonly \\
	\end{tabular}
\end{center}



RI types can be used in generic contexts. For example,
an Iterator class may be declared covariant in its element type,
so the IR type of the element is also covariant.
The programmer can declare that such an iterator is \emph{mutable}, but
contains \emph{readonly} elements~-- in this case, the iterator object itself
can be modified, but elements returned by the iterator cannot be modified.
%Elements can be added or removed from this iterator, but elements retrieved
%from the list cannot be modified.
The iterator is a subtype of a \emph{polyread} or \emph{readonly} iterator
with \emph{readonly} elements, but not a subtype of any iterator
with non-\emph{readonly} elements.

%Covariance, contravariance, and invariance are already supported by Scala.
%To the extent possible, RI types adhere to 

The purpose of the \emph{polyread} type is for \emph{viewpoint adaptation}.
Viewpoint adaptation allows a method's return type to vary (or be adapted) depending on
the types of its parameters. For example, consider a basic getter function
(shown with an explicit receiver \code{this}):
\begin{lstlisting}
class C {
	private[this] var x: X @polyread = ...
	def get (this: C @polyread): X @polyread = this.x
}
val ro_c: C @readonly = new C
val mu_c: C = new C
val ro_x: X @readonly = ro_c.get()   (*// readonly this yields readonly x*)
val mu_x = mu_c.get()                (*// mutable this yields mutable x*)
\end{lstlisting}

The \emph{polyread} type allows the programmer to express that \code{get} returns a \emph{mutable}
result when the receiver object is \emph{mutable},
but a \emph{readonly} result when the receiver object is \emph{readonly}.
Without a \emph{polyread} type, I would need two versions of \code{get}
to achieve this flexibility: one version returning \emph{mutable}, and
another version returning \emph{readonly}.

I should note that in this example, the field \code{x} could be declared \code{mutable}
instead of \code{polyread} without having any effect on the semantics or type-correctness
of the code. The access path \code{this.x} in the body of \code{get} has a resulting type
of \code{polyread} in either case, which is consistent with definition~\ref{accesspath-ri-def}.

%The return type of \code{C.get} must be \code{@polyread} to avoid code duplication.
%Otherwise, we would require the programmer to create two versions of \code{C.get}: one returning
%\code{@readonly} and the other returning \code{@mutable}.

Viewpoint adaptation of \emph{polyread}-returning methods
occurs at the point of method application.
If any \emph{polyread} parameter is given a \emph{readonly} argument,
then the return type is \emph{readonly}.
%We must conservatively assume
I must conservatively assume that any \emph{polyread} parameter may be
%returned or assigned into (the locality of) the returned reference,
returned,
so the adapted return type must be \emph{readonly} to protect the \emph{readonly} guarantee
of the \emph{readonly} argument.
%If none of the arguments are \emph{readonly}, but at least one is \emph{polyread},
%then the return type is \emph{polyread}.
%If all arguments to \emph{polyread} parameters are \emph{mutable},
Only if all \emph{polyread} parameters are given \emph{mutable} arguments
can the return type be adapted to \emph{mutable}.
Consider the following function \code{getter}:

\begin{lstlisting}
def getter (a:@polyread, b:@polyread, c:@mutable, d:@readonly):@polyread = {
	c.f = c    // always OK
	c.f = d    // OK if f is readonly
	c.f = a    // OK if f is readonly
	c.f = b    // OK if f is readonly
	return a   // OK
	return b   // OK
	return c   // OK because mutable <: polyread
	return d   // Error because readonly :> polyread
}
\end{lstlisting}

The only side effects that can occur occur through the \code{mutable}
parameter. An alias to any parameter can be created in the \emph{locality}\footnote{
The \emph{locality} of an object includes that object plus all objects transitively reachable
through that object's mutable fields.
}
of~\code{c}, provided the field~\code{f} has a compatible type.
Note that~\code{f} must be \code{readonly} when aliasing to \code{readonly} \emph{or} \code{polyread}
parameters. If \code{f} is \code{polyread}, then the access path \code{c.f} is \code{mutable},
disallowing assignment of non-mutable parameters.

The following function \code{adaptsGetter} calls \code{getter} with all RI type combinations as
arguments to \emph{polyread} parameters:

\begin{lstlisting}
def adaptsGetter (e:@mutable, f:@polyread, g:@readonly) = {
	val r1:@mutable  = getter(e, e, e, g)    // returns e
	val r2:@polyread = getter(f, f, e, g)    // returns e or f
	val r3:@readonly = getter(g, g, e, g)    // returns e or g
	val r4:@polyread = getter(e, f, e, g)    // returns e or f
	val r5:@readonly = getter(e, g, e, g)    // returns e or g
	val r6:@readonly = getter(f, g, e, g)    // returns e or f or g
}
\end{lstlisting}

Conservatively (without knowing anything about \code{getter} except parameter types and return type),
I must assume that \code{getter} may return any \emph{polyread}
or \emph{mutable} parameter.
Preserving the \emph{readonly} or \emph{polyread} guarantees on argument types
therefore requires that the return type be an upper bound of these argument types.
One corner case is where a method does not have any \emph{polyread} parameters,
but nevertheless has a \emph{polyread} return type. For example, consider a version
of \code{getter} with no \emph{polyread} parameters:

\begin{lstlisting}
def getter (c:@mutable, d:@readonly):@polyread = {
	c.f = c    // always OK
	c.f = d    // OK if f is readonly
	return c   // OK because mutable <: polyread
	return d   // Error because readonly :> polyread
}
\end{lstlisting}

Because there are no \code{polyread} parameters,
there is no way for such a method to return anything other than \code{mutable}.
It is therefore safe to adapt such a method to \code{mutable}.
Viewpoint adaption of a method return type can therefore be the \emph{least} upper bound
of the parameter types, where the least upper bound of an empty set is
defined to be the bottom type \emph{mutable}.
Definition~\ref{viewadapt-def} summarizes the viewpoint adaptation rule for methods.

%If none of the arguments are \emph{readonly}, but at least one is \emph{mutable},
%then we assume that some reference in the locality of the mutable argument may
%be returned, and that the caller may need to mutate that reference after
%the viewpoint-adapted function returns.
%The remaining case is where all of the \emph{polyread} parameters are given
%\emph{polyread} arguments.
%In this case, the returned reference remains in an unadapted state (\emph{polyread}),
%allowing unadapted methods to be composed.

%\begin{table}
%	\begin{center}
%	\begin{tabular}{ccc|c}
%	Mutable    & Polyread   & Readonly   & Adapted \\
%	arguments? & arguments? & arguments? & RI Type \\
%	\hline
%	   &   &   & polyread \\
%	 Y &   &   & mutable \\
%	   & Y &   & polyread \\
%	 Y & Y &   & polyread \\
%	   &   & Y & readonly \\
%	 Y &   & Y & readonly \\
%	   & Y & Y & readonly \\
%	 Y & Y & Y & readonly \\
%	\end{tabular}
%	\end{center}
%	\caption{Viewpoint adaption for \emph{polyread}-returning methods.}
%	\label{viewadapt-methods}
%\end{table}

%Table~\ref{viewadapt-methods} lists the possible combinations of argument RI types
%and the resulting return RI types.
%The resulting viewpoint adaptation is summarized in definition~\ref{viewadapt-def}.

\begin{definition}
	\label{viewadapt-def}
	Let \code{m} be a \emph{polyread}-returning method, and $P$ be the set of all
	parameters of \code{m} that have the RI type \emph{polyread}.
	For a given an application of \code{m}, let $A$ be the set of arguments that
	map to the elements of $P$.
	The adapted type of the application of \code{m} is the least upper bound
	of the elements of $A$, or \emph{mutable} if $A$ is empty.
	%If any elements of $A$ have the RI type \emph{readonly}, then
	%the adapated type of the application is \emph{readonly}.
	%Otherwise, if any elements of $A$ have the RI type \emph{mutable}, then
	%the adapated type of the application is \emph{mutable}.
	%Otherwise, adapated type of the application is \emph{polyread}.
\end{definition}

The rules for viewpoint adaptation only need to account for \emph{polyread} parameters.
If a \emph{polyread}-returning method has \emph{readonly} parameters,
then those parameters cannot be returned or assigned into the return reference,
so they cannot affect viewpoint adaptation.
\emph{Mutable} parameters may be returned or assigned into the returned reference,
but since \emph{mutable} references can be safely used in \emph{mutable}, \emph{polyread},
or \emph{readonly} contexts, they do not place any constraints on
the semantics of viewpoint adaptation.

The viewpoint adaptation discussed in this paper differs from ReIm~\cite{reim} in that
the RI return types are determined based only on argument types
(argument types include the receiver type), whereas
ReIm adapts return types based on the assignment context or call-site context.
I made this decision primarily for compatibility with Scala's local type inference,
which determines types bottom-up -- the types of the left-hand sides
of assignments are not always available before the right-hand side types
must be determined.
The decision of ReIm's authors to consider of the assignment context or call-site context
may permit their inference algorithm to produce more ideal types
in some cases.
Their global inference system, ReImInfer, appears to rely
on call-site context to produce a maximal typing.
Unfortunately, the maximal typing produced by ReImInfer may become inconsistent
if generic types are introduced.
I discuss ReIm and ReImInfer in greater detail in Related Work~(section~\ref{related-work}).
I restrict my focus to the problem of local inference,
leaving global inference of RI types as a separate problem
for future work.
%The authors of ReIm claim that context sensitivity permits a better
%typing.

\begin{comment}
The difference is due to the nature of the inference used.
ReImInfer is a global inference system, but Scala uses local type inference.
Under local type inference, the type of the right-hand side of an assignment
is determined without taking the type of the left-hand side into account.
Therefore, we can expect the parameter types to be available at a function
application site, but not necessarily the type that the return value must
be adpated to.
Because of the restrictions of local type inference,
we cannot guarantee a ``best'' typing like ReImInfer.
However, we should note that ReImInfer does not support generic types,
and it is questionable whether the approach it takes can
find best typings that will type-check correctly in the presence of generics.
We consider global inference and local inference to be separate problems.
Where global inference addresses the issue of generating or suggesting annotations
for non-annotated code, local inference is normally used to reduce the programmer's
annotation burden when working with annotated code.
My work focuses on the integration of RI types with local type inference,
leaving global inference as a possibility for future work.
\end{comment}

%These are roughly the same types as used by ReIm~\cite{reim} and Javari~\cite{javari}
%(although Javari uses the term \emph{romaybe} instead of \emph{polyread}).
%However, our interpretation of the distinction between \emph{mutable} and \emph{polyread} differs slightly
%from ReIm and Javari -- we detail our interpretation below.

\subsection{Defaults}
\label{ri_type_defaults}

One of the objectives of the RI type system implementation is to permit
RI annotations to be incrementally added to existing code.
Another objective is to allow type annotations to be locally inferred.
Meeting these objectives requires choosing appropriate defaults
where RI types are not annotated and cannot be inferred.

For code with no RI type information (as would be the case when using a
pre-compiled library that was compiled without RI types),
every reference is defaulted to \emph{mutable}.
Parameters must be \emph{mutable} because of the conservative assumption that any called
method may mutate the locality of any parameter.
Return values can safely be assumed \emph{mutable} because in the absence of RI type annotations,
there is no way to apply \emph{polyread} or \emph{readonly} constraints.
Fields may be typed as \emph{mutable} or \emph{polyread}, since in both cases
access through a \emph{mutable} reference will cause the field to behave as \emph{mutable}.
Local variables may similarly be typed as \emph{mutable} or \emph{polyread} by default.
If \emph{polyread}, then access through a mutable frame reference %(see section~\ref{framedef_examples})
will cause the variable to behave as \emph{mutable}.
These defaults should allow unannotated code to type-check cleanly.

One implication of a \emph{mutable}-everywhere default is that code with
RI type annotations cannot pass \emph{polyread} or \emph{readonly} arguments
to unannotated methods.
While this implication may seem restrictive, it is necessary to
preserve reference immutability constraints.
Because it is unknown whether the called method will perform mutations,
it cannot be guaranteed that a call to that method will preserve the
\emph{readonly} or \emph{polyread} constraints.
The ideal way to address this issue is to annotate the called method
(and as necessary, methods it depends on) and re-compile.
If it becomes necessary in practice to allow such calls without
annotating called methods, another annotation may be needed at the call site
that makes an exception to the type-checking rules (perhaps an intentionally ugly annotation).

For code that does contain annotations, the \emph{mutable} default applies
anywhere the RI type is not declared or inferred.
\emph{Mutable} references are assignable to \emph{polyread} or \emph{readonly}
variables, hopefully relieving programmers from the need to declare RI types
except on method signatures and certain key variables.

The RI types of fields can be inferred from the constructor, as is usual
for Scala code.
The \emph{mutable} default is acceptable here also, since according to
definition~\ref{accesspath-ri-def}, the resulting RI type of a field access
is the same regardless of whether the field's type is \emph{mutable} or \emph{polyread}.
For \emph{readonly}, the programmer still has the option to annotate it explicitly
or allow it to be inferred.

\subsection{Generic RI Types}
\label{generic-ri-types}

RI types can be used in generic contexts.

Consider a \emph{mutable} iterator of \emph{mutable} elements.
Iterators in the Scala collections library are covariant in their element types,
so a \emph{mutable} iterator of \emph{mutable} elements
is a subtype of any of the following:

\begin{center}
	\begin{tabular}{ccc}
		& & \emph{mutable} iterator of \emph{readonly} elements \\
	\emph{mutable} iterator of \emph{mutable} elements & <: & \emph{readonly} iterator of \emph{mutable} elements \\
		& & \emph{readonly} iterator of \emph{readonly} elements \\
	\end{tabular}
\end{center}

These types form a lattice with the following additional relationships:

\begin{center}
	\begin{tabular}{cc}
	\emph{mutable} iterator of \emph{readonly} elements &
		\multirow{2}{*}{<: \emph{readonly} iterator of \emph{readonly} elements} \\
	\emph{readonly} iterator of \emph{mutable} elements \\
	\end{tabular}
\end{center}

Most of these types provide the permissions one would naturally expect: e.g.,
\emph{readonly} iterators of \emph{readonly} elements do not permit any modification, and
\emph{mutable} iterators of \emph{readonly} elements permit mutation of the iterator itself
but not the elements it refers to.
However, \emph{readonly} iterators of \emph{mutable} elements do not necessarily
permit mutation of their elements even though the elements are declared \emph{mutable}.
To show why this occurs, consider an iterator over a list of \emph{mutable} elements:

%To show why this occurs, consider an iterator over a list that allows non-destructive peeking
%of the next element:

\begin{lstlisting}
trait IteratorOverList[A] {
	var list: List[A]
	def setElement(elem: A) = { this.list.head = elem }
}
val iterator: IteratorOverList[AnyRef @mutable] @readonly = ...
iterator.setElement(...)   (*// Error: @readonly(this) is not compatible with @mutable*)
\end{lstlisting}

The expression \code{this.list.head} has the type \code{A},
which is declared \code{@mutable} in the concrete instantiation assigned to \code{iterator}.
%which may or may not be mutable in a concrete instantiation of \code{IteratorOverList}.
However, RI type rules demand that
\code{this} must be mutable in \code{setElement} to allow assignment to \code{this.list.head},
which in turn causes a type error where \code{setElement} is called on a \emph{readonly}
iterator.


\begin{comment}
\begin{lstlisting}
trait IteratorOverList[+A] {
	var list: List[A]
	def peek(): A = list.head
}
val iterator = new IteratorOverList[AnyRef @mutable] @readonly { var list = ... }
\end{lstlisting}

The concrete instantiation of \code{IteratorOverList} will have the following definition of \code{peek}
(shown with an explicit receiver \code{this}):

\begin{lstlisting}
def peek(this: IteratorOverList[AnyRef @mutable] @readonly): AnyRef @mutable = this.list.head
\end{lstlisting}

Under definition~\ref{accesspath-ri-def}, this version of method \code{peek()} should fail
to type-check because the type of \code{this.list.head} would be \emph{readonly},
which is incompatible with \code{AnyRef @mutable}.
Scala, however, would not consider this to be a type error because the original definition
of \code{peek()} type-checks.
One option is to allow this situation to type-check correctly,
but at the cost of allowing strong \emph{readonly} guarantees to be broken
where certain uses of generic types are involved.

If we enforce the deep (transitive) guarantees of RI types even where generics are involved,
we need to add some default restrictions on type parameters.
Ideally, these guarantees would be modularly checkable and enforceable (for example,
we shouldn't have to re-check the definition of \code{peek} everywhere
\code{IteratorOverList} is extended).
We plan to consider these choices in greater detail as work progresses.
\end{comment}

%We should note here that if we do want to
One way to
let the programmer express that an iterator cannot be modified, but its
elements can,
is to introduce 
%we always have the option of introducing the equivalent
the equivalent
of C++'s \emph{const} type.
Introducing a \emph{const} type that is distinct from \emph{readonly}
would allow the programmer to clearly distinguish between
transitive guarantees (\emph{readonly}) and non-transitive guarantees (\emph{const}).

\begin{comment}
Under the assumption that all types are shallow (nontransitive), this

trait X {
	// if T has a lower bound of X:
	//   X mutable -> T anything
	//   X polyread -> T readonly or polyread
	//   X readonly -> T readonly
	type T
	val t: T
}
\end{comment}

\section{Integrating RI Types with Scala}
\label{scala-integration}

Scala code uses language features that are not supported or not frequent in Java code,
but nevertheless impact the design of an RI type system. These features include:
\begin{itemize}
	\item \emph{Unrestricted nesting of methods and classes.}
		Methods may modify the local variables in enclosing methods.
		
	\item \emph{Frequent use of closures and lambda expressions.}
		Scala encourages a functional programming style.
		Functions that take other functions as parameters are ubiquious in the Scala standard library.
		
	\item \emph{Local type inference.} Classical data types are often inferred automatically
		where type annotations do not exist. To keep the annotation burden managable,
		RI types should be inferred also, ideally taking advantage of type inference and type checking
		systems that already exist.
	
	\item \emph{Covariant types, contravariant types, structural types, abstract types, and
		other type system features.}
		The RI type system to interact with these features in a reasonable way.

	%\item \emph{Trait composition and inheritance.}
	%	Scala supports very flexible trait compositions.
	%	Unlike Java, which only permits multiple inheritance of interfaces, Scala's traits
	%	can contain defined methods and fields.
	%	We would also like RI types to 

\end{itemize}

\subsection{Basic Annotations}
\label{basic-annotations}

I plan to implement RI types as annotations on ordinary data types.
My goal is to allow any data type to also have an RI type,
even if the data type is abstract.
(Abstract data types have upper and lower type bounds rather than being
constrained to a single data type. Where there are data type bounds, I also want to allow
RI type bounds.)
I begin with some examples of variable/field declarations and method signatures:

\begin{lstlisting}
val a: A @mutable
val b: B @polyread
val c: C @readonly
def d(a: A @mutable, b: B @polyread, c: C @readonly): D @mutable
def e(a: A @mutable, b: B @polyread, c: C @readonly): E @polyread
def f(a: A @mutable, b: B @polyread, c: C @readonly): F @readonly
\end{lstlisting}

If RI annotations are allowed on any type expression, then
some type-checking may have to occur when these declarations are encountered.
For example, consider:
\begin{lstlisting}
class C {
	type A = AnyRef @readonly
	val a: A @mutable = ...
}
\end{lstlisting}
The type \code{A} is declared \code{@readonly},
but field \code{a} is declared to have type \code{A @mutable}.
There are a few alternatives for interpreting the \code{@readonly}
annotation in this example: 1) All references of type \code{A} must
have RI types compatible with \code{@readonly}, in which case
the expression \code{A @mutable} would produce a type error;
2) References of type \code{A} are \code{@readonly} by default, but
the RI type may be overridden in uses of \code{A}; or
3) It is an error to use RI annotations in definitions of abstract types.
Option (2) is the least restrictive, and the only option that does not
require additional type checks.

%For example, the type expression \mbox{\code{A @mutable}} only type-checks
%if~\code{A} itself is mutable.
%The type expression \mbox{\code{C @readonly}} always type-checks
%because all RI types are compatible with \code{@readonly}.

%This extra check prevents the removal of readonly constraints on types,
%which we hypothesize will be important for supporting generic types
%without breaking transitive guarantees (cf. section~\ref{generic-ri-types}).
%The extra check is also necessary to preserve correctness when functions are
%passed as parameters. For example, if the type \code{A} is a readonly-returning function,
%then the expanded signature of method \code{d} is:
%\begin{lstlisting}
%def d(a: (() => T @readonly) @mutable, b: B @polyread, c: C @readonly): D @mutable
%\end{lstlisting}
%which clearly should not type-check because calling \code{a()}
%from within \code{d} would otherwise discard the declared \code{@readonly} constraint
%on \code{a}'s return type.

For convenience,
RI annotations may be placed directly on variables and fields:

\begin{lstlisting}
@mutable  val a: A     // same as:  val a: A @mutable
@polyread val b: B     // same as:  val b: B @polyread
@readonly val c: C     // same as:  val c: C @readonly
\end{lstlisting}

There must also be a way to express how a method modifies its enclosing scope(s),
which include its receiver object (referenced through \code{this}).
For example, let a method \code{g} modify a variable \code{v} that is local
to an enclosing method \code{f}:
\begin{lstlisting}
def f() = {
	var v = ...
	def g() = {
		v = ...
	}
}
\end{lstlisting}

Method \code{g} can be annotated as follows:
\begin{lstlisting}
def f() = {
	var v = ...
	@mutable(f) def g() = {
		v = ...
	}
}
\end{lstlisting}
which means that \code{g} mutates a variable within the body of \code{f}.
The parameter \code{f} within \code{@mutable(f)}
may be understood as a parameter of \code{g} that references the outer environment \code{f}.
%may be understood as a reference
%to a closure of \code{f} that is implicitly passed to \code{g} whenever \code{g}
%is called.

Treating the parameter \code{f} as a reference, the assignment to \code{v}
inside \code{g} is essentially an assignment to \code{f.v}.
(Although there is no language mechanism for directly specifying the equivalent of
the parameter \code{f} in a source program, the equivalent reference does exist
and is manipulated by the compiler.)
According to definition~\ref{accesspath-ri-def},
the assignment to \code{f.v} is permitted where \code{f} is \emph{mutable}
and \code{v} is not \emph{readonly}.
%(Scala syntax prohibits using \code{f} explicitly in this way)
%The assignment to \code{v} is therefore only permitted where
%\code{f} is \code{@mutable}.

Callers of \code{g} must have available a \emph{mutable} reference to a closure
of method \code{f}.
Such a reference is always available from within the body of \code{f}.

If \code{f} is a class rather than a method, then the annotations are quite similar:
\begin{lstlisting}
class f {
	var v = ...
	@mutable(f) def g() = {
		v = ...
	}
}
\end{lstlisting}
In this case, the parameter \code{f} is an alias of \code{this}.
The assignment to \code{v} is therefore permitted where
\code{f} (or \code{this}) is \code{@mutable}.
Callers of \code{g} always have available a reference to an object of
\mbox{\code{class f}}, which is merely the receiver object.
Where the receiver reference is \emph{mutable},
the call to \code{g} is permitted.

%We expect that classes and traits can be handled identically
%for the purposes of RI type integration.

Classes and traits may also be nested within methods.
If, for example, \code{f} was a method and \code{g} was a class,
the same annotations can be used:
\begin{lstlisting}
def f() = {
	var v = ...
	@mutable(f) class g {
		v = ...
	}
}
\end{lstlisting}
In this case, the annotation \code{@mutable(f)}
means that the default constructor of \code{g} may mutate some variable
in the closure of \code{f}.

Nesting may occur to an arbitrary depth, and at each method every
enclosing scope may have a different mutability.
In the following example, \code{g} is mutable in its receiver and
in \code{f}, and \code{h} is mutable in its receiver and
in \code{C}.

\begin{lstlisting}
class C {
	var v = ...
	def f () = {
		var w = ...
		@mutable(f,C) def g () = {
			v = ...
			w = ...
			class D {
				var x = ...
				@mutable(D,C) def h () = {
					x = ...
					v = ...
				}
			}
		}
	}
}
\end{lstlisting}

%The \code{@polyread()} annotation may be used in the same places as \code{@mutable()}
%to indicate that there are enclosing scopes that may be viewpoint-adapted
%when the associated method is called.

An alternative interpretation of these annotations is to assume that
the annotation \code{@mutable(X)} for some enclosing scope name \code{X}
means that all intermediate scopes between the annotation and \code{X}
are mutable.
In the foregoing example, \code{@mutable(f,C)} would be equivalent
to \code{@mutable(C)}, and \code{@mutable(D,C)} would be equivalent
to \code{@mutable(C)} and \code{@mutable(C,f,g,D)}.
While this intrepretation may (in some cases) reduce the total number
of annotation parameters the programmer must specify, it also reduces
the precision available to the programmer.

\subsection{Arguments to Enclosing-Scope Parameters}

I address here the problem of finding the RI types of the
outer-scope references discussed above.
The RI type system must be able to handle 
arbitrary nestings of methods and classes.
%methods within methods, classes within classes, and
%methods within classes, and classes within methods.


For a call to a nested method,
the references to enclosing methods' closures are immediately available.
For example, consider a call to \code{h()} within \code{f} and \code{g}
(outer scopes are designated \code{@mutable} to make them explicit
in this example):

\begin{lstlisting}
@mutable(all) def f() = {
	@mutable(f) def g() = {
		@mutable(f,g) def h() = {
		}
		h()
	}
}
\end{lstlisting}
The argument to parameter \code{f} is available at the call site
because it was passed as a parameter of \code{g},
and the argument to parameter \code{g} is implicitly available from within the
body of \code{g}.
The argument to \code{f} at the call to \code{h()} has the same RI type
as was declared on method \code{g}, or \code{@mutable(f)}.
The argument to parameter \code{g} is always mutable, since a method is always allowed to mutate its local
variables.
This example would raise a type error if the parameter \code{f} on method \code{g}
was instead \emph{readonly}.

If \code{g} is a class instead of a method,
then \code{h()} could be called from a location
other than the body of \code{g}:
\begin{lstlisting}
@mutable(all) def f() = {
	@mutable(f) class g() {
		@mutable(f,g) def h() = {
		}
	}
	val instance = new g()
	instance.h()
}
\end{lstlisting}
The argument to the parameter \code{@mutable(f)} on class \code{g}
is implicitly available within the body of \code{f}.
Since the call to \code{h()} does not necessarily occur within \code{f},
an appropriate parameter is not necessarily available at the call site.
Fortunately, like other class parameters, \code{f} is a field of \code{g},
and it is this field that is implicitly passed to \code{h()}.
Therefore, the call to \code{instance.h()} implicitly passes the
arguments \code{instance.f} and \code{instance} for parameters
\code{f} and \code{g}.
A type error is raised if the types of \code{instance.f} and \code{instance}
are not compatible with \code{f} and \code{g} respectively.

\begin{comment}
Therefore, for the purpose of type-checking the parameter \code{f}
on method \code{h}, it is sufficient to make sure that
the RI type of parameter \code{f} on method \code{h}
is a supertype of the RI type of parameter \code{f} on class \code{g}.
At the site of the call to \code{h}, the RI type of \code{instance}
must be a subtype of the parameter \code{g} on method \code{h}.
For example, the following code has two type errors:
\begin{lstlisting}
@mutable(all) def f() = {
	@readonly(f) class g() {
		@mutable(f,g) def h() = {  // error: mutable not a supertype of readonly
		}
	}
	val instance: g @readonly = new g()
	instance.h()   // error: readonly not a subtype of mutable
}
\end{lstlisting}
\end{comment}


%\subsection{Annotations and Extra Parameters}  %on parameters, fields, local variables, method definitions, and class definitions
%\label{extra-parameters}

\subsection{Defaults and Special Annotation Parameters}

By default, a method is assumed to possibly mutate anything in any of its
enclosing scopes. I introduce the special annotation parameter \code{all} to represent
this default. Other special parameters include \code{none} to represent
no mutation of enclosing scopes and \code{this} as an equivalent to the immediately
enclosing class or trait.

\begin{lstlisting}
@mutable(all)        // may mutate anything (default)
@mutable(none)       // does not mutate anything except local variables/parameters
@mutable(X)          // mutable in class/trait/method X
@mutable(this)       // mutable in innermost enclosing class/trait
\end{lstlisting}

The \code{X} in \code{@mutable(X)} may also be the name of a package.
In this case, the annotation means that the method mutates
something in the locality of package \code{X}.
%something in the locality of that package.
%the current global
%instance of that package.

The \code{@polyread()} annotation is also supported wherever \code{@mutable()}
is supported. For example, the \code{@polyread(this)} annotation
indicates that the associated method is viewpoint-adaptable in its receiver.

The \code{@readonly()} annotation may not be necessary because anything
that is not \emph{mutable} or \emph{polyread} should probably be assumed
\emph{readonly}.

\subsubsection{A Possible Extension to Handle I/O Effects}

References to files and streams can be arguments to \code{@mutable()}.
For example:

\begin{lstlisting}
@mutable(java.lang.System.out)    // may mutate standard output
\end{lstlisting}
I may more fully examine the possibilities as work progresses.

\subsection{Closures}

Closures require some special handling.
When a closure is created, partially-applied parameters and references to outer scopes are bound.
The act of creating a closure involves some loss of information
because the caller of a closure does not necessarily know anything about which
method the closure calls or which parameters have already been bound.

Therefore, closure objects need to carry some information about
their effects to avoid making the conservative assumption
\code{@mutable(all)} for every closure call.

Consider a version of method \code{f} that
creates and calls a closure of \code{g}:
\begin{lstlisting}
def f() = {
	var v = ...
	@mutable(f) def g() = {
		v = ...
	}
	val closure_g: (() => Unit) = g _
	closure_g()
}
\end{lstlisting}


Instances of the type \code{(() => Unit)} may be understood as objects with
\code{apply()} methods. Since \code{apply} in this example calls \code{g()},
\code{apply} should have the same effect as \code{g()}, or \code{@mutable(f)}.
To make the effect annotation available to callers of the closure,
I allow the function type itself to carry the effect annotation \code{@mutable(f)}.
In the following listing,
the closure creation and application inthe foregoing example are made explicit:
%we add the effect annotation on the return type of \code{f}
%and make explicit the creation of the closure object:
\begin{lstlisting}
def f() = {
	var v = ...
	@mutable(f) def g() = {
		v = ...
	}
	val closure_g: (() => Unit) @mutable(f) = new (() => Unit) {
		@mutable(f) def apply() = g()
	}
	closure_g.apply()
}
\end{lstlisting}

Note that the parameterized annotation \code{@mutable()}
has a different meaning from the unparameterized annotation \code{@mutable}.
When used as a type annotation (instead of a method annotation),
\code{@mutable(f)} means that if the referred-to object
has an \code{apply} method, then that \code{apply} method will have
an effect type compatible with \code{@mutable(f)}.

It is possible to pass the closure reference to a location where
the symbol \code{f} is not in scope.
It is also possible for the closure to bind partially-applied parameters,
which are not visible to callers.
These are issues that do not affect ordinary method applications.
I plan to further investigate RI annotation systems for closures.

%how to handle the RI types of closures that
%bind scopes and parameters that are not visible to callers.


\begin{comment}
We must make sure that where \code{f} goes out of scope,
the \code{@mutable(f)} annotation is conservatively adjusted .
For example, if method \code{f} above is enclosed by another method \code{e}
that returns the same closure:
\begin{lstlisting}
def e(): (() => Unit) @mutable(hidden) = {
	def f(): (() => Unit) @mutable(f) = {
		var v = ...
		@mutable(f) def g() = {
			v = ...
		}
		return g _
	}
	return f()
}
\end{lstlisting}
The return type of \code{e} has the effect \code{@mutable(e)}
instead of \code{@mutable(f)}. The symbol \code{f} goes out of scope outside of \code{e},
so we substitute the enclosing scope \code{e} for \code{f}.
We must therefore, take \code{@mutable(e)} to mean that the closure's \code{apply}
method could mutate anything within \code{e} or a scope within \code{e}.
This approach should give the same precision for calling a method directly
as creating a closure and immediately calling it.

If a closure performs a partial application of parameters,
and some of those parameters are \code{@mutable},
then we could use the effect annotation \code{@mutable(partials)}.

Generally, whenever a closure is passed as an argument or otherwise leaves
its current scope, its effect annotation may need to be adjusted
to make sense within the new scope.
If we define \mbox{\code{@mutable(partials)}} to 

The \code{@polyread()} annotation could also be used anywhere a \code{@mutable()}
annotation is allowed. It is not clear to us whether a \code{@readonly()}
annotation would be useful, since for any closure that declares
a \mbox{\code{@mutable()}} or \code{@polyread()} annotation, anything that is not
mutable or polyread is assumed to be readonly.
If no annotations are specified or inferred, then we assume \code{@mutable(all)}.

Alternatively, we could use simple \code{@pure} / \code{@impure} annotations
on closures, where \code{@impure} would be used wherever the closure
is mutable in any scope or any partially-applied parameter.
This approach is less precise for closures than for direct method applications.

\end{comment}


%closures
%1st-class functions (closures) as parameters

%alternatives

%io

%Handling traits

\subsection{Purity}

If all parameters to a method are \emph{readonly} or \emph{polyread},
then the method is \emph{pure} or \emph{side-effect free}.
Parameters include those specified by annotations on the method,
so purity requires that the method be \emph{readonly} or \emph{polyread}
in all of its enclosing scopes and all packages.
Since memory side effects are only permitted through \emph{mutable} references,
and since there is no way to obtain a mutable reference from
a \emph{readonly} or \emph{polyread} parameter, the foregoing is a
sufficient condition for method purity.

\subsection{Local Type Inference}

One good reason to perform inference on Reference Immutability (RI) types is that inference
reduces the annotation burden -- the programmer does not need to
explicitly declare RI types where they can be inferred from the context.
One approach to RI inference is to build an inference algorithm that runs
separately from the type checker -- this is the approach taken by ReIm~\cite{reim}.
ReIm's inference algorithm is guaranteed to find a maximal typing.
The disadvantage of this approach is that certain complexities
in the Scala type system (such as covariance, contravariance, and trait linearization)
will increase the complexity of type inference and checking.
Another approach is to rely on Scala's built-in type inference mechanism,
which is the approach attempted for Efftp~\cite{efftp}.
The potential advantage of this second approach is that many complexities
in Scala can be handled automatically.
One disadvantage of this approach is that more annotations may be required to
achieve a maximal typing -- but on the other hand, RI types would behave
similarly to other types.\footnote{I should note here that Efftp's difficulties are
likely due to the attempt to use Scala's type inference mechanisms to
propagate arbitrary effect types, which do not necessarily
behave like standard data types. RI types, on the other hand, are merely constraints
on references, and so can be propagated very much like standard data types.}

A few modifications of Scala's type inference system are required to implement
RI types.
First, the inference must be aware of the RI type hierarchy,
specifically that \code{mutable} is a subtype of \code{polyread} and
\code{polyread} is a subtype of \code{readonly}.
Each standard data type also has an RI type.
If no RI type has been declared or can be inferred, the RI type defaults to \code{mutable}.
Whenever the type inferencer performs a least-upper-bound (LUB) or
greatest-lower-bound (GLB) operation on a data type, a LUB or GLB is also performed
on the RI type.

A type is considered equal to another type if and only if
the data types are equal \emph{and} the RI types are equal
(and similarly for subtype relationships).
I initially planned on ignoring the RI type when computing type equality
and subtyping relationships to ensure that adding RI annotations would never change
the program semantics. However, ignoring the RI type would also mean that I forfeit
use of Scala's built-in mechanisms for type-checking.
(Type checking for a language like Scala is non-trivial due to covariance,
contravariance, abstract types, structural types, and other features.)
My current approach is to allow the Scala type-checker to consider
RI types. In the case of unannotated code, all RI types are defaulted to
\code{mutable}, resulting in no change to program semantics.
If annotations are added, then those annotations will be taken into account
when subtyping relationships are evaluated.
Changed subtyping relationships could cause Scala's inference engine
to make different decisions about which overloaded methods or which
implicit functions most closely match a given context.
However, these changes should not be problematic in general;
one of the following situations must apply for any given decision about overloading choices:
1) adding annotations results in no change to the choice of overload;
2) adding annotations results in the previous best choice to become non-best,
	resulting in a different selection;
3) adding annotations results in the previous best choice to be rejected due
to incompatible RI types, resulting in a different selection; or
4) adding annotations results in all overloads to be rejected due to incompatible RI types,
	resulting in an error.
%We assert that all of these situations are correct in theory,
In theory, the selection of a different overloaded method should not change
a program's (abstract) meaning,
%In theory, all of these cases should be type-correct and preserving of program meaning,
although in practice the process of adding annotations to existing software
may break the code in unexpected ways if
the implementations of different overloads produce inconsistent results.
%it depends on overloads being selected in a particular way,
%overloads or implicits have inconsistent implementations.
%the code does not carefully manage overloads or implicits.
I expect that the \emph{mutable} default will help in practice,
since unannotated overloads (all \emph{mutable} parameters) will tend not to match annotated code
(which is likely to use at least a few \emph{readonly} or \emph{polyread} variables),
generating a match failure to remind programmers to re-examine the implicits or overloads used.

\subsubsection{Where Default Type Inference and Checking Must Be Overridden}

%then one of the following situations
%will apply whenever Scala's inference engine makes a decision
%If annotations are present, it is possible that the compiler may make different
%selections of implicit functions or overloaded methods if the preferred
%selection (before annotation) is disallowed due to incompatible RI types.
%Disallowing method selections with incompatible RI types is the behaviour
%we want; the only question is whether we allow alternative methods to be
%selected silently, or do we force an error?

RI types are like ordinary types in the sense that they represent constraints
on the usage of fields and local variables.
Their similarity to ordinary types is what makes them reasonable candidates for
integration with Scala's default type inference mechanisms.
However, exceptions to the default operation of the inference system must be made
to account for the cases where RI types differ semantically from ordinary types.
These cases include viewpoint adaptation, the transitive \emph{readonly} guarantee, and
checking the extra parameters discussed in section~\ref{basic-annotations}.
I note that (unlike Efftp~\cite{efftp}) none of the changes we propose to make involves
transforming the abstract syntax tree or otherwise changing the default order
in which types are inferred.

%are exceptions to ordinary type rules, so some exceptions to the default
%type inference mechanisms must be made.

The first exception to default inference mechanisms is modification of the returned type
at call sites.
Where a method application would ordinarily have RI type \emph{polyread},
rule definition~\ref{viewadapt-def} must be applied to determine the correct RI type.

The second exception is the modification of RI types along access paths.
Whenever the type of a field selection is determined,
rule definition~\ref{accesspath-ri-def} must be applied to determine the
correct RI type.

The third exception is the modification of the RI type of
each standalone identifier use and
the RI type of the first identifier in every access path.
For correct RI type checking, each such identifier (except for local variables,
which are accessed through the local frame) must be understood as
being accessed through some path leading from an extra parameter (section~\ref{basic-annotations}).
Access path rules apply here, so an identifier that refers to a variable in
a \emph{readonly} scope must be \emph{readonly}.

The fourth exception is the checking of the RI types of the extra parameters at call sites and
object instantiation sites. Declared parameters are checked automatically,
but extra checks of the RI type-correctness of the receiver object and closure references
must be performed.
%Because it is possible for extra parameters to have the type \emph{polyread},
%viewpoint adaptation must take the extra parameters into account.

One area where an exception does \emph{not} have to be made is support for arrays or
multidimensional arrays.
Unlike Java, Scala has substantial built-in support for generic types,
and arrays are implemented as generic types.
%If we are careful about how RI types are defined, then 
%we should get RI typing on arrays for free.

%Not overriding type qualifiers?
%Readonly on classes?

\begin{comment}
\subsubsection{Pluggable Type Systems}

%not enough examples of truly pluggable problems!
% generic plug-in architecture a waste of time

One open question is how a general plug-in interface
for a type inference/checking system should work.
My experience, however, leads me to believe that attempting to implement
a general interface or framework without several well-motivated concrete
examples of expected usage generally leads to poor design and engineering decisions.
Therefore, I plan to implement the reference immutability system as a fork
of the Scala compiler, which is the fastest way to develop the system in the short term.
I expect that this work will yield some observations
that will help answer the question of what kinds of features a pluggable type-system
interface should expose,
but I leave the implementation of such an interface for future work.
\end{comment}

%Efftp~\cite{efftp} is built on one possible interface design~\cite{lightweight-polymorphic-effects},
%but the evidence regarding the utility of that design is far from conclusive.
%until we can answer the question of what kinds of features
%we want such plug-ins to implement and what 


%\input{framedef_examples_content}

%Purity discussion
%This is a stronger definition of prestate**? than used by Rountev~\cite{precise-sideeffectfree-java},

%orthogonal conerns:
% definition of purity, and its effect on precision.
% discussion of freshness and locality, and relationships with reference immutability.
% local vs. global inference (differences in intent)
% anything said about verification objectives or concurrency objectives?

\section{Evaluation and Plan for Completion}
\label{evaluation}

%evaluation

%implementation in Dotty

I initially plan to implement RI types for the Dotty~\cite{dotty} compiler,
which is an experimental new front-end for Scala.
If time permits, I may also attempt to implement a plug-in for
the production Scala compiler, which would allow RI types to be used
in production code.

%sanity check

I plan to evaluate the RI types implementation
by running Dotty on a number of Scala projects,
which may include the Scala compiler, the Dotty compiler,
the Scala standard library, the PSP collections library~\cite{psp-page},
and others.
Initially, I will run a sanity check;
I want to make sure that, in the absence of annotations,
each of these projects type-checks cleanly.

Starting with code that type-checks cleanly, I plan to incrementally add RI annotations
to some projects.
I expect to answer the following questions, using concrete examples of
RI-annotated code to justify the answers:
\begin{itemize}
	\item Does the proposed RI type system allow for easy incremental annotation
	of existing code, or are there cases where large numbers of annotations must
	be added at the same time?
	\item Can the proposed RI type system achieve near-maximal RI typings
	with few annotations,
	or does most code have to be very heavily annotated to achieve near-maximal typings?
	\item Is the proposed RI type system precise enough to express method purity where purity exists,
	or is the RI system too imprecise to express method purity in some instances?
\end{itemize}

%Scala compiler
%Scala standard library
%PSP collections library

The RI type implementation will have some compile-time overhead, but
that overhead should be negligible.
Nevertheless, I plan to run some tests to determine how much extra time
and memory is consumed by the compiler when RI types are enabled.

The RI type system includes local inference, but
a global inference system
(such as ReImInfer~\cite{reim})
is left as a possibility for future work.
%Implementation of an RI inference system is a possibility for future work.

\section{Related Work}
\label{related-work}

Key related works include Javari, ReIm, JPure, and Efftp.
Also related are ownership types, type-and-effect systems,
other immutability type systems, and purity inference systems.

% BIG related works section!

\subsection{Javari}

Javari~\cite{javari} is an extension of Java that supports reference immutability annotations.
The basic annotations supported by Javari are similar to mine, except that
I do not permit the \code{mutable} annotation to make an exception to
\emph{readonly} guarantees,
and I use the term \code{polyread} in place of Javari's \code{romaybe}.
I also have slightly different rules for RI typing along access paths --
notably, \emph{mutable} field accesses through \emph{readonly}
references are \emph{readonly} in my rules where Javari types them as \emph{mutable}.
Javari has features to help programmers incrementally annotate large programs --
Javari allows references to be cast to different RI types,
and checks the correctness of potentially unsafe assignments at runtime.
Javari also clarifies the relationship between \emph{assignability},
which involves permissions to write to a single field, and
reference immutability, which involves transitive permissions to
write to entire objects.

\subsection{ReIm and ReImInfer}

ReIm~\cite{reim} is another reference immutability system for Java that
was designed to use reference immutability to infer method \emph{purity}.
The motivation for ReIm and its associated inference system, ReImInfer,
was to determine method purity to assist Universe Type inference (see section~\ref{universe-types}).
The access path type rules are the same as Javari, except that
fields are never permitted to be \emph{mutable}.
In my work, I allow fields to be mutable to create a logical consistency
between fields and local variables, but I redefine access path typing rules
to prevent \emph{mutable} fields from breaking \emph{readonly} guarantees.
I also have different rules for viewpoint adaptation of methods.
Where ReIm attempts to adapt methods based on assignment context,
I adapt methods based on argument types.
My approach is designed to integrate with Scala's local type inference,
which does not provide a method application's assignment context prior to the time
the method application's return type must be determined.

ReImInfer performs a global inference of reference immutability.
ReImInfer infers a best (maximal) reference immutability typing for any program.
However, ReImInfer does not support generic types~--
since the best type for each reference is chosen independently of
other references, the resulting typing is not guaranteed to be consistent
if generic RI types are involved.

\subsection{JPure}

JPure~\cite{jpure} is a purity annotation and inference system for Java.
The annotations in JPure as designed to be modularly checkable.
The key idea behind JPure is that pure methods are allowed to
modify \emph{fresh} objects, or objects that are not visible from the method's
\emph{prestate}.
By definition, objects are fresh when they are created,
and remain fresh even when returned from pure methods.
The notion of \emph{freshness} is extended to include
an object's \emph{locality}, which includes all objects transitively reachable through \emph{local}
fields.

JPure's semantics are nearly identical to reference immutability semantics.
JPure's \emph{local} and \emph{fresh} are both equivalent to \emph{mutable}.
The opposite concepts, \emph{nonlocal} and \emph{stale}, are equivalent to \emph{readonly}.
Where a method in JPure is labelled \emph{pure}, it has the same meaning
as purity under reference immutability~-- all parameters, including the receiver reference,
are \emph{readonly}. Where a method is labelled \emph{local},
it may take \emph{local} (\emph{mutable}) parameters, but other parameters are \emph{readonly} by default.
JPure calls \emph{local} methods \emph{pure} because they do not modify non-\emph{fresh}
state. I do not include such methods in my definition of \emph{purity},
since they do in fact modify their own prestates~-- I consider the question of
whether a method can modify the prestates of its callers to be a somewhat different
issue.

\begin{comment}
Surprisingly, I find that the guarantees offered by JPure's annotations
are nearly identical to ReIm's guarantees.
%A straightforward transformation exists to convert JPure-annotated programs into
%ReIm-annotated programs.
A \emph{fresh} reference in JPure is a \emph{mutable} reference in ReIm,
and conversely a \emph{stale} reference in JPure is a \emph{readonly} reference in ReIm.
\emph{Local} fields are \emph{polyread} fields, and \emph{local} parameters are \emph{mutable} parameters.
Assignment of a \emph{stale} reference to a \emph{local} field is not
permitted, nor can a \emph{stale} reference be returned from a \emph{fresh} method,
just as \emph{readonly} reference is not compatible with \emph{mutable}.
\end{comment}

\subsection{Efftp}

Efftp\footnote{The name ``efftp'' is not given in the paper, but it is the
implementation's project name in the \code{github} source code repository.}~\cite{efftp,rytz-thesis}
is a type system extension for Scala that can express JPure-like notions
of freshness, locality, and purity.
The basic annotations supported by Efftp include
\code{@mod()}, which indicates the parameters whose localities a method may modify;
\code{@loc()}, which indicates the localities a method may return;
\code{@assign()}, which indicates the localities may be aliased by a method;
\code{@local}, which may be applied to fields and parameters as in JPure;
and \code{@pure}, which indicates that a method does not have side effects.
Efftp is designed to be modularly checkable (like JPure) and flow-insensitive (unlike JPure).
Efftp is built on a generalized effect-typing framework~\cite{lightweight-polymorphic-effects,rytz-thesis}.

Efftp's notion of \emph{freshness} is more general than JPure's because
Efftp can express that a returned reference is fresh if the arguments to
certain parameters are fresh (\emph{conditional freshness}).
The proposed RI type system uses \emph{polyread} to achieve a similar result:
a \emph{polyread} return type is considered \emph{mutable} (\emph{fresh}) if all \emph{polyread} parameters
are given \emph{mutable} arguments.
However, Efftp's annotation system allows effect annotations to be placed
on a method's return type, which complicates interactions with Scala's
type inference system.
Effects in general are descriptions of behaviours of blocks of code,
not constraints on the uses of individual references or objects.
In my view, the former are properly expressed as annotations on methods or code blocks,
while the latter can be properly expressed as annotations on types, variables, or fields.

\subsection{Ownership Types}

Reference immutability has its roots in Ownership Type systems~\cite{ownership-types}.
The original ownership types system by Clarke et al.~\cite{ownership-types-1998}
proposed that each object is \emph{owned} (or \emph{contained}) by another object, and that
each object is only permitted to contain references to: 1) its owner,
2) other objects with the same owner, or 3) other objects \emph{outside} its owner
(where \emph{outside} is the inverse of the \emph{inside} relation, which is the
transitive, reflexive closure of the owned-by relation~\cite{ownership-types}).
Adherence to these rules is called the \emph{owners-as-dominators} discipline.
Types are used to define the permitted ownership relationships, enabling the validity
of these relationships to be statically checked.
Unfortunately, \emph{owners-as-dominators} is too inflexible to allow common programming patterns.
For example, consider creating an iterator over a list that owns its data elements:
if the iterator is \emph{inside} the list, then no objects \emph{outside} the list could
access the iterator; if the iterator is \emph{outside}, then the iterator could not
access the data.
There have been a number of possibilities proposed to improve the flexibility of
ownership systems, including temporary relaxation of restrictions for stack-based
variables~\cite{ownership-disjoint}, granting temporary permissions for methods to
access objects accessible to their callers~\cite{jot2000,external-uniqueness},
allowing exceptions for \emph{readonly} references (the \emph{owners-as-modifiers} discipline)~\cite{universes},
allowing programmers to flexibly declare ownership structures (\emph{ownership domains})~\cite{ownership-domains},
\emph{multiple ownership}~\cite{multiple-ownership}, and \emph{owners-as-ombudsmen}~\cite{owners-as-ombudsmen}.

\subsubsection{Universe Types}
\label{universe-types}

Originally inspired by Flexible Alias Protection~\cite{flexible-alias},
the Universes system~\cite{universes} provides lightweight annotations to express
the \emph{owners-as-modifiers} discipline for a subset of Java.
Unlike \emph{owners-as-dominators}, which restricts \emph{where} references
can point, \emph{owners-as-modifiers} restricts \emph{how} references can be used.
Enforcement of the \emph{owners-as-modifiers} discipline only requires that references
passed across ownership boundaries are declared \emph{readonly}.
The Generic Universe Types system extends Universes to support generic types~\cite{generic-universe-types}.

\subsubsection{Ownership Types for Region-based Memory Management and Parallelism}

Ownership types can be used for region-based memory management~\cite{ownership-regions}.
In region-based systems, objects are owned by named \emph{regions} rather than other objects.
Deterministic Parallel Java (DPJ)~\cite{deterministic-parallelism} uses regions to partition the heap, and
relies on method annotations to determine which regions are affected by the execution of the method.
If two methods operate on disjoint regions, then they can be safely executed in parallel.
In addition to verifying the disjointness of method effects, DPJ performs static checks
to ensure that annotations on methods soundly describe the actual effects generated by
those methods.

SafeJava~\cite{safejava} uses region-based memory management to support real-time programming.
Every object is allocated in a \emph{region}, and when the region is deleted,
all objects it contains are also deleted (avoiding the overhead of garbage collection).
SaveJava uses a region-based effect typing system to statically ensure that
deletion of regions is safe~-- that is, that no accesses to objects in deleted regions can occur.
SafeJava implements a relaxed owners-as-dominators system: the owner of an object can be changed
if there is a unique reference to that object, and inner classes can access their enclosing classes
even if their owners are different.

Reference immutability by itself does not provide strong enough guarantees
to support region-based memory management or automatic parallelization.
In particular, a \emph{readonly} reference does not prevent \emph{mutable}
references to the same object, or guarantee that the \emph{readonly} reference is unique.
The former guarantee is \emph{object immutability}, which is supported by other immutability systems
(section~\ref{other-immutability-systems}).
\emph{Fractional permissions}~\cite{fractional-permissions} are one way of providing the latter guarantee~--
fractional permissions allow copies of a reference to be distributed while allowing
the uniqueness of the original reference to be recovered later.


\begin{comment}
\subsection{Acyclic Region Type System (ARTS)}

Lu and Potter~\cite{} created the
Acyclic Region Type System (ARTS) to statically detect and prevent
unwanted cycles in object reachability graphs.
%, observe that 

\end{comment}

\subsection{Type-and-Effect Systems}

Some work has been done to define general frameworks for extending type systems
to support effect annotations. The aim of these frameworks is to reduce the
time and effort required to develop type systems to handle specific effects.
Marino and Millstein~\cite{generic-type-and-effect} propose a framework that
supports the notion of ``effects as privilege checking.''
To implement a specific effect system, Marino and Millstein's framework must be
given a description of a \emph{privilege discipline} that meets monotonicity requirements.
Provided the given privilege discipline meets monotonicity requirements,
the resulting effect system is guaranteed to be sound.
Rytz et al.~\cite{lightweight-polymorphic-effects,rytz-thesis} created a
flow-insensitive, polymorphic, modularly-checkable type system extension
for Scala. Rytz et al.'s work builds on Marino and Millstein's theoretical framework,
but adds support for polymorphic effects and modularly-checkable annotations.
Schwerter et al.~\cite{gradual-effect-systems} extend Marino and Millstein's framework
with a notion of \emph{unknown effects} to soundly support the gradual addition of effect
annotations to large programs.

%type-and-effect-systems

\subsection{Other Immutability Systems}
\label{other-immutability-systems}

Immutability Generic Java~(IGJ)~\cite{igj}
supports reference immutability, object immutability, and \emph{class immutability},
which guarantees that all instances of an immutable class are also immutable.
References in IGJ are \emph{readonly}, \emph{mutable}, or \emph{immutable}.
\emph{Immutable} references can only refer to immutable objects,
\emph{mutable} references can only refer to mutable objects,
and \emph{readonly} references can refer to any object.
By necessity, mutation of immutable objects is allowed during construction.

OIGJ~\cite{oigj} extends IGJ by including ownership annotations
and a \emph{raw} reference type.
OIGJ allows immutable objects to remain mutable (``raw'') for some time after construction,
but become immutable (``cooked'') later, permitting (for example) the construction of immutable circular lists.
OIGJ's ownership system supports the extended cooking phase --
an object is cooked when its owner is cooked.

\mbox{Joe\textsubscript{3}}~\cite{joe3} supports reference, object, and class immutability, and
also addresses certain criticisms~\cite{boyland-readonly} of reference immutability.
In particular, \mbox{Joe\textsubscript{3}} can express that a method parameter
is not \emph{captured} by its method, that a \emph{readonly} reference is unique
(and therefore refers to an immutable object), or that a unique reference is
temporarily duplicated, but becomes unique again at a later time (\emph{fractional permissions}).
These features can prevent \emph{observational exposure} (which allows state changes
to be observed through \emph{readonly} references).


\subsection{Other Inference Systems}

Sălcianu and Rinard~\cite{purity-effect-java} created a side-effect analysis that
detects method purity, \emph{readonly} parameters, and \emph{safe} parameters.
Although my work is focussed on local type inference rather than whole-program analysis,
Sălcianu and Rinard's definitions of method \emph{purity} and \emph{readonly} parameters
are the same as mine and ReIm's.
% maximally unaliased arguments
Sălcianu and Rinard's analysis assumes that the parameters of any given method
are maximally unaliased.
The maximally-unaliased assumption allows compositionality;
a single parameterized result is generated for each method analysed.
Aliasing relationships among parameters, if they exist, are computed at
method call sites.
%\emph{safe} parameters
A method parameter is \emph{safe} if the method does not mutate the locality of
that parameter (the parameter is \emph{readonly}) and the method
does not create any new access paths that allow the parameter's locality
to be seen from the method's prestate.
It may be possible to extend the RI type system to include notions of \emph{safety}
in addition to mutability.

Kneuss et al.~\cite{effect-analysis-callbacks} build on Sălcianu and Rinard's work
to create a side-effect and purity analysis for Scala.
Kneuss et al.'s inference system performs a flow-sensitive analysis and
provides support for first-class function parameters (used for callbacks).
The system infers read- and write-effects from the bytecode generated
by the Scala compiler.
Like Sălcianu and Rinard's system, Kneuss et al.'s system performs whole-program
(rather than local) inference.

%All the stuff that's really related.

%Controlling effects:
%RI, Immutability in general
%Fractional permissions
%Inferring effects?

%Verification:
%Universe types
%Separation Logic

%Ownership

%Inferring Aliasing relationships
%Controlling

\section{Conclusion}
\label{conclusion}

Reference Immutability (RI) types allow programmers to provide deep
or transitive guarantees that no writes will be performed on objects
reached through certain references.
I plan to implement RI types as an extension to the Scala type system.
Unlike existing type-and-effect systems for Scala, my approach handles RI types
as much as possible like ordinary data types.
There are some details of the RI implementation that still need to be resolved,
particularly with respect to generic types and the types of closure references.
A key concern is that the design of the RI system always protect the
transitive nature of \emph{readonly} guarantees without placing too many
unnecessary constraints on the programmer.
%However, I am confident that reasonable solutions to the RI system design
%do exist.


%\bibliographystyle{alpha}
%\bibliographystyle{plain}
\bibliographystyle{abbrv}
\bibliography{../references}

\end{document}
