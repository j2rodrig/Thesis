\documentclass[letterpaper,11pt]{article}

\usepackage{alltt}
\usepackage{amsmath}
\usepackage[greek,english]{babel}
%\usepackage{booktabs}
%\usepackage{cite}
%\usepackage{color}
\usepackage{fixltx2e}
\usepackage[T1]{fontenc}
%\usepackage{fullpage}
\usepackage[top=1in, bottom=1in, left=1in, right=1in]{geometry}
\usepackage{graphicx}
\usepackage[utf8]{inputenc}
\usepackage{listings}
%\usepackage{mathtools}
\usepackage{mdwlist}
%\usepackage{needspace}
\usepackage{url}
\usepackage{verbatim}
\usepackage{xspace}

% Fonts and symbols
%\usepackage{amsfonts}
%\usepackage{amssymb}
%\usepackage[scaled=.65]{beramono}
\usepackage{courier}
\usepackage{times}

\newenvironment{absolutelynopagebreak}
  {\par\nobreak\vfill\penalty0\vfilneg
   \vtop\bgroup}
  {\par\xdef\tpd{\the\prevdepth}\egroup
   \prevdepth=\tpd}

\newcommand{\greekmu}{\greektext m\latintext}

\newcommand{\tab}[0]{\hspace{0.4cm}}

\renewcommand{\ttdefault}{txtt}

\lstset{
	basicstyle=\footnotesize\ttfamily,
	tabsize=4,
	extendedchars=true,
	breaklines=true,
	escapeinside={(*}{*)},
	frame=l,  % trbl for single-line frames, TRBL for double-line frames
	xleftmargin=1.5\parindent
}

\newcommand{\code}[1]{\lstinline$#1$}



\begin{document}
	
	{}

Generic type parameters may be overloaded by any mutability.

Where mutability checks are needed:
- defs: rhs TMT <: lhs TMT
- assignment
- overloaded def/var/val: parameters are contravariant, results are covariant.
	
%Inheritance must preserve the mutability of generic type parameters.


	trait A {
		type T
		def foo(p1: T)
		def z: T
		val w: T
		var v: T
		w.x = y
	}

	trait B extends A {
		type U = T @readonly
		override def foo(p1: U)   // Error: original definition of p1 is not @readonly
		override def z: U
		override val w: U
		override var v: U         // Error: v_=(p1) parameter mismatch: original p1 is not @readonly
		w.x = y  // not allowed - w is @readonly
		v = A.super.v
	}
	
	{}

Only method types may use \code{@polyread}.

Parameterless definitions %(ExprType)
have exactly the result type declared.
This rule creates a consistency between variable defintions and parameterless
method definitions, in that changing a defintion from a \code{var} to a \code{def}
does not affect its type.

%example here

%(PolyType)

%(MethodType)


\section{}

The TMT of a variable depends on where that variable is viewed from.
Just as a physical object can block line-of-sight,
the presence of a \emph{readonly} constraint can block mutation privileges.
If a \emph{readonly} constraint is present anywhere along an access path (the ``line-of-sight''),
then the ability to perform mutation is blocked.

At any location in the source code, the \emph{readonly} constraint is
\emph{present}, \emph{absent}, or \emph{conditional}.
Where the constraint is present, the type is \emph{readonly};
where it is absent, \emph{mutable}; and
where it is conditional, \emph{polyread}.

The \emph{polyread} type requires further explanation --
i.e., \emph{polyread} is conditional upon what?
The normal use of \emph{polyread} is as a generic type to reduce code duplication
where a method may be called from either a \emph{readonly} viewpoint
or a \emph{mutable} viewpoint~(cf. \emph{romaybe} in Javari~\cite{javari}).
ReIm~\cite{reim} also uses \emph{polyread} for viewpoint-adapatable fields;
but since I allow \emph{mutable} fields to be viewpoint-adapted,
I drop this meaning of \emph{polyread}.
The meaning of \emph{polyread} at a particular location in the source code
depends upon the method whose result type is being adapted.
Polyread

There are potentially as many \emph{polyread} types as methods in the source code~--
each 


The subtyping relationship is:
$M <: P_o <: R$ for any $P_o$,
where $o$ is the \emph{origin} of the type $P_o$.
The origin is the method symbol.

By default, \emph{polyread} types can only be declared on method arguments and return types.
For such declarations, the origin is the method symbol.
Local variables may have inferred \emph{polyread} types, in which case the inferred type has the
same origin as the type of the right-hand expression.

In code examples, we denote $P_o$ by the annotation \code{@polyread(o)}.

Nested methods present a challenge.
For example:

\begin{lstlisting}
	def n(n1: @polyread(n), n2: @polyread(n)): @polyread(n) = {
		def m(m1: @polyread(m), m2: @polyread(m)): @polyread(m) = {
			m(m1, m2)   // result type @polyread(m)
			m(n1, n2)   // result type @polyread(n)
			n(m1, m2)   // result type @polyread(m)
			n(n1, n2)   // result type @polyread(n)
			m(n1, m1)   // result type @polyread(n) | @polyread(m)
			n(n1, m1)   // result type @polyread(n) | @polyread(m)
		}
		n(n1, n2)   // result type @polyread(n)
		m(n1, n2)   // result type @polyread(n)
	}
\end{lstlisting}

A few things to note about the previous listing:
First, that the return type of a polyread method application depends only
on the argument types.
Second, method \code{m} should be allowed to return \code{n1}, \code{n2},
or other expressions yielding type \code{@polyread(n)}.
An application of method \code{m} belongs to a particular instance of method \code{n}

\begin{lstlisting}
	def n(n1: @polyread(n), n2: @polyread(n)): @polyread(n) = {
		var nv1 = n1
		def m(m1: @polyread(m), m2: @polyread(m)): @polyread(m) = {
			var mv1 = m1  // OK: type is @polyread(m)
			var nv2 = n1  // OK: type is @polyread(n)
			return nv2    // error: @polyread(n) is not compatible with @polyread(m)
			return mv1    // OK: type @polyread(m) is compatible with @polyread(m)
		}
	}
\end{lstlisting}


%\bibliographystyle{alpha}
%\bibliographystyle{plain}
%\bibliographystyle{abbrv}
%\bibliography{../references}

\end{document}
