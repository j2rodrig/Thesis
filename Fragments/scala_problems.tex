\documentclass[letterpaper,11pt]{article}

\usepackage{alltt}
\usepackage{amsmath}
\usepackage[greek,english]{babel}
%\usepackage{booktabs}
%\usepackage{cite}
%\usepackage{color}
\usepackage{fixltx2e}
\usepackage[T1]{fontenc}
%\usepackage{fullpage}
\usepackage[top=1in, bottom=1in, left=1in, right=1in]{geometry}
\usepackage{graphicx}
\usepackage[utf8]{inputenc}
\usepackage{listings}
%\usepackage{mathtools}
\usepackage{mdwlist}
%\usepackage{needspace}
\usepackage{url}
\usepackage{verbatim}
\usepackage{xspace}

% Fonts and symbols
%\usepackage{amsfonts}
%\usepackage{amssymb}
%\usepackage[scaled=.65]{beramono}
\usepackage{courier}
\usepackage{times}

\newenvironment{absolutelynopagebreak}
  {\par\nobreak\vfill\penalty0\vfilneg
   \vtop\bgroup}
  {\par\xdef\tpd{\the\prevdepth}\egroup
   \prevdepth=\tpd}

\newcommand{\greekmu}{\greektext m\latintext}

\newcommand{\tab}[0]{\hspace{0.4cm}}

\renewcommand{\ttdefault}{txtt}

\lstset{
	basicstyle=\footnotesize\ttfamily,
	tabsize=4,
	extendedchars=true,
	breaklines=true,
	escapeinside={(*}{*)},
	%frame=l,  % trbl for single-line frames, TRBL for double-line frames
	xleftmargin=1.5\parindent
}

\newcommand{\code}[1]{\lstinline$#1$}



\begin{document}

\def\redcolor{\color{red}}
\def\greencolor{\color{green!60!black}}
\def\bluecolor{\color{blue}}
\def\blackcolor{\color{black}}


%Scala-specific problems
% Background and Previous work
% Nested Methods/Deeply nested methods
% Classes within methods/Classes within classes
% Closures
% Annotation system (with defaults)


\section{Scala-specific Problems}


\subsection{Problems with the Previous Solution}

One purpose of this work is to permit programmers to declare that certain methods are
functionally pure.
Pure methods cannot modify anything in their prestates, including variables in
their enclosing environments.

Scala permits code like the following:

\begin{lstlisting}
	def f() = {
		var v = ...
		def g() = {
			v = ...
		}
		g()
	}
\end{lstlisting}

It should be possible to prove that \code{f} is pure
even though \code{g} modifies \code{v}.
Under the previously-propsed system, \code{g} was implicitly passed a reference to the runtime frame of \code{f},
and accesses to variables defined in \code{f} use this reference (implicit elements are shown within comments):

\begin{lstlisting}
	def f() = {
		var v = ...
		def g(/*f: @mutable*/) = {
			/*f.*/v = ...
		}
		g(/*current_stack_frame*/)
	}
\end{lstlisting}

I proposed that the annotation \code{@mutable(f)} be placed on method \code{g} to denote the
mutability of the implicit parameter.
However, I have discovered that Scala does not permit arbitrary symbols as arguments to annotations.
Rather, annotation arguments must be compilable expressions.

Where a method name is passed as an argument,
it is handled like a method application:

\begin{lstlisting}
	def f() = {
		var v = ...
		@mutable(f) def g() = { // error: cyclic reference involving method f
			/*f.*/v = ...
		}
		g(/*current_stack_frame*/)
	}
\end{lstlisting}

For similar reasons, class and trait names cannot be passed directly as arguments:

\begin{lstlisting}
	class C {
		@mutable(C) def f() = ...        // error: not found: C
		@mutable(this) def f() = ...     // OK
		@mutable(C.this) def f() = ...   // OK
		@mutable(new C()) def f() = ...  // OK (although perhaps not useful)
	}
\end{lstlisting}

Therefore, a new solution is needed.

\subsection{New Solution}

I think it makes more sense to pass references to enclosing-scope variables as parameters.
For example:

\begin{lstlisting}
	def f() = {
		var v = ...
		def g(/*reference_to_v: @mutable*/) = {
			v = ...
		}
		g(/*reference_to_v*/)
	}
\end{lstlisting}

Using an annotation to declare the mutability of \code{v} in \code{g}:

\begin{lstlisting}
	def f() = {
		var v = ...
		@mutable(v) def g(/*reference_to_v: @mutable*/) = {
			v = ...
		}
		g(/*reference_to_v*/)
	}
\end{lstlisting}


\subsection{Accessing Class/Trait Environments}

Classes and traits can be nested within methods or other classes/traits.
Methods of those classes can mutate variables in the outer environment:

\begin{lstlisting}
	def f() = {
		var v = ...
		class C() {
			def g() = {
				v = ...
			}
		}
		val c = new C()
		c.g()
	}
\end{lstlisting}


The reference to \code{v} is captured as a member of class \code{C}.
Showing extra parameters:

\begin{lstlisting}
	def f() = {
		var v = ...
		class C(/*reference_to_v: @mutable*/) {
			def g(/*this: @mutable*/) = {
				/*this.*/v = ...
			}
		}
		val c = new C(/*reference_to_v*/)
		c.g(/*c*/)
	}
\end{lstlisting}

The reference to \code{v} must be available where the instance of \code{C} is created,
but it does not need to be available at the call site of \code{c.g}.

\begin{lstlisting}
	def f() = {
		var v = ...
		@mutable(v) class C(/*reference_to_v: @mutable*/) {
			@mutable(this) def g(/*this: @mutable*/) = {
				/*this.*/v = ...
			}
		}
		val c = new C(/*reference_to_v*/)
		c.g(/*c*/)
	}
\end{lstlisting}


\begin{lstlisting}
	trait E() {
		def g()
	}
	def f(): E = {
		var v = ...
		class C() extends E() {
			def g() = {
				v = ...
			}
		}
		val c = new C()
		return c
	}
	val e = f()
	e.g()
\end{lstlisting}


\begin{lstlisting}
	trait E() {
		def g()
	}
	def f(): E = {
		var v = ...
		class C() extends E() {
			def g() = {
				v = ...
			}
		}
		val c = new C()
		return c
	}
	val e = f()
	e.g()
\end{lstlisting}





\begin{lstlisting}
	def f() = {
		var v = ...
		def g() = {
			def h() = {
				v = ...
			}
			h()
		}
		g()
	}
\end{lstlisting}

\begin{lstlisting}
	def f() = {
		var v = ...
		def g(/*v: @mutable*/) = {
			def h(/*v: @mutable*/) = {
				v = ...
			}
			h(/*v*/)
		}
		g(/*v*/))
	}
\end{lstlisting}



\subsection{Paths to }

	class C {
		var v = ...
		class D {
			def f() = {
				v = ...
			}
		}
	}

\section{Closures}

Closure creation involves instantiating an anonymous closure class.
For example, consider the following creation and execution of a closure:

\begin{lstlisting}
	def f() = {
		var v = ...
		def g() = {
			v = ...
		}
		val cl_g = g _   // partial evaluation of g
		cl_g()           // execution of g
	}
\end{lstlisting}

The above is roughly equivalent to:

\begin{lstlisting}
	def f() = {
		var v = ...
		def g() = {
			v = ...
		}
		class AnonymousClosure() {
			def apply() = g()
		}
		val cl_g = new AnonymousClosure()
		cl_g.apply()
	}
\end{lstlisting}

Adding implicit parameters:

\begin{lstlisting}
	def f() = {
		var v = ...
		def g(/*v: @mutable*/) = {
			v = ...
		}
		class AnonymousClosure(/*v: @mutable*/) {
			def apply(/*this: @mutable*/) = {
				g(/*this.v*/)
			}
		}
		val cl_g = new AnonymousClosure(/*v*/)
		cl_g.apply(/*cl_g*/)
	}
\end{lstlisting}

The \code{AnonymousClosure} class takes the same extra parameters as \code{g}.

\subsection{A Brief Note on Defaults}

Like the previous proposal, declaring some variables as \emph{mutable} via the \code{@mutable(...)} annotation
means that all other enclosing-scope parameters are \emph{readonly}.
An empty mutation list \code{@mutable()} means that all enclosing-scope variables are \emph{readonly}.

If no enclosing-scope annotations are given, then all enclosing-scope variables are assumed \emph{mutable}.


\end{document}
