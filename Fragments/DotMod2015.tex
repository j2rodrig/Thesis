\documentclass[11pt]{report}

\usepackage{fullpage}

\usepackage{booktabs}

\usepackage{amsmath,amssymb,amstext} % Lots of math symbols and environments
\usepackage[pdftex]{graphicx} % For including graphics N.B. pdftex graphics driver 

%custom packages
\usepackage{cancel}
\usepackage{comment}
\usepackage{courier}
\usepackage{fixltx2e}  % for subscript text in sections/bookmarks
\usepackage{listings}
%\usepackage[12pt]{moresize}
\usepackage{multirow}
%\usepackage{ulem}
\usepackage{url}
\usepackage[utf8]{inputenc}
\usepackage[usenames,dvipsnames,svgnames,table]{xcolor}

\newcommand{\cd}{\ttfamily\footnotesize}

\lstdefinelanguage{ScalaExamples}{
	morekeywords={
		def,var,val,object,package,class,abstract,trait,extends,with,
		match,case,
		import,
		this,super,override,
		private,protected,public,private[this],protected[this],
		final,sealed,null,new,
		for,do,while,until,yield,return},
	sensitive=true,
	morecomment=[l]{//},
	morecomment=[s]{/*}{*/},
	morestring=[b]"
}
\lstset{
	language={ScalaExamples},
	basicstyle=\cd,
	tabsize=2,
	keywordstyle=\bf\cd\color{Blue},
	identifierstyle=\bf\cd,
	commentstyle=\bf\scriptsize\color{gray},
	xleftmargin=\parindent,
	extendedchars=true,
	breaklines=true,
	captionpos=b,
	frame=b,
	float=htp,
	emph={@pure,@readonly,@bounded,@rothis,
		@mutEnv[C],mutEnv[C],@mutEnv[D],mutEnv[D],@mutable,@fresh},
	emphstyle={\bf\cd\color{OliveGreen}},
	moredelim=**[is][\bf\cd\color{OliveGreen}]{@*}{*@},
	escapeinside={(*}{*)}
}

\begin{document}


% all language structures with mutable/readonly
%  0. argument for conceptual simplicity over optimization
%  1. field selection
%  2. method selection
%  3. object construction and type selection
%  4. environment references and closures and package objects
%  5. accessor methods

% extension1: rothis, roparam, fresh

% defaults1: accessors, function types, bounded types, new objects, ...
%   function types and syntactic sugar

% holes1: bounded, asInstanceOf


% basic qualifiers
% qualifiers allowed on fields (vs local variables)

% List of bits to include:
%   rothis - polymorphism
%   viewpoint adaptation
%   closures
%   polymorphic purity
%   fresh results

\title{RIP-Scala:\\Compiler-checkable Reference Immutability and Purity\\for the Scala Language}
\author{Jonathan Rodriguez\\University of Waterloo}
\maketitle

% RESEARCH AGENDA

% does type substitution/refinement create holes in the system?
% under what conditions are methods really pure (not creating externally observable effects)?
% viewpoint adpation and lower bounds of type members
% acceptable qualifier defaults


\chapter{Introduction}

% the motivation behind this work

% background: reference immutability

% RI as a system of type qualifiers

%  lookahead: arbitrary nesting of classes and methods, environment references, closures,
%  env-reference polymorphism, freshness

The motivation behind this work is to help programmers reduce unexpected
mutations of shared state by means of a compile-time annotation-checking mechanism.
Unexpected mutation of shared state is a pervasive problem in object-oriented systems,
both single-threaded and concurrent.
{\em The Geneva Convention on the Treatment of Object Aliasing}~\cite{geneva}
pinned the blame for this problem on the unrestricted copying of object references,
and suggested that such aliasing may be managed through some combination
of aliasing detection, advertisement, prevention, and control.
Of particular interest in the current work are aliasing advertising
and prevention; alias advertising involves statically declaring (for example)
the aliasing operations that a method may perform, and alias prevention involves
the creation of mechanisms for statically checking alias advertisements.

One prominent attack on the problem of object aliasing was Ownership Types~\cite{ownership-types},
which introduced a statically-checkable notion of {\em object ownership},
where references to any given object cannot be created or passed outside of its owning object.
Although the original formulation of ownership types was inflexible
and annotation-heavy, it spawned the research into a large number of
lighter-weight and more-flexible systems of alias restriction.
%increasingly-flexible
%and lighter-weight ownership systems.  % such as universes, lightweight alias protection, etc.
One of the concepts that emerged from the exploration of more flexible systems
is the {\em owners-as-modifiers} discipline, which (unlike the
{\em owners-as-dominators} discipline of the original ownership types system)
allows arbitrary references into an ownership domain provided that
no mutations are performed on objects reached through those references.

{\em Readonly} references turn out to be useful for more than simple support of
ownership type systems.
Even in the absence of explicit aliasing or ownership information,
knowledge of which references are {\em readonly} is frequently sufficient to
prove that certain program methods are pure (or {\em side effect free}).
Where method purity is not guaranteed,
annotations of {\em readonly} references can still serve as compiler-checkable documentation
that certain objects (namely, those reachable through {\em readonly} references) are
not expected to be mutated.
The next chapter (chapter~\ref{chap-background}) discusses the strengths and limitations
of {\em readonly} as a reference annotation mechanism, and how {\em readonly} may be
combined with other types of knowledge to produce stronger guarantees.
Limitations notwithstanding, the addition of {\em readonly} remains a
worthwhile first step in the limitation of shared-memory side effects.

The current work explores a particular approach to introducing {\em readonly}
and related annotations into the Scala language.
Scala includes a number of advanced language-level features that complicate
the practical introduction of reference immutability.
These features include path-dependent type selection, synthetic accessor methods,
and arbitrary nesting of classes and methods.
Scala is not the only programming language with these features;
the ideas in the current work may be transferrable to languages with
similar features, although the investigation of such transferrence is
outside of the scope of the current work.
Correct handling of Scala's features necessitates a more complex approach to
reference immutability than previous approaches for Java,
particularly if the result is to be practical for use in production code.

The focus in the current work is on a compiler-checkable annotation system.
These annotations are compiler-checkable documentation of the ways in which programmers
expect annotated references to be copied and used.
The focus is furthermore on the limitation of shared-memory side effects
rather than the description of general side effects, although further work
may show that reference immutability is sufficient to describe stateful changes
in the external world as well as in memory.
Related dimensions of investigation not addressed by the current work
include run-time considerations, gradual typing considerations,
or whole-program inference.
Although investigation of these dimensions would be useful,
they are outside of the scope of the current work.

The current work defines reference immutability as a system of {\em type qualifiers}.
The type system restricts the set of allowed programs to those programs that
the compiler can prove type-correct; addition of reference-immutability {\em qualifiers}
further restricts allowed programs to those that are consistent with respect to
reference immutability constraints.
A multi-phase compiler plugin implements the necessary reference-immutability-related logic.

Chapter~\ref{chap-background} contains background on reference immutability and
on the Scala language.
Chapter~\ref{chap-mutable-readonly} presents a basic reference immutability system
for the Scala language, and chapter~\ref{chap-flexibility-scala} presents a set of
extensions to the basic system.
Chapter~\ref{chap-dot-calculus-emend} formalizes key parts of the system
as an extension to the DOT calculus.
The type qualifier system itself does not address all practical concerns;
the question of appropriate default qualifiers is discussed in chapter~\ref{chap-defaults},
and mapping of programmer-facing annotations to appropriate qualifiers is
addressed in chaper~\ref{chap-annotations}.
Chapter~\ref{chap-related-work} compares the current work with related work,
and chapter~\ref{chap-evaluation} evalutes the work in terms of support for
code idioms, compiler performance impact, and experience annotating
the standard library.
Future work is briefly described in chapter~\ref{chap-future-work},
and chapter~\ref{chap-conclusion} contains a conclusion.

\chapter{Background} \label{chap-background}
	
\section{Reference Immutability}

TBD. {\em Basic description. Possibly some examples in C/C++.}

%\section{How can Reference Immuabilty Help Prove Purity?}

\section{On the Limitations of Reference Immutability}

TBD. {\em Talk about observational exposure here.}

\section{The Scala Language}

TBD.




\chapter{Reference Immutability for Scala}
	\label{chap-mutable-readonly}
	% all language structures with mutable/readonly
%  0. argument for conceptual simplicity over optimization
%  1. field selection
%  2. method selection
%  3. object construction and type selection
%  4. environment references and closures and package objects
%  5. accessor methods

This chapter describes a simple reference immutability system for the Scala language.
Reference immutability is defined here as a system of type qualifiers (see sec.~\ref{sec-type-qualifiers}) that restrict mutations on certain objects.
To preserve reference immutability guarantees, viewpoint adaptation is performed
for field selections (sec.~\ref{sec-field-sel}) and method selections (sec.~\ref{sec-method-sel}).

Unlike previous work~\cite{javari,reim} for the Java languge, the current work
addresses certain new concerns that arise in the Scala language.
These concerns include the handling of accessor methods (sec.~\ref{sec-accessor-method-sel}),
abstract type bounds (sec.~\ref{sec-vp-bounded}),
and arbitrarily nested classes and methods (sec.~\ref{sec-env-ref}).
Environment reference qualifiers, which are introduced to
reasonably handle nested classes, limit what can be done
with a reference that refers to an object in an enclosing environment.
Environment reference qualifiers also restrict the construction of new
objects (see sec.~\ref{sec-type-sel}).

%Previous work for the Java language~\cite{javari,reim} discussed the restriction
%of mutations and the viewpoint adaptation of field selections and method selections.
%The current work discusses additional concerns arising from Scala-like languages,
%including accessor method selection (sec.~\ref{sec-accessor-method-sel}),
%environment references (which are necessary because of arbitrary nesting of classes and methods),
%and path-dependent types (and associated restrictions on construction).
Later chapters discuss extensions to the basic system (which include qualifier polymorphism),
qualifier defaults, and
practical considerations associated with closures.


\section{Type Qualifiers} \label{sec-type-qualifiers}

The current work defines reference immutability in terms of {\em type qualifiers}.
A {\em type qualifier} is associated with an underlying type,
and is normally expressed in code as an annotation of that underlying type.
Qualifiers add restrictions to what constitutes an acceptable code expression.
For any given piece of code to compile without errors,
it must be both type-correct and qualifier-correct.

The simplest form of reference immutability is a system which supports
the qualifying of every reference as either {\em readonly} or {\em mutable}.
{\em Readonly} references do not allow their referent objects to be mutated.
Listing~\ref{lst-readonly} shows how {\em readonly} restricts modification of a field {\cd f}
(underlying types are omitted).
Value types may also be qualified as {\em mutable} or {\em readonly},
but these qualifiers have no meaning for value types.

\begin{lstlisting}[caption={Mutation Restriction},float={htp},label={lst-readonly}]
var x: @readonly = ...
var y: @mutable = ...
x.f = ...   // Error
y.f = ...   // OK
\end{lstlisting}


To protect the {\em readonly} guarantee, the compiler will not accept
assignment of {\em readonly}-qualified references into {\em mutable}-qualified
fields or variables. Listing~\ref{lst-assign} illustrates the basic assignment restriction
(underlying types are omitted).
\begin{lstlisting}[caption={Assignment Restriction},float={htp},label={lst-assign}]
var x: @readonly = ...
var y: @mutable = ...
x = y   // OK
y = x   // Error
\end{lstlisting}

Since {\em readonly} is more restrictive than {\em mutable},
the relationship between {\em readonly} and {\em mutable}
may be understood as a subtype relationship.
Specifically, {\em mutable} is a subtype of {\em readonly}:
$$mutable <: readonly$$

%The following sections describe the mechanisms used to protect the {\em readonly} guarantee
%in this simple system.

%Additional qualifiers offering increased precision and flexibility are
%discussed in later sections.

% What's a type qualifier
% Qualifiers are inferred when types are inferred
% Qualifiers restrict acceptable subtype judgements; (during separate phase)


\section{Field Selection} \label{sec-field-sel}

Reading a field of an object, or {\em field selection}, requires {\em viewpoint adaptation}
to ensure the preservation of the {\em readonly} guarantee.
Specifically, a field with a type that is qualified {\em mutable} should be treated as
{\em readonly} when accessed via a {\em readonly} reference.
Listing~\ref{lst-field-sel} shows an example of viewpoint adaptation on field selection.
Field {\cd f} is assumed to be {\em mutable}.

\begin{lstlisting}[caption={Field Selection},float={htp},label={lst-field-sel}]
var x: @readonly = ...
var y: @mutable = ...
x.f.f = ...   // Error: x.f is readonly
y.f.f = ...   // OK
\end{lstlisting}

The adaptation of qualifiers on selections is defined by the {\em viewpoint adaptation operator},
which is symbolized by the right-facing triangle~$\triangleright$.
The simple form of the viewpoint adaptation operator, which is defined for
the {\em mutable} and {\em readonly} qualifiers, is shown in figure~\ref{fig-vp-simple}.

\begin{figure}[htp]
	\center
	\begin{tabular}{ccccc}
		Left & & Right & & Result \\
		\hline
		q & $\triangleright$ & {\em mutable} & = & q \\
		\_ & $\triangleright$ & {\em readonly} & = & {\em readonly} \\
	\end{tabular}
	\caption{The Viewpoint Adaptation Operator $\triangleright$, Simple Form}
	\label{fig-vp-simple}
\end{figure}

Figure~\ref{fig-vp-simple} states that the selection of a {\em mutable}
field results in whatever qualifier was on the reference used to reach the object,
and that the selection of a {\em readonly} field always results in {\em readonly}.
The symbol q represents either qualifier, and the underscore (\_) means ``don't care.''

The viewpoint adaptation operator is conservative~-- field selection
cannot produce a reference with a qualifier that is more permissive
than the field's given qualifier.

\section{Method Selection} \label{sec-method-sel}

Calling a method of an object (or {\em method selection}) involves
viewpoint adaptation of the method's result qualifier.
Method selection uses the same viewpoint adaptation operator as field selection~--
if the reference to the object (or {\em receiver reference})
is {\em readonly}, then the qualifier on the returned result is also
forced to be {\em readonly}.
Conservatively, viewpoint adaptation is required because the method may return
a field of the receiver object.

Listing~\ref{lst-method-sel} shows an example of method selection.
Method {\cd m} in class {\cd C} returns a {\em mutable} result,
but selecting {\cd m} through a {\em readonly} reference must produce
a {\em readonly} result after viewpoint adaptation.

\begin{lstlisting}[caption={Method Selection},float={htp},label={lst-method-sel}]
class C {
	def m(): @mutable = ...
}
var x: @readonly = new C
var y: @mutable = new C
x.m().f = ...   // Error: x.m() has a readonly result
y.m().f = ...   // OK
\end{lstlisting}

Successful method selection also requires the receiver-reference qualifier to
be compatible with the qualifier on the {\cd this} reference inside the method.
If the method is expecting a {\em mutable} {\cd this}, then it cannot be called
with a {\em readonly} receiver reference.
Conservatively, receiver compatibility is required because a method
with a {\em mutable} {\cd this} may modify a field of the receiver object.

Listing~\ref{lst-method-sel-this} shows an example of receiver-reference compatibility.
The {\cd this} parameter on method {\cd m} is shown explicitly for clarity
(Scala does not support explicit {\cd this} parameters).
Selecting {\cd m} through the {\em readonly} {\cd x} produces an error.
If {\cd m}'s {\cd this} parameter was {\em readonly} instead, then there would be
no receiver-reference compatibility errors.

\begin{lstlisting}[caption={Method Selection, {\cd this} Compatibility},float={htp},label={lst-method-sel-this}]
class C {
	def m(this: @mutable) = ...
}
var x: @readonly = new C
var y: @mutable = new C
x.m()   // Error: x is readonly, m expects mutable
y.m()   // OK
\end{lstlisting}


\section{Accessor Method Selection} \label{sec-accessor-method-sel}

The viewpoint adaptation of method selection also applies to {\em accessor methods}.
Scala has a built-in notion of accessor methods, which include
{\em getters} and {\em setters}. Getters and setters are meant to return and set (respectively)
elements of the object state.

A field declaration in Scala entails the automatic generation of accessor methods
for that field, which allow a derived class to declare fields that override
accessor-like methods in its base class, and vice versa. For example,
in listing~\ref{lst-accessor-methods}, class {\cd C} declares getter and setter
methods that are overridden in derived class {\cd D}.
More precisely, the declaration of field {\cd f} causes the automatic creation of
a getter method that returns the contents of the field,
and a setter method that mutates the field.

\begin{lstlisting}[caption={Accessor Methods},float={htp},label={lst-accessor-methods}]
class C {
	def f: @mutable = ...               // getter-like method: no parameter list
	def f_= (x: @mutable): Unit = ...   // setter-like method: name ends with suffix: _=
}
class D extends C {
	var f: @mutable = ...               // field overrides accessor-like methods
}
\end{lstlisting}

Accessor methods a subject to the same viewpoint adaptation rules as ordinary methods.
A call to the getter {\cd f} in listing~\ref{lst-accessor-methods} results in a {\em readonly}
result if the receiver reference is {\em readonly}, and a {\em mutable} result
if the receiver reference is {\em mutable}, removing the distinction between
a direct field read and a call to a getter-like method.
Setter methods do not return results, so viewpoint adaptation has no effect.

Getter methods should be side-effect-free by default, and
setter methods should allow mutation within their immediately-enclosing classes only.
See chapter~\ref{chap-defaults}
on default qualifier assignments for a discussion of default receiver-reference qualifiers
on accessor-like methods.

Notably, the type of the setter method's parameter correctly restricts
the references that can be assigned to the field.
In listing~\ref{lst-accessor-methods}, the parameter to the setter-like method {\cd f\_=}
has the same {\em mutable} qualifier as the field declaration {\cd f} in class {\cd D},
preventing the method from accepting a reference that could not be safely written into
the field.

\section{Viewpoint Adaption of Bounded Types} \label{sec-vp-bounded}

Scala supports bounded abstract types.
A bounded type has a lower bound and an upper bound;
a concrete instantiation of a bounded type must be between that type's lower and upper bounds.
The same rule applies to the qualifiers on the bounded type's bounds~--
the qualifier on the concrete instantiation must be
at least as restrictive as the lower bounds' qualifier, but no more restrictive
than the upper bounds' qualifier.

\begin{lstlisting}[caption={Bounded Type Instantiation},float={htp},label={lst-bounded-inst}]
class C[T] { }                     // T has maximal bounds: @mutable .. @readonly
class D[T <: @mutable] { }         // T is at most @mutable
class E[T >: @readonly] { }        // T is at least @readonly
class F extends C[@mutable] { }    // OK
class G extends C[@readonly] { }   // OK
class H extends D[@readonly] { }   // Error: @readonly is not <:@mutable
class I extends E[@mutable] { }    // Error: @mutable is not >:@readonly
\end{lstlisting}

Listing~\ref{lst-bounded-inst} shows classes {\cd C}, {\cd D}, and {\cd E} each with
an abstract type member {\cd T} (some underlying type expressions omitted).
Class {\cd C}'s abstract type member {\cd T} has undeclared bounds (which default to the
maximal bounds {\em mutable} and {\em readonly}).
The {\cd T} in class {\cd D} has an upper bound of {\em mutable},
and the {\cd T} in class {\cd E} has a lower bound of {\em reaodnly}.
Either {\em mutable} or {\em readonly} can be substituted for {\cd T} in {\cd C},
but only {\em mutable} can substitute for {\cd T} in {\cd D},
and only {\em readonly} can substitute for {\cd T} in {\cd E}.

Viewpoint adaptation of a bounded type modifies the upper bound on that type,
leaving the lower bound unchanged.
Figure~\ref{fig-vp-bounded} shows the definition of the viewpoint adaptation operator
where one or both operands are bounded types
(q, l, and u are arbitrary qualifiers, and the notation \mbox{(l .. u)} refers to a qualifier bound
of at least l and at most u).

\begin{figure}[htp]
	\center
	\begin{tabular}{ccccc}
		Left & & Right & & Result \\
		\hline
		q & $\triangleright$ & (l .. u) & = & (l .. q $\triangleright$ u) \\
		(l .. u) & $\triangleright$ & q & = & u $\triangleright$ q \\
		(\mbox{l$_1$} .. \mbox{u$_1$}) & $\triangleright$ & (\mbox{l$_2$} .. \mbox{u$_2$}) & = &
			(\mbox{l$_2$} ..
				\mbox{u$_1$} $\triangleright$ \mbox{u$_2$}) \\
	\end{tabular}
	\caption{The Viewpoint Adaptation Operator $\triangleright$, Bounded Types}
	\label{fig-vp-bounded}
\end{figure}

Viewpoint adaptation of the upper bound prevents violation of reference immutability
restrictions. For example,
where the upper bound of a type is {\em readonly}, mutations are prevented~--
regardless of the lower bound.

Leaving the lower bound unchanged prevents the qualifier bounds on an abstract type
from being narrowed due to viewpoint adaptation (although viewpoint adaptation may
widen the qualifier bounds where appropriate).
The argument for not narrowing qualifier bounds is an argument for consistency:
if it is possible to substitute a type with a particular qualifier for a given
abstract type, then all uses of that abstract type should have bounds wide enough
to include the substituting type's qualifier.
%As declared, an abstract type has specific qualifier bounds~--
%for consistency, all uses of that abstract type should have bounds that are no narrower than
%declared.
For example, if a {\em mutable}-qualified type can be substituted for an abstract type {\cd T},
then all uses of {\cd T} should have a lower bound that is no less than {\em mutable}.


% Why not adjust the lower bound?
%By adjusting the upper bound, reference immutability restrictions are preserved~--
%a bounded type with an upper bound of {\em readonly} has the same mutation restrictions
%as an ordinary type qualified by {\em readonly}.
%By adjusting the upper bound only, the viewpoint-adapted bounds remain compatible
%with the original bounds in most cases.
%The only condition under which incompatibility occurs is where the viewpoint adaptation
%causes the upper bound to become higher (more restrictive~-- viewpoint adaptation
%should never result in a less-restrictive upper bound).

\begin{lstlisting}[caption={Viewpoint Adaptation of Bounded Types},float={htp},label={lst-vp-bounded}]
class C[T] {
	var v: T
	def getV(this: @readonly): T = this.v   // `unsafe' viewpoint adaptation
}
class D extends C[@mutable] { }   // instantiates T as @mutable
val d = new D
val d_r = new D: @readonly
d.getV.f = ...     // OK: d.getV is @mutable
d_r.getV.f = ...   // Error: d_r.getV is @readonly
\end{lstlisting}

Listing~\ref{lst-vp-bounded} shows an example of the viewpoint adaptation of
a bounded type. The {\cd getV} method returns the contents of field {\cd v},
viewpoint-adapted with a {\em readonly} {\cd this}.
Both {\cd v} and method {\cd getV} have type {\cd T}, which has the bounds
({\em mutable}~.. {\em readonly}).
Although the viewpoint adaptation is technically unsafe (since the lower bound
remains {\em mutable} despite a {\em readonly} {\cd this}),
no violation of reference immutability occurs in this example.
The method {\cd getV} itself cannot mutate through {\cd v}, since
its upper qualifier bound remains {\em readonly}.
Even if a derived class ({\cd D}) instantiates {\cd T} to a type
qualified by {\em mutable}, neither {\cd v} nor the result of {\cd getV}
can be mutated when accessed through a {\em readonly} reference ({\cd d\_r}).

\begin{lstlisting}[caption={Bounded Types and the Getter Class},float={htp},label={lst-vp-bounded-2}]
class C[T] {
	var v: T
	class Getter(env: @readonly) {
		def getV: T = this.env.v     // `unsafe' viewpoint adaptation
	}
}
class D extends C[@mutable] { }  // instantiates T as @mutable
val d_r = new D: @readonly
val d_r_g = new d_r.Getter
d_r_g.getV.f = ...               // OK: d_r_g.getV returns @mutable
\end{lstlisting}

However, there is a way to obtain a {\em mutable} reference through a {\em readonly}
reference. Listing~\ref{lst-vp-bounded-2} introduces a {\em getter class},
a class that cannot mutate, but can return, elements of its enclosing class.
The getter class {\cd Getter} has a {\em readonly} {\em environment reference}
(see section~\ref{sec-env-ref}) that does not allow methods within {\cd Getter}
to mutate fields of the enclosing class through {\cd this}.
The {\cd Getter} class does have a method {\cd getV} that returns a field of
its enclosing class. Since the type of field {\cd v} is {\cd T} (that is,
with qualifier bounds ({\em mutable}~.. {\em readonly})),
the result type of {\cd getV} is the same as {\cd v} despite access through
a {\em readonly} environment reference.
When an instance of the {\cd Getter} class is constructed (fresh reference {\cd d\_r\_g}),
the call to {\cd getV} from {\cd d\_r\_g} produces a {\em mutable} result.
Although it may seem surprising at first to get a {\em mutable} result
when starting from a {\em readonly} reference ({\cd d\_r}),
this behaviour is not technically a violation of reference immutability.
What has happened in this example is that a fresh reference has been created
to refer to a {\em mutable}-qualified field.
In general, reference immutability does not prevent the simultaneous existence of
a {\em readonly} reference and a {\em mutable} reference to the same object.

Listing~\ref{lst-vp-bounded-2} is able to produce a {\em mutable} reference because
of the simultaneous coexistence of a number of conditions.
First, that class {\cd C} has an abstract type member {\cd T} with a {\em mutable}
lower bound and a {\em readonly} upper bound.
The {\em mutable} lower bound allows a derived class to replace {\cd T} with
a {\em mutable}-qualified type.
The {\em readonly} upper bound allows the type of field {\cd v}
to remain unmodified under viewpoint adaptation.
Second, that class {\cd C} contains a getter class (either an inner class
or a type refinement), and that the getter class
contains a method that returns a reference of type {\cd T}.
Third, the getter class has a {\em readonly} environment reference, which
allows it to be constructed through a {\em readonly} access path (see section~\ref{sec-type-sel}
on path-dependent restrictions on object construction).
If any one of these conditions were not met, then listing~\ref{lst-vp-bounded-2}
could not produce a {\em mutable} reference from a {\em readonly} reference.


\begin{comment}
\begin{lstlisting}[caption={An Iterator over a Collection},float={htp},label={lst-iterator}]
class List[T](env: @readonly) {
	val head: T
	val tail: List[T]
	
	class Iterator(env: @readonly) {
		var current = List.this    // List.this == Iterator.env
		def next(this: @*@mutEnv[Iterator]*@): T = {
			val t = current.head     // `unsafe' viewpoint adaptation
			current = current.tail
			t
		}
	}
}
val ls_r = new List[@readonly]
val ls_m = new List[@mutable]
val it_r = new ls_r.Iterator   // produces an Iterator over @readonly
val it_m = new ls_m.Iterator   // produces an Iterator over @mutable
... = it_r.next                // produces a @readonly element
... = it_m.next                // produces a @mutable element
\end{lstlisting}
\end{comment}

\section{Environment References} \label{sec-env-ref}

An inner class accesses the fields and methods of its enclosing class via
an environment reference.
Every class has an environment reference.
For example, listing~\ref{lst-enclosing-env} shows
an inner class {\cd D} with an explicit environment-reference parameter
that refers to an enclosing class {\cd C}.
Access to class {\cd C}'s field {\cd f} occurs via {\cd D}'s environment reference,
which is accessed through {\cd D}'s {\cd this} reference.
\begin{lstlisting}[caption={Accessing an Enclosing Environment},float={htp},label={lst-enclosing-env}]
class C {
	var f = ...
	class D(env: C) {
		... = f   // same as: ... = this.env.f
		f = ...   // OK if this.env is mutable
	}
}
\end{lstlisting}

The compiler invisibly handles the passing of the environment reference.
However, the plugin must account for the environment reference's mutability.
Reads of field~{\cd f} within {\cd D} must be viewpoint-adapted to
{\cd D}'s environment reference {\cd env}.
Similarly, writes to field~{\cd f} are allowed only if {\cd D}'s environment
reference {\cd env} is qualified {\em mutable}.

\subsection{When Local Variables Become Fields}
\label{sec-local-fields}

The Scala language allows a class to be declared within a method.
The rules for handling classes within methods are (at least superfically)
similar to the rules for handling classes within classes. For example,
in listing~\ref{lst-enclosing-env-2}, class {\cd D} is within method {\cd m}.
\begin{lstlisting}[caption={Accessing an Enclosing Environment (2)},float={htp},label={lst-enclosing-env-2}]
def m() {
	var f = ...
	class D(env: M) {  // where M is the type of m's closure
		... = f          // same as: ... = this.env.f
		f = ...          // OK if this.env is mutable
	}
}
\end{lstlisting}
When an instance of class~{\cd D} is created, it is given a reference
to a closure of its enclosing method~{\cd m}.
Local variables in {\cd m} are accessed as fields of {\cd m}'s closure
from within class {\cd D}.
Reads of variable~{\cd f} from within {\cd D} must be viewpoint-adapted to
{\cd D}'s environment reference {\cd env}.
Similarly, writes to variable~{\cd f} are allowed only if {\cd D}'s environment
reference {\cd env} is qualified {\em mutable}.

\subsection{Nested Methods}
\label{sec-nested-methods}

Methods may be nested within other methods.
Inner methods have access to their enclosing methods' local variables.
Technically, access to an enclosing method's local variables is obtained through
a reference to the enclosing method's stack frame or closure,
where the inner method treats the enclosing method's local variables as fields.

\begin{lstlisting}[caption={Accessing an Enclosing Environment (3)},float={htp},label={lst-enclosing-env-3}]
def m1() {
	var f = ...
	def m2(env: M1) {  // where M1 is the type of m1's closure
		... = f          // same as: ... = env.f
		f = ...          // OK if env is mutable
	}
}
\end{lstlisting}


Nested methods can access variables in their enclosing methods.
This access occurs through an environment reference,
which in this case may be roughly understood as a pointer to the stack
frame of the enclosing method.
Listing~\ref{lst-encl-var-access} shows a nested method~{\cd m2}
that modifies a variable {\cd v} from its enclosing method~{\cd m1}.
The access of {\cd v} goes through {\cd m2}'s environment reference~{\cd env}.

\begin{lstlisting}[caption={Enclosing-method Variable access},float={htp},label={lst-encl-var-access}]
def m1(this) {
	var v = ...
	def m2(env: @mutable) {
		v = ...      // same as: env.v = ...
	}
}
\end{lstlisting}



Nested methods can be converted to closures.

\begin{lstlisting}[caption={Nested Methods},float={htp},label={lst-nested-methods}]
def m1(this) {
	var v = ...
	def m2() {
		v = ...
	}
	m2()
}
def m1() {
	var v = ...
	class M2(env: M1) {
		def apply(m2_this: M2) {
			v = ...          // same as: m2_this.env.v = ...
		}
	}
	val m2 = new M2
	m2.apply()           // call is converted to closure application
}
\end{lstlisting}


\subsection{Increased Precision for Environment References} \label{sec-env-precision}

It is useful to be able to say that a receiver reference can
mutate fields of an inner class, but not fields of some enclosing class.
Unfortunately, a simple \mbox{{\em mutable}} qualifier is not precise enough for
this purpose~-- a merely {\em mutable} receiver allows mutation in all enclosing environments.
In this section, I introduce qualifiers that support increased precision of environment
references, followed by examples of the relationships among these qualifiers,
followed by a formal definition of the subtype relationships among these
qualifiers.

I introduce here a set of qualifiers \mbox{{\em mutable}$_T$}, where $T$ is a
type definition~-- type definitions include classes, traits (interfaces),
and method closures.
Informally, a reference with a type qualified by \mbox{{\em mutable}$_T$}
is considered {\em mutable} when accessing a field within the definition of $T$,
but is considered {\em readonly} when accessing a field lexically scoped outside
the definition of $T$.
For example, in listing~\ref{lst-enclosing-env-3},
a method {\cd m} in class {\cd D} writes to field {\cd f} in class {\cd C}.
The necessary receiver and environment references are shown explicitly
(the notation {\cd mutEnv[T]} stands for the qualifier \mbox{{\em mutable}$_T$}).
In listing~\ref{lst-enclosing-env-3}, the field {\cd f} may be reassigned from within method {\cd m}
only if {\cd m}'s {\cd this} parameter and class {\cd D}'s {\cd env}
parameter are at least \mbox{{\em mutable}$_C$}.

\begin{lstlisting}[caption={Accessing an Enclosing Environment (3)},float={htpb},label={lst-enclosing-env-3}]
class C {
	var f = ...
	class D(env: C) {
		def m(this: D) {
		            // Access to f is via this.env;
			f = ...   // OK if this and env are mutable or mutEnv[C],
			          // error if this or env is mutEnv[D] or readonly
		}
	}
}
\end{lstlisting}

A qualifier \mbox{{\em mutable}$_S$} is compatible with a qualifier \mbox{{\em mutable}$_T$}
if $S$ is defined at the same lexical scope as $T$, or
$T$ is lexically scoped within $S$.
Informally, if $S$ is at the same scope level as $T$, then $S$ and $T$ share the same
enclosing scope, which is considered {\em readonly} for both.
If $T$ is within $S$, then all scopes enclosing $S$ also enclose $T$,
so all scopes considered {\em readonly} for \mbox{{\em mutable}$_S$} are also {\em readonly}
for \mbox{{\em mutable}$_T$}.
For example, method {\cd m1} in listing~\ref{lst-param-and-env} is allowed
to call {\cd m2}, but not the reverse.
Method {\cd m1}'s {\cd this} is \mbox{{\em mutable}$_C$}, allowing mutation within either
class {\cd C} or class {\cd D}.
Method {\cd m2}'s {\cd this} is \mbox{{\em mutable}$_D$}, allowing mutation within class
{\cd D}, but not class {\cd C}.
The qualifier \mbox{{\em mutable}$_D$} here is strictly more restrictive than
the qualifier \mbox{{\em mutable}$_C$}, resulting in the
judgement \mbox{{\em mutable}$_C$} $<:$ \mbox{{\em mutable}$_D$},
but not \mbox{{\em mutable}$_D$} $<:$ \mbox{{\em mutable}$_C$}.

\begin{lstlisting}[caption={Parameters and Environment References},float={htpb},label={lst-param-and-env}]
class C {
	class D {
		def m1(this: @*@mutEnv[C]*@) = m2(this)   // OK: mutEnv[C] <: mutEnv[D]
		def m2(this: @*@mutEnv[D]*@) = m1(this)   // Error
	}
}
\end{lstlisting}

A qualifier \mbox{{\em mutable}$_S$} is compatible with a qualifier \mbox{{\em mutable}$_T$}
if $S$ is a subclass of $T$, or $T$ is a subclass of $S$.
The primary reason for the forgoing provision is to support
inheritance and overriding.
For example, in listing~\ref{lst-inherit-and-env}, method {\cd m} in class {\cd D}
overrides method {\cd m} in class {\cd C}.
The override is allowed because of the judgement \mbox{{\em mutable}$_C$} $<:$ \mbox{{\em mutable}$_D$},
and the call to the superclass method is allowed because of
the judgement \mbox{{\em mutable}$_D$} $<:$ \mbox{{\em mutable}$_C$}.
Informally, allowing the method override is acceptable here because \mbox{{\em mutable}$_C$}
within class {\cd C} and \mbox{{\em mutable}$_D$} within class {\cd D} are both
expressions of the same invariant: mutations are allowed inside the enclosing
class, but not outside.
Although listing~\ref{lst-inherit-and-env} shows the defintions of class {\cd C}
and class {\cd D} in the same scope (in which case \mbox{{\em mutable}$_C$} and \mbox{{\em mutable}$_D$}
are mutually compatible without the inheritance provision), the inheritance provision
allows {\cd D} to inherit from {\cd C} when they are defined in different scopes.

\begin{lstlisting}[caption={Inheritance and Environment References},float={htpb},label={lst-inherit-and-env}]
class C {
	def m(this: @*@mutEnv[C]*@) = ...
}
class D extends C {
	override def m(this: @*@mutEnv[D]*@) = {   // OK: mutEnv[C] <: mutEnv[D]
		C.super.m(this)                      // OK: mutEnv[D] <: mutEnv[C]
	}
}
\end{lstlisting}

One pattern supported by the inheritance provision
is mutation within an inherited environment.
For example, listing~\ref{lst-inherit-and-env-2} shows a method {\cd m}
inside the environment class~{\cd Env2} that mutates a field
inside environment class~{\cd Env1}.
The mutation is allowed because {\cd Env2} extends {\cd Env1}.

\begin{lstlisting}[caption={Inheritance and Environment References (2)},float={htp},label={lst-inherit-and-env-2}]
class Env1 {
	var v = ...  // v is defined in Env1
}
class Env2 extends Env1 {
	class C {
		def m(this: @*@mutEnv[Env2]*@) = {
			v = ...  // OK: mutEnv[Env2] allows mutation of Env1
		}
	}
}
\end{lstlisting}


In general, the subtype relationships in figure~\ref{envqual-subtype} hold.
\begin{figure}[htpb]
\begin{equation*} %\tag{SUBTYPE-ENVQUAL} \label{equ-SUBTYPE-ENVQUAL}
\begin{array}{l c l l}
	(1)\quad mutable_S &<:& mutable_T & \text{where}\ owners(T) \subseteq owners(S) \\
	(2)\quad mutable_S &<:& mutable_T & \text{where}\ S \in parents(T)\ \text{or}\ T \in parents(S) \\
	(3)\quad mutable   &<:& mutable_T & \text{for all}\ T \\
	(4)\quad mutable_T &<:& readonly  & \text{for all}\ T \\
\end{array}
\end{equation*}
\caption{Environment-reference Subtype Relationships}
\label{envqual-subtype}
\end{figure}

Rule 1 in figure~\ref{envqual-subtype} is the lexical enclosure rule.
The qualifier \mbox{{\em mutable}$_S$} is a subtype of the qualifier \mbox{{\em mutable}$_T$}
if all definitions enclosing the definition of $T$ also enclose the definition of $S$.
The function \mbox{$owners$} takes a type definition $T$ and returns the set of all
type definitions enclosing~$T$.

Rule 2 is the inheritance provision rule.
The qualifiers \mbox{{\em mutable}$_S$} and \mbox{{\em mutable}$_T$} are mutually compatible
where $S$ and $T$ have an inheritance relationship.
The function $parents$ takes a type definition $T$ and returns the set of
all type definitions that are base types of~$T$.

Rule 3 states that \mbox{{\em mutable}} (which allows mutation in any scope)
is always a subtype of \mbox{{\em mutable}$_T$} (which allows mutation only
within the definition of $T$ or within $T$'s base/derived types).

Rule 4 states that \mbox{{\em mutable}$_T$} is always a subtype of \mbox{{\em readonly}}.
The \mbox{{\em mutable}$_T$} qualifier allows mutation within some scopes, but
\mbox{{\em readonly}} does not allow mutation within any scope.

The determination of a subtyping relationship may require multiple applications
of rules~1 and~2. For example, listing~\ref{lst-multirule-qual-sub}
shows a mutation of a field in a base class declared within an inherited environment.
\begin{lstlisting}[caption={Multi-rule Qualifier Subtyping},float={htp},label={lst-multirule-qual-sub}]
class Env1 {
	class C {
		var v = ..
	}
}
class Env2 extends Env1 {
	class D extends C {
		def m(this: @*@mutEnv[Env2]*@) = {
			v = ...  // OK: mutEnv[C] <: mutEnv[D] and mutEnv[D] <: mutEnv[Env2]
		}
	}
}
\end{lstlisting}

Field {\cd v} in class {\cd C} is written within a method of derived class {\cd D}.
The method takes a receiver parameter qualified by \mbox{{\em mutable}$_{Env2}$}.
No single rule allows the deduction:
	$$mutable_C <: mutable_{Env2}$$
since {\cd C} is not within {\cd Env2}, nor does it have an inheritance
relationship with {\cd Env2}.
However, {\cd C} does have an inheritance relationship with class {\cd D},
which is within {\cd Env2}.

There is more than one way to show that $mutable_C <: mutable_{Env2}$ in
listing~\ref{lst-multirule-qual-sub}.
For example, it is possible to deduce that $mutable_C <: mutable_{Env1}$
due to lexical enclosure, and that $mutable_{Env1} <: mutable_{Env2}$
due to inheritance.

\subsection{Viewpoint Adaptation with Increased-Precision Environment Qualifiers} \label{sec-vp-env-qual}

The simple viewpoint adaptation operator (see fig.~\ref{fig-vp-simple})
must be augmented to include the increased-precision qualifiers from section~\ref{sec-env-precision}.
Figure~\ref{fig-vp-inc-prec} shows the augmented viewpoint adaptation operator.

\begin{figure}[htp]
	\center
	\begin{tabular}{ccccc}
		Left & & Right & & Result \\
		\hline
		q & $\triangleright$ & {\em mutable} & = & q \\
		q & $\triangleright$ & \mbox{{\em mutable}$_T$} & = & $\text{q} \lor mutable_T$ \\
		\_ & $\triangleright$ & {\em readonly} & = & {\em readonly} \\
	\end{tabular}
	\caption{The Viewpoint Adaptation Operator $\triangleright$, with Environment Qualifiers}
	\label{fig-vp-inc-prec}
\end{figure}

Compared with figure~\ref{fig-vp-simple},
figure~\ref{fig-vp-inc-prec} adds a rule to support qualifiers of the form \mbox{{\em mutable}$_T$}
on the right-hand side. (No changes are necessary in the other rules to support
\mbox{{\em mutable}$_T$} on the left-hand side.)

The new rule performs viewpoint adaptation by joining ($\lor$) the left-hand-side qualifier
with \mbox{{\em mutable}$_T$}. The join operator on qualifiers is defined here as
the least upper bound.

%The join operator ($\lor$) is defined as the least restrictive
%qualifier that is no less restrictive than either of its operands.
%That is, with the definition of the qualifier subtype relationships in figure~\ref{envqual-subtype},
%$$q_1 \lor q_2 = \text{the least restrictive}\ q$$
%such that
%\begin{equation*}
%	q_1 <: q\ \text{and}\ q_2 <: q.
%\end{equation*}

\subsection{Type Selection and Object Construction} \label{sec-type-sel}

Preservation of reference immutability guarantees requires limits on object
construction. An object should not be constructable if it contains
an environment reference that could be used to break reference immutability
guarantees. Listing~\ref{lst-object-construction} shows a class {\cd C} with
a {\em mutable} environment reference and a class {\cd D} with a {\em readonly}
environment reference (environment references are shown explicitly).
Class {\cd C} is able to construct an object of class {\cd D},
but class {\cd D} cannot construct an object of class {\cd C}
due to an incompatible environment reference qualifier.

\begin{lstlisting}[caption={Environment References and Object Construction},float={htp},label={lst-object-construction}]
class Env {
	class C(env: @mutable) {
		new D(env)    // OK
	}
	class D(env: @readonly) {
		new C(env)    // Error: class C could mutate within Env
	}
}
\end{lstlisting}

Scala supports path-dependent type selection.
An object of a given class may or may not be constructible depending on
the path used to select that class.
For example, listing~\ref{lst-typesel-construct} shows two attempts to
construct an instance of an inner class {\cd D}.
To be constructible, class {\cd D} requires a {\em mutable} reference to its immediately-enclosing
environment, which is an instance of class {\cd C}.
If an attempt is made to reach class {\cd D} through a {\em readonly} reference,
then the resulting type may not be safe to construct because the {\em readonly}
reference is not compatible with the {\em mutable} environment reference
expected by {\cd D}.

\begin{lstlisting}[caption={Type Selection and Object Construction},float={htp},label={lst-typesel-construct}]
class C(env: @mutable) {
	class D(env: @mutable) {
	}
}
var cr: @readonly = new C
var cm: @mutable = new C
new cr.D   // Error: cr is readonly, but D.env is mutable
new cm.D   // OK
\end{lstlisting}

A path-dependent type selection is considered well-formed only if the
viewpoint-adapted reference used to select the type has a qualifier compatible
with that type's environment reference. The formalization of path-dependent type
selection in chapter~\ref{chap-dot-calculus-emend} contains more information.

\subsection{Receiver References and Inheritance} \label{sec-rcv-inherit}

Derived classes may override methods from base classes.
The parameters of the overriding method must be at least as restrictive
as the corresponding parameters in the overridden method.
The receiver reference {\cd this}, although not explicitly shown in Scala
code, is also considered a parameter~-- the qualifier on the overriding
method's {\cd this} must be at least as restrictive as the overridden method.

\begin{lstlisting}[caption={Receiver Reference Contravariance},float={htp},label={lst-rcv-contravar}]
class C {
	def m1(this: @readonly)
	def m2(this: @mutable)
}
class D extends C {
	override def m1(this: @mutable)    // Error
	override def m2(this: @readonly)   // OK
}
\end{lstlisting}

Listing~\ref{lst-rcv-contravar} shows a class {\cd C} that overrides some methods
from class {\cd D}. The override of method {\cd m2} is accepted, but the override
of {\cd m1} is not accepted due to an incompatible receiver reference qualifier.
The receiver reference for {\cd C}'s version of {\cd m1} is {\em readonly}, which means that it
can accept an argument with any qualifier, but {\cd D}'s version of {\cd m1}
can accept an argument with a {\em mutable} qualifier only, causing an incompatibility.

\subsection{Environment Reference Qualifers are Covariant} \label{sec-env-covariance}

Environment reference qualifers are covariant with class hierarchies.
That is, the environment reference on a derived class may be less restrictive than
the environment reference on the base class, but
it may not be more restrictive.
For example, listing~\ref{lst-env-covariance} shows classes {\cd D} and {\cd E},
which both attempt to extend class {\cd C}.
The defintion of class~{\cd D} is acceptable because it has a less restrictive environment reference than {\cd C},
but class~{\cd E} is not acceptable because it has a less permissive environment reference
than {\cd C}.

\begin{lstlisting}[caption={Covariance of Environment References},float={htp},label={lst-env-covariance}]
class Env {
	class C(env: @*@mutEnv[Env]*@) {}
	class D(env: @mutable) extends C {}   // OK: D <: C, mutable <: mutEnv[Env]
	class E(env: @readonly) extends C {}  // Error
}
\end{lstlisting}

Although covariance of environment references may appear counterintuitive at first,
covariance actually provides the correct restrictions on object creation.
An environment reference that is less restrictive with respect to mutation permissions
is more restrictive with respect to construction~-- a class with a {\em readonly}
environment reference does not have any path-dependent construction requirements,
but a class with a {\em mutable} environment reference can be constructed only when
accessed through a {\em mutable} reference.

\begin{comment}
\begin{lstlisting}[caption={Covariance of Environment References},float={htp},label={lst-env-covariance}]
class Env {
	class Function1[-S,+T](env:@*@mutEnv[Env]@*) {
		def apply(this:@*@mutEnv[Env]*@, s: S): T
	}
	class Function1D[-S,+T](env:@readonly) extends Function1 {  // Error: @readonly fails covariance test
		def apply(this:@mutEnv[Env], s: S): T   // Error: @mutEnv[Env] fails consistency test
	}
	class Function1E[-S,+T](env:@mutable) extends Function1 {
		def apply(this:@mutable, s: S): T       // Error: override fails contravariant-parameter test
		def apply(this:@*@mutEnv[Env]*@, s: S): T
		def apply(this:@readonly, s: S): T
	}
}
\end{lstlisting}
\end{comment}


\subsection{Consistency Requirements for Environment References} \label{sec-env-consistency}

All methods of a class must have receiver references that are at least as restrictive
as that class' environment reference.
Otherwise, a method could violate the qualifier on the path used to select the
class prior to construction.
For example, listing~\ref{lst-env-consistency-1} shows a set of methods declared within
a class~{\cd C}. Class~{\cd C}'s environment reference is {\em readonly},
so it is acceptable to create an object of type~{\cd C} where {\cd C} is
reached through a {\em readonly} reference.
Preservation of the {\em readonly} guarantee means that no method of {\cd C}
can be allowed to mutate outside of {\cd C} through the {\cd this} reference.
(Or, at least that no such mutating method can be {\em called} on an object of type {\cd C}
where {\cd C} is reached through a {\em readonly} reference~-- see section~\ref{sec-vp-new}
on the viewpoint adaptation of the {\cd new} operator for a discussion of this possibility.)

\begin{lstlisting}[caption={Environment Reference Consistency (1)},float={htp},label={lst-env-consistency-1}]
class Env {
	class C(env: @readonly) {
		def m1(this: @readonly)     // OK
		def m2(this: @*@mutEnv[C]*@)    // OK
		def m3(this: @*@mutEnv[Env]*@)  // Error: may mutate outside of C
		def m4(this: @mutable)      // Error: may mutate outside of C
	}
}
\end{lstlisting}

Note that method {\cd m2} is accepted despite having a receiver reference
qualified by \mbox{$mutable_C$}. The meaning of \mbox{$mutable_C$} is
such that all environment references at or outside of {\cd C} are {\em readonly}.
Therefore, the {\em readonly} on the environment reference of {\cd C}
can be replaced by \mbox{$mutable_C$} without any loss of safety or precision.

\begin{lstlisting}[caption={Environment Reference Consistency (2)},float={htp},label={lst-env-consistency-2}]
class Env {
	class C(env: @readonly) {
		class D1(env: @readonly) {}    // OK
		class D2(env: @*@mutEnv[C]*@) {}   // OK
		class D3(env: @*@mutEnv[Env]*@) {} // Error: may mutate outside of C
		class D4(env: @mutable) {}     // Error: may mutate outside of C
	}
}
\end{lstlisting}

Listing~\ref{lst-env-consistency-2} is the same as listing~\ref{lst-env-consistency-1},
except that the methods of class~{\cd C} are replaced by nested classes.
The same restrictions that apply to the receiver references on the methods
also apply to the environment references on the nested classes.

Environment-reference consistency requirements only need to be enforced
for concrete classes. Abstract classes are not constructible


\section{Nested Methods} \label{sec-nested-methods}

\begin{lstlisting}[caption={},float={htp},label={lst-}]
def m(this: @readonly) {
	def g() { ... }
	g()
}
\end{lstlisting}

\begin{lstlisting}[caption={},float={htp},label={lst-}]
class M(env: @readonly) {
	def apply() {
		class G(env: @mutEnv[M]) {
			def apply() { ... }
		}
		val g = new G
		g.apply()
	}
}
\end{lstlisting}

\section{Function Types} \label{sec-function-types}

\begin{lstlisting}[caption={},float={htp},label={lst-}]
def m() {
	def g() { ... }
	g()
}
\end{lstlisting}

\begin{lstlisting}[caption={},float={htp},label={lst-}]
class C {
	def m(this: @mutEnv[C]) {
		class G(env: @mutEnv[C]) extends Function0[Unit] {
			def apply() { ... }
		}
		val g = new G
		g.apply()
	}
}
\end{lstlisting}

\begin{lstlisting}[caption={},float={htp},label={lst-}]
class Function0[U](env: @readonly) {
	def apply(this: @readonly)
}
def m1() {
	class G(env: @readonly) extends Function0[Unit] {
		def apply(this: @readonly) { ... }
	}
	val g: @fresh = new G(env)
	val gr: @readonly = g   // Error: gr.apply cannot be called, so don't allow the assignment
	g.apply()   // OK: fresh <: mutEnv[G]
	m2(g)       // OK: fresh <: readonly
	
	// if I can call m2(g), and m2 can call g.apply,
	//  then qual(g) <: rcv(g.apply).
	
	//gr.apply()  // Error: expecting @mutEnv[G], got @readonly
	//m2(gr)      // special case ---> : cannot call gr.apply, so forbid here
	// The rule may be: do not allow passing of a function that cannot be invoked.
	// Therefore, any callee may assume that any function passed can be invoked.
}
def m2(this: @readonly, f: Function0[Unit] @mutable) {
	f.apply()   // OK
}
def m2(this: @readonly, f: Function0[Unit] @readonly) {
	f.g()      // Error: @readonly not compat with @mutEnv[Function0]
	f.apply()  // OK: special exception for @readonly
}
\end{lstlisting}

A loss of generalization occurs when the apply method is special-cased...
How do we know which methods the called method will call?

Perhaps something like a @calls[C] annotation,
where we assume the called method can call any method on class C,
where C is a non-strict supertype of the underlying reference type.



\chapter{Extensions of Reference Immutability for Scala} \label{chap-flexibility-scala}
	There are a number of ways to improve the flexibility of the
basic reference immutability system.

\section{Locality-dependent Result Qualifers}

It is useful to be able to say that a method's result-type qualifier
depends only on the receiver's qualifier,
only on the qualifier of a parameter of that method,
or on neither the receiver's qualifier nor a parameter's qualifier.


For example, if a method returns one of its receiver's fields,
and the method is called with a {\em mutable} receiver reference,
then the result should also be considered {\em mutable} even though
the method itself treats the receiver as {\em readonly}.
Listing~\ref{lst-loc} shows

\begin{lstlisting}[caption={Locality-dependent Results},label={lst-loc},float=htp]
class C {
	val f = ...
	def m1(this: @readonly, c: @readonly): @readonly = {
		this.f
	}
	def m2(this: @readonly, c: @readonly): @readonly = {
		c.f
	}
}
val c = new C
val cr: @readonly = new C
c.m1(c)   // result is @readonly
c.m1(cr)  // result is @readonly
cr.m1(c)  // result is @readonly
c.m2(c)   // result is @readonly
c.m2(cr)  // result is @readonly
cr.m2(c)  // result is @readonly
\end{lstlisting}





It is useful to be able to differentiate the following cases:



It is useful to be able to say that the result of a method was reached through
a path containing one of that method's parameters.




It is also useful to be able to say that the result of a method
was not reached through any method parameter.

Listing~\ref{lst-loc} shows three versions of method~{\cd m}, each with a
different result locality.


\begin{lstlisting}[caption={Locality-dependent Results (2)},label={lst-loc-2},float=htp]
class C {
	def m1(this: @rothis, c: @readonly): @rothis = {
		this.f
	}
	def m2(this: @readonly, c: @roparam): @roparam = {
		c.f
	}
}
val c = new C
val cr: @readonly = new C
c.m1(c)   // result is @mutable
c.m1(cr)  // result is @mutable
cr.m1(c)  // result is @readonly
c.m2(c)   // result is @mutable
c.m2(cr)  // result is @readonly
cr.m2(c)  // result is @mutable
\end{lstlisting}

\begin{lstlisting}[caption={Locality-dependent Results},label={lst-loc},float=htp]
class C {
	def m1(this: @readonly, p: @readonly): @readonly = {
		this.f
	}
	def m2(this: @readonly, p: @readonly): @readonly = {
		p.f
	}
	def m3(this: @mutable, p: @mutable): @mutable = {
		new D()
	}
}
val c = new C
val cr: @readonly = new C
val p: @mutable = ...
val pr: @readonly = ...
c.m1(p)
\end{lstlisting}



\begin{lstlisting}[caption={Polymorphism with {\em rothis}},label={lst-rothis},float=htp]
class C {
  private var f = ...
  def getF(this: @rothis) = { this.f }
}
val mc: C @mutable
val mf = mc.getF     // returns a mutable reference
val rc: C @readonly
val rf = rc.getF     // returns a readonly reference
\end{lstlisting}





\section{}

\begin{lstlisting}
class C {
	def m1(this: @mutable)
}
def m2(c: C @readonly) {
	c.m1()     // Error: Cannot call m1 due to incompatible receiver
}
val c: @mutable = new C
m2(c)        // OK: mutable <: readonly
\end{lstlisting}

\begin{lstlisting}
class C {
	def m1(this: @mutable)
}
def m2(c: C @readonly @*@calls[C]*@) {
	c.m1()     // OK: allowed by @calls[C]
}
val c: @mutable = new C
val cr: @readonly = new C
m2(c)        // OK: calls to c.m1 are allowed here
m2(cr)       // Error: calls to cr.m1 are not allowed here
\end{lstlisting}

What restrictions exist on copying references with @calls qualifiers?
1. calls[D] <: calls[C] where D <: C.
		calls[D] is less restrictive than calls[C],
		because the set of methods in D is a superset of the set of methods in C.
2. The default calls qualifer is @calls[Any] (or $calls_\top$).
		@calls[Any] entails no call-site restrictions, but neither does it allow
		any in-callee exceptions.
3. Therefore,
	var c: @calls[C]
	var d: @calls[Any]
	d = c   // OK
	c = d   // Error
4. The key to this system is that a calls[Any] reference r may be
		passed to a method expecting a calls[C] reference,
		provided r is of type C and qual(r) <: rcv(m), where
		m is any method of class C.

\section{Qualifier Polymorphism on Receivers}
\label{sec-poly-qual}

I introduce a polymorphic qualifier {\em rothis},\footnote{
	The reason for the name {\em rothis} is that {\em rothis}-qualified
	references are those which are expected to be reached through an access path
	rooted at a {\em readonly this.}
}
which is similar in purpose and function to
Javari's {\em romaybe}~\cite{javari} and ReIm's {\em polyread}~\cite{reim}.
The purpose of {\em rothis} is to improve the flexibility of getter methods.
Like {\em mutable}-returning methods, {\em rothis}-returning methods are adapated to
the receiver qualifier at call sites~--
an {\em rothis}-returning method called with a {\em mutable}
receiver will yield a {\em mutable} result, and the same method called
with a {\em readonly} receiver will yield a {\em readonly} result.

% methods with an rothis this will not mutate the locality of this
However, an {\em rothis}-qualified reference, like a {\em readonly}-qualified reference,
does not allow mutation of its referent object.
Also like {\em readonly}, the non-mutation guarantee offered by {\em rothis}
is transitive.
Therefore, a method where the receiver parameter {\cd this} is qualified
by {\em rothis} can be safely called with a {\em readonly} receiver reference~--
a method with a {\em mutable}-qualified {\cd this} cannot.

There is no particular requirement that an {\em rothis}-returning method
must have an {\em rothis} {\cd this}, nor that a method with an {\em rothis} {\cd this}
must have an {\em rothis} result.
However, it is likely to be the case that the type inference mechanism
will automatically infer an {\em rothis} result for methods that have an {\em rothis} {\cd this}.

% avoidance of code duplication (const)
Code duplication is prevented in cases where a method has both an {\em rothis} {\cd this}
and an {\em rothis} result. Without {\em rothis}, many getter methods would need at least two versions:
one that returns {\em mutable} results, and another that accepts {\em readonly} receivers.


% lattice: between mutable and readonly
The {\em rothis} qualifier is positioned below {\em readonly} on the qualifier lattice.
The positioning below {\em readonly} guarantees that an {\em rothis}-returning method
will never return a reference that was reached through a path that contains
a {\em readonly} element, unless that path element happens to be {\cd this}.

For example, listing~\ref{lst-rothis} shows a getter method {\cd getF} that
returns a private field {\cd f}. Calling {\cd getF} with a {\em mutable} receiver
returns a {\em mutable} reference, and calling {\cd getF} with a {\em readonly}
receiver returns a {\em readonly} reference.
The receiver parameter {\cd this} is shown
explicitly, qualified by {\em rothis}. (Scala does not support explicit {\cd this} parameters~--
the example here does not compile as shown. I will discuss syntactially-valid
alternative annotation schemes in a later chapter.)
I assume here that the field {\cd f} is {\em mutable}, and that the result of
method {\cd getF} is inferred.

\begin{lstlisting}[caption={Polymorphism with {\em rothis}},label={lst-rothis},float=htp]
class C {
  private var f = ...
  def getF(this: @rothis) = { this.f }
}
val mc: C @mutable
val mf = mc.getF     // returns a mutable reference
val rc: C @readonly
val rf = rc.getF     // returns a readonly reference
\end{lstlisting}


% other approaches: polyread/romaybe

% other approaches: generalization over parameters (prev) - handling of multiple params
Initially, I tried a multi-parameter interpretation of {\em rothis}
where any number of method parameters (in addition to {\cd this}) participated
in the viewpoint adaptation operation.
In the multi-parameter interpretation, the result of an {\em rothis}-returning
method is the least upper bound of the qualifiers on all arguments passed by the caller
to {\em rothis}-qualified method parameters.
However, I have yet to find a practical or idiomatic Scala example that
would benefit from a multi-parameter interpretation.

\begin{comment}
% The reason built-in polymorphism doesn't work is that what we usually
%  want is to relate the result qualifier to the qualifiers of one or more parameters,
%  even though the underlying types of the parameters and result could be incompatible.
Scala has built-in support for type polymorphism.
Listing~\ref{lst-polymorph-via-types} shows 
I conjecture that, for any practical cases where qualifier polymorphism beyond {\em rothis}
is required, Scala's standard facilities for type polymorphism will be adequate.
\begin{lstlisting}[caption={Qualifier Polymorphism Via Type Polymorphism},label={lst-polymorph-via-types},float=htp]
def m(t: T): R = ...                                      // monomorphic
def m[U <: T @readonly, V <: R @readonly](t: U): V = ...  // polymorphic 
\end{lstlisting}
\end{comment}

When read through an environment reference, a local variable must be observed to
have a type with a field-admissible qualifier.
Specifically, variables with {\em fresh} or {\em rothis} qualifiers are observed
to have {\em mutable} or {\em readonly} qualifiers (respectively) when read from within nested classes.
The qualifier adjustment occurs before viewpoint adaptation.

%The reason for not allowing a nested-class variable read to yield an {\em rothis}
%result is that {\em rothis} is a polymorphic qualifier that does not have the
%same meaning when observed from inside a nested class.


%The reason for not allowing a nested-class variable read to yield a {\em fresh}
%result is that an object of the type of the nested class could survive past
%the return of the enclosing method.
%After the enclosing method returns, the {\em fresh}-qualified reference is

When a variable is written from within a nested class, its qualifier is not adjusted.
If variable type qualifiers were adjusted on writes, then {\em mutable} references could
get written to {\em fresh} variables, or {\em readonly} references could get written to
{\em rothis} variables.

\section{Qualifier Polymorphism on Formal Parameters}
\label{sec-poly-qual-param}

TBD.

\section{Returning Fresh References} \label{sec-fresh}

TBD.

\section{Viewpoint Adaptation on New Object Construction} \label{sec-vp-new}

TBD.
% Basic idea: have "new" return a qualifier equal to the path-dependent qualifier used to select the type

\section{The Package Hierarchy} \label{sec-packages}

Environment reference qualifiers can be extended to express the ability to mutate
within a particular package.
Listing~\ref{lst-package-env} shows a class {\cd C} that updates a static counter
during construction. Class {\cd C}'s environment reference allows mutation
within the inner package~{\cd B}, but not the outer package {\cd A}.\footnote{
	Packages in production code are usually arranged in a file system hierarchy
	rather than a scoped hierarchy as shown in listing~\ref{lst-package-env}.
	Listing~\ref{lst-package-env} shows a scoped hierarchy for clarity of presentation.
}
The environment qualifier~\mbox{{\em mutable}$_B$} allows mutation within
object~{\cd D}, which is within package {\cd B}.

%Like nested classes, packages are arranged in a hierarchy.
%Unlike ordinary classes, package contents are always instantiated in the static environment.

\begin{lstlisting}[caption={Packages and Environment References},float={htp},label={lst-package-env}]
package A {
	package B {
		class C(env: @*@mutEnv[B]*@) {
			D.initCount += 1           // update a static counter
		}
		object D {
			var initCount = 0
		}
	}
}
\end{lstlisting}

In Scala, hierarchies of packages are intentially similar to hierarchies of objects.
Listing~\ref{lst-object-env} shows the replacement of the {\cd package}
keyword with the {\cd object} keyword, with no change to code meaning.

\begin{lstlisting}[caption={Objects and Environment References},float={htp},label={lst-object-env}]
object A {
	object B {
		class C(env: @*@mutEnv[B]*@) {
			D.initCount += 1           // update a static counter
		}
		object D {
			var initCount = 0
		}
	}
}
\end{lstlisting}

At first glance, supporting mutation permissions at the package level
may seem less than useful in a purity system~-- packages are globally-visible
entities, which means that side effects within packages are potentially
observable from anywhere.
However, limitation of observational exposure is a separate problem.
It may be possible to show in future work that some methods can
observe side effects only within certain packages, making them independent from
methods that mutate only within different packages.
In any case, knowing that a method can mutate only within certain packages
could aid the programmer in understanding what side effects a method can produce.

\section{Side Effects in the External World} \label{sec-external-world}

TBD.



\chapter{A Modification of the DOT Calculus} \label{chap-dot-calculus-emend}
	% subtyping relationship: S <: T only if qual(S) <: qual(T)
% entailments

% DOT calculus syntax: doesn't model inhertance, failure to prove preservation - but preservation is only one way to prove type safety - see uDOT

% type assignment rules

%  field assignment as a setter method call -- setter methods have a mutEnv[T] receiver where T is the imm. encl. class

% declaration subsumption rules

The Dependent Object Types (DOT) calculus~\cite{dot}
is a formalization of several of Scala's key language features.
These features include abstract type members, type refinements,
and path-dependent types.
I extend the DOT calculus to show precisely how type qualifiers
interact with key language features.

The newer \mbox{$\mu$DOT} calculus~\cite{udot}
features a smaller syntax designed specifically to model
path-dependent types.
Unlike DOT, \mbox{$\mu$DOT} has a complete type-safety proof.
However, I chose to express type qualifier interactions
as extensions to DOT rather than \mbox{$\mu$DOT} because the small-step operational
semantics of DOT seemed to afford a clearer presentation
of qualifier interactions than the big-step semantics of \mbox{$\mu$DOT}.

In any case, the lack of a preservation proof (and corresponding failure
to build a classical type-safety proof) for DOT is not necessarily a hinderance to
extending DOT to describe type qualifier interactions.
The qualifier-extended language should be type-safe to the extent that
the underlying language is type-safe,
provided the set of accepted programs in the qualifier-extended language
is a subset of the programs accepted in the original language.
%extensions do not modify the semantics of the underlying language.
%The type qualifier extension should only restrict the set of
%accepted programs, not modify the underlying language semantics.
If a complete type-safety proof for DOT is discovered in the future,
then the qualifier-extended version of DOT should also be type-safe.

The basic extensions to DOT are the following:

\begin{enumerate}

\item{\bf Any type can be annotated with a qualifier.}
The extended DOT syntax includes the type expression \mbox{$T@Q$},
where $T$ is a type and $Q$ is a qualifier.
Furthermore, qualifier judgments of the form \mbox{$qual(T)=Q..R$}
are described in section~\ref{sec-qualifier-judgments}.
These qualifier judgments associate every type $T$ (annotated or not)
with lower and upper qualifier bounds $Q$ and~$R$.

%\item {\bf Every type is associated with exactly one type qualifier.}
%For every type $T$ in the original DOT formulation,
%I substitute $T_q$, where $q$ is the reference immutability qualifier
%on~$T$.

%For any type $T \in \tau$, the function $$qual: \tau \rightarrow Q$$
%yields the type qualifier $q \in Q$ associated with type $T \in \tau$.
%The related function $$withQual: \tau \rightarrow Q \rightarrow \tau$$
%yields a type identical to its first argument, except that the resulting type is
%associated with a qualifier identical to the second argument, so that
%the following relation holds:
%$$qual(withQual(T,q)) = q$$

\item {\bf Certain judgments are modified to take qualifiers into account.}
Section~\ref{sec-qualifier-judgments} introduces {\em qualified subtype}
judgments of the form~\mbox{$S <:_q T$}.
The conditions required to form the judgment \mbox{$S <:_q T$} are
stronger than the conditions required for an ordinary subtype judgment~\mbox{$S <: T$}.
A selection of type assignment judgments and declaration-related judgments
are modified to use \mbox{$<:_q$} rather than \mbox{$<:$} in their conditions,
which has the effect of eliminating judgments that are incorrect with respect to qualifiers.

%\item {\bf Type qualifiers eliminate some subtype judgments.}

%For qualified types $T_q$ and $U_r$, the subtype judgment $$\Gamma \vdash T_q <: U_r$$
%may exist only if $$q <: r$$
%I do not explicitly amend DOT to include elimination of
%subtype judgments. However, every valid relationship \mbox{$T_q <: U_r$}
%should be understood to imply \mbox{$q <: r$.}

%Due to covariance of environment references, \mbox{$T_q <: U_r$}
%%also implies \mbox{$env(T_q) <: env(U_r)$}.
%(See section~\ref{dot-aux-functions}
%on the definition of $env$ and section~\ref{sec-env-covariance} on environment-reference
%covariance.)

\item {\bf The notion of a {\em concrete type} is replaced with the more general
notion of a {\em constructible type}.}
Constructibility judgments of the form~\mbox{$T\ {\bf constr}$} indicate
that the type $T$ satisfies all requirements necessary for construction.
In the unmodified DOT, all concrete types are constructible;
in the modified DOT, a concrete type may not be constructible if
an environment-reference qualifier restriction is violated.
Section~\ref{sec-constructibility-judgments} discusses constructibility judgments.

%\item {\bf Every concrete type is associated with an environment-reference qualifier.}
%Path-dependent type selections where the path's type qualifier
%does not conform to the selected type's environment reference
%should not be constructible.
%Section~\ref{dot-env-def} describes the handling of non-conforming type
%selections, and rule~\ref{equ-new} in section~\ref{dot-type-assign}
%is modified to prevent construction of objects with non-conforming type selections.

\end{enumerate}


Type qualifiers also restrict other judgments.
The sections below describe the DOT language, and qualifier-related restrictions
on that language, in detail.
Section~\ref{sec-dot-syntax} briefly describes the DOT syntax, amended to
account for qualifiers.
Section~\ref{dot-union-intersect} briefly discusses qualifier-related concerns
with type intersections and unions.
Section~\ref{dot-aux-functions} describes the auxiliary functions used to handle
qualifiers in the modified DOT rules.
Section~\ref{dot-env-def} describes the environment-qualifer function \mbox{$env$}.
Following section~\ref{dot-env-def} are presentations of rule-specific
modifications of DOT.
What will be shown in section~\ref{dot-qual-typesafe} is that the modifications
of DOT do not introduce type unsoundness into DOT.
A few considerations not modeled in the formalism, but nonetheless must be
considered in the implementation, are discussed in section~\ref{dot-unaddressed}.


\section{DOT Syntax}
\label{sec-dot-syntax}

The DOT calculus syntax is reproduced here (figure~\ref{fig-dot-syntax}),
but with the addition of type qualifiers ($q,r,u,w$ in the box at the bottom of
the right-hand side).
I highlight a few of the salient features of DOT here.
For more detailed information on these and other aspects of DOT, see the DOT paper~\cite{dot}.

\begin{figure}[htbp]
	\begin{equation*}
	\begin{array}{ll|ll}
		x,y,z & \text{Variable} &
			L::= & \text{Type label} \\
		l & \text{Value label} &
			\qquad L_c & \quad \text{class label} \\
		m & \text{Method label} &
			\qquad L_a & \quad \text{abstract type label} \\
		v::= & \text{Value} &
			S,T,U,V,W::= & \text{Type} \\
		\qquad x & \qquad \text{variable} &
			\qquad p.L & \quad \text{type selection} \\
		t::= & \text{Term} &
			\qquad T\ \{ z \Rightarrow \overline{D} \} & \quad \text{type refinement} \\
		\qquad v & \qquad \text{value} &
			\qquad T \land T & \quad \text{intersection type} \\
		\qquad {\bf val}\ x = {\bf new}\ c; t & \qquad \text{new instance} &
			\qquad T \lor T & \quad \text{union type} \\
		\qquad t.l & \qquad \text{field selection} & \qquad \top & \quad \text{top type} \\
		\qquad t.m(t) & \qquad \text{method invocation} &
			\qquad \bot & \quad \text{bottom type} \\
		p::= & \text{Path} &
			\qquad \boxed{T@B} & \quad \text{qualified type} \\
		\qquad x & \qquad \text{variable} &
			\cancel{S_c, T_c ::=} & \cancel{\text{Concrete type}} \\
		\qquad p.l & \qquad \text{selection} &
			\multicolumn{2}{l}{
			\qquad \cancel{p.L_c\ |
				\ T_c\ \{ z \Rightarrow \overline{D} \}\ |
				\ T_c \land T_c\ |
				\ \top}
				%\ |
				%\ \boxed{T_c@Q}
			} \\
		c::=T\ \{\overline{d}\} & \text{Constructor} &
			D::= & \text{Declaration} \\
		d::= & \text{Initialization} &
			\qquad L:S..U & \quad \text{type declaration} \\
		\qquad l=v & \qquad \text{field initialization} &
			\qquad l:T & \quad \text{value declaration} \\
		\qquad m(x)=t & \qquad \text{method initialization} &
			\qquad m:S \rightarrow U & \quad \text{method declaration} \\
		s::=\overline{x \mapsto c} & \text{Store} &
			\Gamma ::= \overline{x:T} & \text{Environment} \\
	\multicolumn{2}{l|}{
				\fbox{
					\begin{minipage}{0.47\linewidth}
						$B,A::= Q..R \qquad \quad \text{Qualifier Bounds}$
					\end{minipage}
				}
				} &
			\multicolumn{2}{l}{
				\fbox{
					\begin{minipage}{0.43\linewidth}
						$Q,R::= \qquad \quad \ \ \text{Type qualifier}$
						
						$\ \ \ \ mutable\ |\ mutable_{L_c}\ |\ readonly\ |$
						
						$\ \ \ \ rothis\ |\ fresh$
					\end{minipage}
				}
				} \\
	\end{array}
	\end{equation*}
	\caption{DOT Syntax with Type Qualifiers}
	\label{fig-dot-syntax}
\end{figure}

\begin{comment}
\begin{figure}[htbp]
	\begin{equation*}
	\begin{array}{ll|ll}
		x,y,z & \text{Variable} & L::= & \text{Type label} \\
		l & \text{Value label} & \qquad L^c & \quad \text{class label} \\
		m & \text{Method label} & \qquad L^a & \quad \text{abstract type label} \\
		v::= & \text{Value} & S_q,T_q,U_q,V_q,W_q::= & \text{Type} \\
		\qquad x & \qquad \text{variable} & \qquad p.L & \quad \text{type selection} \\
		t::= & \text{Term} & \qquad T_q\ \{ z \Rightarrow \overline{D} \} & \quad \text{type refinement} \\
		\qquad v & \qquad \text{value} & \qquad T_q \land T_q & \quad \text{intersection type} \\
		\qquad {\bf val}\ x = {\bf new}\ c; t & \qquad \text{new instance} & \qquad T_q \lor T_q & \quad \text{union type} \\
		\qquad t.l & \qquad \text{field selection} & \qquad \top & \quad \text{top type} \\
		\qquad t.m(t) & \qquad \text{method invocation} & \qquad \bot & \quad \text{bottom type} \\
		p::= & \text{Path} & S_q^c, T_q^c ::= & \text{Concrete type} \\
		\qquad x & \qquad \text{variable} &
			\multicolumn{2}{l}{
			\qquad p.L^c\ |\ T_q^c\ \{ z \Rightarrow \overline{D} \}\ |\ T_q^c \land T_q^c\ |\ \top} \\
		\qquad p.l & \qquad \text{selection} & D::= & \text{Declaration} \\
		c::=T_q^c\ \{\overline{d}\} & \text{Constructor} & \qquad L:S_q..U_q & \quad \text{type declaration} \\
		d::= & \text{Initialization} & \qquad l:T_q & \quad \text{value declaration} \\
		\qquad l=v & \qquad \text{field initialization} & \qquad m:S_q \rightarrow U_q & \quad \text{method declaration} \\
		\qquad m(x)=t & \qquad \text{method initialization} &
			\Gamma ::= \overline{x:T_q} & \text{Environment} \\
		s::=\overline{x \mapsto c} & \text{Store} &
			\multicolumn{2}{l}{\boxed{q,r,u,w \qquad \qquad \qquad \text{Type Qualifier}}}\\
	%\multicolumn{4}{c}{\boxed{q,r,s,u \qquad \text{Type Qualifier}}} \\
	\end{array}
	\end{equation*}
	\caption{DOT Syntax with Type Qualifiers}
	\label{fig-dot-syntax}
\end{figure}
\end{comment}

Every type in DOT contains a number of declarations. A declaration $D$ can be
a type declaration, a value (field) declaration, or a method declaration.
A set of declarations is denoted by~$\overline{D}$.

When an object is constructed, every field must be initialized
to a particular value, and every method must correspond to a particular program term.
Constructors have the form \mbox{$T_q^c\ \{ \overline{d} \}$},
where \mbox{$T_q^c$} is a concrete type, and \mbox{$\overline{d}$} is
a set of initializers.

Only concrete types can be constructed.
A concrete type can be the empty type~$\top$ (containing no declarations),
the intersection of two concrete types (the intersection carries all declarations present
in either type),
a refinement of a concrete type (the refinement adds declarations),
or path-dependent class selection~(\mbox{$p.L^c$}).

A path-dependent class selection~\mbox{$p.L^c$} selects a class (with
the unique label~\mbox{$L^c$}), as reached through a particular path~$p$.
Path-dependent class selection is important for supporting nested classes.
Since nested classes are able to access members of their enclosing classes,
an instance of a nested class requires a reference to the correct instance of
its enclosing class.
The path $p$ yields the necessary environment reference;
section~\ref{sec-env-ref} discusses environment references in detail,
and rule~\ref{equ-new} in section~\ref{dot-type-assign} shows how the reference immutability
qualifier on the type of $p$ is used to restrict new object construction.

%type bounds
DOT also supports abstract type members.
There is no additional concession required to support qualifier-related
restrictions on type substitution (section~\ref{sec-vp-bounded});
the normal subtype-related restrictions enforced during type
refinement should also correctly enforce the qualifier bounds
on abstract types.

%not inheritance
DOT does not attempt to model inheritance.
Although DOT allows the structural contents of a type to be derived
from another type via type refinement,
DOT does not include any notion of a nominal class hierarchy.
Although inheritance is not modeled formally, it is dealt with in the language implementation.

The following sections describe modifications to the DOT rules.
Only those rules with specific modifications are discussed~--
for a discussion of other rules, see the DOT paper~\cite{dot}.



\section{Qualifier Judgements} \label{sec-qualifier-judgments}

I extend DOT to allow the environment $\Gamma$ to include judgments of the form:
$$\Gamma \vdash qual(T) = Q..R$$
where $qual$ is a mapping from types to qualifier bounds.
If $T$ is a concrete type, then normally \mbox{$Q=R$}.
If $T$ is an abstract type, then the lower qualifier bound~$Q$ may
differ from the upper qualifier bound~$R$.

The $qual$ mapping for a well-formed qualified type \mbox{$T@Q..R$} unconditionally produces
\mbox{$Q..R$},
as shown in rule~\ref{equ-t-q}.
The addition of a qualifier to $T$ in a sense ``overrides'' any previous
qualifier on $T$~-- that is, \mbox{$qual(T)$} and \mbox{$qual(T@Q..R)$} are not related
in any way.
%If a qualified type \mbox{$T@Q$} is well-formed, then \mbox{$qual(T@Q)$}
%is equivalent to $Q$. 

\begin{equation*}\tag{T-Q}\label{equ-t-q}
\begin{array}{c}
\Gamma \vdash T@Q..R\ {\bf wf} \\
\midrule
\Gamma \vdash qual(T@Q..R)=Q..R \\
\end{array}
\end{equation*}

\vspace{0.4cm}

Additional qualifier judgments are needed to associate every type with
specific qualifier bounds. These additional judgments follow.

The top type $\top$ is associated with the top qualifier \mbox{$readonly$} (rule~\ref{equ-top-q}),
and the bottom type $\bot$ is associated with the bottom qualifer \mbox{$fresh$} (rule~\ref{equ-bot-q}).

\begin{equation*}\tag{$\top$-Q}\label{equ-top-q}
\begin{array}{c}
%\Gamma \vdash \top\ {\bf wf} \\
%\midrule
\Gamma \vdash qual(\top)=readonly..readonly \\
\end{array}
\end{equation*}

\vspace{0.4cm}

\begin{equation*}\tag{$\bot$-Q}\label{equ-bot-q}
\begin{array}{c}
%\Gamma \vdash \top\ {\bf wf} \\
%\midrule
\Gamma \vdash qual(\bot)=fresh..fresh \\
\end{array}
\end{equation*}

\vspace{0.4cm}

Type intersection involves qualifier intersection (rule~\ref{equ-and-q}).
Similarly, type union involves qualifier union (rule~\ref{equ-or-q}).

\begin{equation*}\tag{$\land$-Q}\label{equ-and-q}
\begin{array}{c}
\Gamma \vdash qual(T)=Q..R\ ,\ qual(T')=Q'..R' \\
\midrule
\Gamma \vdash qual(T \land T')=Q \land Q' .. R \land R' \\
\end{array}
\end{equation*}

\vspace{0.4cm}

\begin{equation*}\tag{$\lor$-Q}\label{equ-or-q}
\begin{array}{c}
\Gamma \vdash qual(T)=Q..R\ ,\ qual(T')=Q'..R' \\
\midrule
\Gamma \vdash qual(T \lor T')=Q \lor Q' .. R \lor R' \\
\end{array}
\end{equation*}

\vspace{0.4cm}

A refined type is associated with the same qualifiers as the underlying type (rule~\ref{equ-rf-q}).

\begin{equation*}\tag{RF-Q}\label{equ-rf-q}
\begin{array}{c}
\Gamma \vdash T\ \{ z \Rightarrow \overline{D} \}\ {\bf wf}\ ,\ qual(T)=Q..R \\
\midrule
\Gamma \vdash qual(T\ \{ z \Rightarrow \overline{D} \})=Q..R \\
\end{array}
\end{equation*}

\vspace{0.4cm}

Class types are $mutable$ by default (rule~\ref{equ-lc-q}).
Abstract types are associated with qualifier bounds.
Rule~\ref{equ-la-q} associates the path-dependent abstract type \mbox{$p.L_a$}
with a lower qualifier bound derived from the lower type bound $S$
and an upper qualifier bound derived from the upper type bound $U$.

\begin{equation*}\tag{LC-Q}\label{equ-lc-q}
\begin{array}{c}
\Gamma \vdash qual(p.L_c)=mutable..mutable \\
\end{array}
\end{equation*}

\vspace{0.4cm}

\begin{equation*}\tag{LA-Q}\label{equ-la-q}
\begin{array}{c}
\Gamma \vdash p \ni L_a : S..U\ ,\ qual(S)=Q..R\ ,\ qual(U)=Q'..R' \\
\midrule
\Gamma \vdash qual(p.L_a)=Q..R' \\
\end{array}
\end{equation*}

\vspace{0.4cm}

The inclusion of qualified types requires rules for performing subtype
judgments involving qualified types.
The required subtyping rules are \ref{equ-subtype-qual} and \ref{equ-qual-subtype}.

\begin{equation*}\tag{$<:$-QUAL}\label{equ-subtype-qual}
\begin{array}{c}
\Gamma \vdash S <: T \\
\midrule
\Gamma \vdash S <: T@Q..R \\
\end{array}
\end{equation*}

\vspace{0.4cm}

\begin{equation*}\tag{QUAL-$<:$}\label{equ-qual-subtype}
\begin{array}{c}
\Gamma \vdash S <: T \\
\midrule
\Gamma \vdash S@Q..R <: T \\
\end{array}
\end{equation*}

\vspace{0.4cm}

Notice that the added subtyping rules do not take qualifier annotations into account;
ordinary subtype judgments (which ignore qualifiers) are distinct from
subtype judgments that take qualifiers into account (see rule~\ref{equ-subq}).

Rule~\ref{equ-subq} defines judgments of the form \mbox{$S <:_q T$}.
%forms subtype judgments that take qualifiers into account.
In general, $S$ is a {\em qualified subtype} of $T$ if \mbox{$S <: T$} and
the qualifier bounds on $S$ conform to the qualifier bounds on $T$.

\begin{equation*}\tag{$<:_q$}\label{equ-subq}
\begin{array}{c}
\Gamma \vdash S <: T\ ,\ R<:Q' \\
\Gamma \vdash qual(S)=Q..R\ ,\ qual(T)=Q'..R' \\
\midrule
\Gamma \vdash S <:_q T \\
\end{array}
\end{equation*}

\vspace{0.4cm}

Since rule~\ref{equ-subq} fails to form subtype judgments for
equivalent abstract types where the upper and lower qualifier bounds
differ, another rule~\ref{equ-subq-equiv} is introduced.
Rule~\ref{equ-subq-equiv} forms the judgment \mbox{$S <:_q T$}
where $S$ and $T$ are equivalent types, and the qualifier bounds on $S$
are not wider than the qualifier bounds on $T$.
%where $S$ and $T$ are equivalent types with equivalent qualifier bounds.
(In general, \mbox{$S \equiv T$} implies \mbox{$S <: T$} and \mbox{$T <: S$},
and similarly for qualifiers.)
%and \mbox{$Q \equiv Q'$} implies \mbox{$Q <: Q'$} and \mbox{$Q' <: Q$}.)

\begin{equation*}\tag{$<:_q$-EQUIV}\label{equ-subq-equiv}
\begin{array}{c}
\Gamma \vdash S \equiv T\ ,\ Q \equiv Q'\ ,\ R <: R' \\
\Gamma \vdash qual(S)=Q..R\ ,\ qual(T)=Q'..R' \\
\midrule
\Gamma \vdash S <:_q T \\
\end{array}
\end{equation*}

\vspace{0.4cm}

%Any type may be annotated with a qualifier. Rule~\ref{equ-t-q-wf}
%allows any well-formed type to remain well-formed when qualified.
Generally, any qualified type is considered well-formed if its
underlying type is well-formed.
Rule~\ref{equ-t-q-wf} is the well-formedness rule for qualified types.

\begin{equation*}\tag{T-Q-WF}\label{equ-t-q-wf}
\begin{array}{c}
\Gamma \vdash T\ {\bf wf} \\
\midrule
\Gamma \vdash T@Q..R\ {\bf wf} \\
\end{array}
\end{equation*}

\vspace{0.4cm}


\section{Constructibility Judgements} \label{sec-constructibility-judgments}

I extend DOT to include judgments of the form:
$$\Gamma \vdash T\ {\bf constr}$$
where $T$ is a type.
The unmodified DOT included a syntactic distinction between concrete
types and abstract types~-- concrete types were constructible, and
abstract types were not.
In the modified DOT, I elect instead to introduce constructibility
judgements rather than syntactic restrictions,
since environment-reference qualifier restrictions can prevent
the construction of some concrete types.

Rule~\ref{equ-lc-constr} defines constructibility of class selections.
A type consisting of a class selection is considered constructible
if the qualifier on the type of the selection path $p$ is compatible
with the selected class's environment-reference qualifier.
I introduce an auxiliary function $env$ that maps from
class labels to environment-reference qualifiers.
%That is, for a given class label $L_c$,
%\mbox{$env(L_c)$} produces the class's environment-reference qualifier~$Q$.

\begin{equation*}\tag{LC-CONSTR}\label{equ-lc-constr}
\begin{array}{c}
\Gamma \vdash p:T\ ,\ p.L_c\ {\bf wf} \\
\Gamma \vdash qual(T)=Q..R\ ,\ R<:env(L_c) \\
\midrule
\Gamma \vdash p.L_c\ {\bf constr} \\
\end{array}
\end{equation*}

\vspace{0.4cm}

The following rules produce constructibility judgements for other types.
Rule~\mbox{\ref{equ-ref-constr}} states that if a type is constructible,
then a refinement of that type is also constructible.
Rule~\mbox{\ref{equ-and-constr}} states that the intersection of constructible types
is itself constructible.
Finally, rule~\mbox{\ref{equ-top-constr}} states that the top type~$\top$
is always constructible.

\begin{equation*}\tag{REF-CONSTR}\label{equ-ref-constr}
\begin{array}{c}
\Gamma \vdash T\ {\bf constr} \\
\midrule
\Gamma \vdash T\ \{ z \Rightarrow \overline{D} \}\ {\bf constr} \\
\end{array}
\end{equation*}

\vspace{0.4cm}

\begin{equation*}\tag{$\land$-CONSTR}\label{equ-and-constr}
\begin{array}{c}
\Gamma \vdash T\ {\bf constr}\ ,\ T'\ {\bf constr} \\
\midrule
\Gamma \vdash T \land T'\ {\bf constr} \\
\end{array}
\end{equation*}

\vspace{0.4cm}

\begin{equation*}\tag{$\top$-CONSTR}\label{equ-top-constr}
\begin{array}{c}
\Gamma \vdash \top\ {\bf constr} \\
\end{array}
\end{equation*}

\vspace{0.4cm}


\section{Viewpoint Adaptation Operator} \label{sec-dot-vp-adapt}

The viewpoint adapation operator is required for some of
the modified type assignment judgements.
Equation~\ref{op-viewpoint-qual} shows the viewpoint adaptation operator
over qualifiers.

\begin{equation*}\tag{$\triangleright$-QUAL}\label{op-viewpoint-qual}
\begin{array}{llll}
Q\ \triangleright\ R &=& R & \text{where}\ R = fresh \\
   & & Q & \text{where}\ R = rothis \\
	 & & Q \lor R & \text{otherwise} \\
\end{array}
\end{equation*}

Since types in the modified DOT are associated with qualifier bounds
rather than individual qualifiers, I also define the viewpoint adaptation
operator over qualifier bounds (\mbox{\ref{op-viewpoint-qualbound}}).

\begin{equation*}\tag{$\triangleright$-QUALBOUND}\label{op-viewpoint-qualbound}
\begin{array}{llll}
(Q..R) \triangleright (Q'..R')
   &=& Q..R & \text{where}\ R'=rothis \\
   & & R'..R' & \text{where}\ R'=fresh \\
   & & (R \lor R') .. (R \lor R') & \text{where}\ Q'=R'\ \text{otherwise} \\
   & & Q' .. (R \lor R') & \text{where}\ Q' \neq R'\ \text{otherwise} \\
\end{array}
\end{equation*}

\section{Qualifier Substitution and Field-legal Qualifiers} \label{sec-aux-dot}



\section{Modifications of Type Assignment Rules} \label{sec-type-assign}

Three type assignment rules are modified from their original forms.
These rules are reproduced here, with modifications in boxes.

Rule~\ref{equ-new} is modified to construct constructible types only,
rather than any concrete type.
Furthermore, the condition \mbox{$S<:U$}, which requires that
the lower bound of every type member conforms to its upper bound,
is modified to \mbox{$S <:_q U$} to require that the lower qualifier bound similarly conforms
to the upper qualifier bound.

\begin{equation*}\tag{NEW}\label{equ-new}
\begin{array}{c}
y \notin fn(T') \\
\Gamma \vdash \boxed{T\ {\bf constr}}\ ,\ T \prec_y \overline{L : S .. U},\overline{D} \\
\Gamma, y : T \vdash \overline{S\ \boxed{<:_q}\ U}\ ,\ \overline{d} : \overline{D}\ ,\ t' : T' \\
%\boxed{ \Gamma \vdash p:V\ ,\ p.L_c \ {\bf wfe} }\ ,\ p.L_c \prec_y \overline{L : S .. U},\overline{D} \\
%\Gamma, y : p.L_c \vdash \overline{S <: U}\ ,\ \overline{d} : \overline{D}\ ,\ t' : T' \\
%\boxed{\Gamma \vdash this : V\ ,\ canSubst(qual(V), env(T_c))} \\
%\boxed{env(owningMethodOrCtor(y)) <: env(T_c)} \\
%\boxed { qual(V) <: env(L_c) } \\
\midrule
\Gamma \vdash {\bf val}\ y =\ {\bf new}\ T\ \{\overline{d}\}\ ;\ t' : T' \\
%\Gamma \vdash {\bf val}\ y =\ {\bf new}\ p.L_c\ \{\overline{d}\}\ ;\ t' : T' \\
\end{array}
\end{equation*}

\vspace{0.4cm}

Rule~\ref{equ-sel}, the field selection rule,
defines the type of the selection \mbox{$t.l$}, where $t$ is the left-hand-side term
and $l$ is a field label. The expression \mbox{$t \ni l : T'$}
means that term $t$ contains the declaration \mbox{$l : T'$} as one of its members.
Originally, rule~\ref{equ-sel} assigned the type $T'$ to the selection \mbox{$t.l$},
but the modified rule annotates the result with the viewpoint-adaptated qualifier bounds
\mbox{$Q'..R \triangleright fieldAdj(R')$}.
The $fieldAdj$ function is a mapping from qualifiers to field-legal qualifiers,
and is defined as follows:
\begin{equation*}\tag{FIELDADJ}\label{equ-fieldadj}
\begin{array}{lcll}
	fieldAdj(Q) &=& readonly & \text{if}\ Q = rothis \\
	            &=& mutable & \text{if}\ Q = fresh \\
							&=& Q & \text{otherwise} \\
\end{array}
\end{equation*}

%The qualifier expression \mbox{$fieldAdj(R')$} ensures that the viewpoint adaptation
%is performed with a field-legal qualifier.
The $fieldAdj$ in this rule is not technically necessary if it is understood that
all local variables are ``converted'' to fields when seen from within nested classes
(see section~\ref{sec-local-fields}), in which case the application of $fieldAdj$
in rule~\ref{equ-vdecl-wf} renders the $fieldAdj$ in this rule redundant.

\begin{equation*}\tag{SEL}\label{equ-sel}
\begin{array}{c}
\Gamma \vdash \boxed{t : T,}\ t \ni l : T' \\
\boxed{\Gamma \vdash qual(T)=Q..R\ , \ qual(T')=Q'..R'} \\
\midrule
\Gamma \vdash t.l : \boxed{T'@Q'..R \triangleright fieldAdj(R')} \\
\end{array}
\end{equation*}

\vspace{0.4cm}

Rule \ref{equ-msel}, the method selection (or invocation) rule, defines the type of
the expression $t.m(t')$, where $t$ is the left-hand-side term,
$m$ is a method label, and $t'$ is the term that produces the argument
passed to $m$. For simplicity, the DOT calculus only defines methods that
have a single parameter.
The first modification of rule \ref{equ-msel} is the inclusion of the
condition $canSubst(q, rcv(m))$, which prevents the formation of a type
judgment where the qualifier of the left-hand-side term $t$ is incompatible with
method $m$'s {\cd this} qualifier.

\begin{equation*}\tag{MSEL}\label{equ-msel}
\begin{array}{c}
\Gamma \vdash \boxed{t: U,}\ t \ni m: S \rightarrow T \\
\Gamma \vdash t': T'\ ,\ T'\ \boxed{<:_q}\ S \\
\boxed{\Gamma \vdash qual(U)=Q..R\ ,\ qual(T)=Q'..R'\ ,\ canSubst(q, rcv(m))} \\
\midrule
\Gamma \vdash t.m(t'): \boxed{T@Q'..R \triangleright R'  _{q\ \triangleright\ u}} \\
\end{array}
\end{equation*}

\vspace{0.4cm}

Originally, rule \ref{equ-msel} defined the type of \mbox{$t.m(t')$}
to be simply~$T$, the result type of $m$.
The second modification of the rule performs viewpoint adaptation
of the returned qualifier, i.e., \mbox{$q\ \triangleright\ u$}.






\begin{comment}
\section{Auxiliary Functions} \label{dot-aux-functions}

I define a selection of auxiliary functions and operations to assist with
the modification of the DOT calculus.
These functions include $rcv$, $env$, $canSubst$, and $fieldAdj$.
%In addition to the functions $qual$ and $withQual$ (defined above),
%I also define the functions $rcv$, $env$, $canSubst$, and $fieldAdj$,
%and a viewpoint adaptation operator over types.

The $rcv$ function takes a method label $m$, and returns the qualifier
on $m$'s receiver ({\cd this}) reference.
Scala's object-oriented formulation makes it easier to treat the receiver parameter
separately from other formal parameters (see the use of $rcv$ in rule \ref{equ-msel} below).

The $env$ function maps class labels and concrete types to their
associated environment references. The $env$ function is defined (section~\ref{dot-env-def})
such that non-conforming type selections cannot be used to construct new objects.

The $canSubst$ function (equation~\ref{equ-cansubst}) takes two qualifier arguments.
It returns true if the first qualifier argument can substitute for the second qualifier argument
in a method selection (see the method section rule~\ref{equ-msel}).
In general, if the second qualifier is polymorphic (e.g. $rothis$),
then the substitution is always possible.
Otherwise, the substitution is possible only if the first qualifier is a subtype
of the second qualifier.
\begin{equation*}\tag{CANSUBST}\label{equ-cansubst}
\begin{array}{lcll}
	canSubst(q,r) &=& \text{true} & \text{if}\ r = rothis \\
							&=& q <: r & \text{otherwise} \\
\end{array}
\end{equation*}

\vspace{0.4cm}

The $fieldAdj$ function (equation~\ref{equ-fieldadj}) takes a qualifier $q$ as an argument.
If $q$ is permissible as a field type qualifier, then $q$ is returned as-is.
If $q$ is not permitted as a field type qualifier, then a more restrictive
qualifier is returned.

\begin{equation*}\tag{FIELDADJ}\label{equ-fieldadj}
\begin{array}{lcll}
	fieldAdj(q) &=& readonly & \text{if}\ q = rothis \\
	            &=& mutable & \text{if}\ q = fresh \\
							&=& q & \text{otherwise} \\
\end{array}
\end{equation*}

%\begin{equation*}\tag{FIELDADJ}\label{equ-fieldadj}
%\begin{array}{lcll}
%	fieldAdj(T_q) &=& T_{readonly} & \text{if}\ q = rothis \\
%	            &=& T_{mutable} & \text{if}\ q = fresh \\
%							&=& T_q & \text{otherwise} \\
%\end{array}
%\end{equation*}
\vspace{0.4cm}

%For convenience, I extend the viewpoint adaptation operator to operate over types
%(equation~\ref{equ-vp-typ}).
%The viewpoint adaptation operator over types returns a type identical to its
%right-hand-side type $U$, except with a qualifier that is viewpoint-adapted
%with the left-hand-side qualifier.
%\begin{equation*}\tag{VP-TYP}\label{equ-vp-typ}
%\begin{array}{ccc}
% %	T \triangleright U &=& withQual(U, qual(T) \triangleright qual(U)) \\
%	T_q \triangleright U_r &=& U_{q \triangleright r} \\
%\end{array}
%\end{equation*}
\end{comment}



\section{Type Assignment Rules} \label{dot-type-assign}

Of the four type assignment rules in DOT, two rules~-- \ref{equ-sel} and \ref{equ-msel}~--
are modified here. Rule \ref{equ-sel}, the field selection rule,
defines the type of the selection \mbox{$t.l$}, where $t$ is the left-hand-side term
and $l$ is a field label. The expression \mbox{$t \ni l : T_r'$}
means that term $t$ contains the declaration \mbox{$l : T_r'$} as one of its members.
Originally, rule \ref{equ-sel} assigned the type $T'$ to the selection \mbox{$t.l$},
but the modified rule qualifies the result with the viewpoint adaptation
\mbox{$q\ \triangleright\ fieldAdj(r)$}.
The qualifier expression \mbox{$fieldAdj(r)$} ensures that the viewpoint adaptation
is performed with a field-legal qualifier.
The $fieldAdj$ in this rule is not technically necessary if it is understood that
all local variables are ``converted'' to fields when seen from within nested classes
(see section~\ref{sec-local-fields}), in which case the application of $fieldAdj$
in rule~\ref{equ-vdecl-wf} renders the $fieldAdj$ in this rule redundant.

\begin{equation*}\tag{SEL}\label{equ-sel}
\begin{array}{c}
\Gamma \vdash \boxed{t : T_q,}\ t \ni l : T_r' \\
\midrule
\Gamma \vdash t.l : \boxed{T_{q\ \triangleright\ fieldAdj(r)}'} \\
\end{array}
\end{equation*}

\vspace{0.4cm}

Rule \ref{equ-msel}, the method selection (or invocation) rule, defines the type of
the expression $t.m(t')$, where $t$ is the left-hand-side term,
$m$ is a method label, and $t'$ is the term that produces the argument
passed to $m$. For simplicity, the DOT calculus only defines methods that
have a single parameter.
The first modification of rule \ref{equ-msel} is the inclusion of the
condition $canSubst(q, rcv(m))$, which prevents the formation of a type
judgment where the qualifier of the left-hand-side term $t$ is incompatible with
method $m$'s {\cd this} qualifier.

\begin{equation*}\tag{MSEL}\label{equ-msel}
\begin{array}{c}
\Gamma \vdash \boxed{t: U_q,}\ t \ni m: S_r \rightarrow T_u \\
\Gamma \vdash t': T_w'\ ,\ T_w' <: S_r \\
\boxed{canSubst(q, rcv(m))} \\
\midrule
\Gamma \vdash t.m(t'): \boxed{T_{q\ \triangleright\ u}} \\
\end{array}
\end{equation*}

\vspace{0.4cm}

Originally, rule \ref{equ-msel} defined the type of \mbox{$t.m(t')$}
to be simply~$T$, the result type of $m$.
The second modification of the rule performs viewpoint adaptation
of the returned qualifier, i.e., \mbox{$q\ \triangleright\ u$}.

The object creation rule~\ref{equ-new} is modified to reject object
creation expressions where the type that would have been constructed
has the bottom environment-reference qualifier \mbox{$\bot_e$}.
As explained in section~\ref{dot-env-def}, the environment qualifier \mbox{$\bot_e$}
is present on any concrete type formed using a non-conforming path-dependent
type selection.
The returned reference $y$ has the type $q$, which is normally {\em fresh}.

\begin{equation*}\tag{NEW}\label{equ-new}
\begin{array}{c}
y \notin fn(T')\ \boxed{,\ env(T_q^c) \neq \bot_e} \\
\Gamma \vdash T_q^c \ {\bf wfe}\ ,\ T_q^c \prec_y \overline{L : S_u .. U_w},\overline{D} \\
\Gamma, y : T_q^c \vdash \overline{S_u <: U_w}\ ,\ \overline{d} : \overline{D}\ ,\ t' : T_r' \\
%\boxed{ \Gamma \vdash p:V\ ,\ p.L_c \ {\bf wfe} }\ ,\ p.L_c \prec_y \overline{L : S .. U},\overline{D} \\
%\Gamma, y : p.L_c \vdash \overline{S <: U}\ ,\ \overline{d} : \overline{D}\ ,\ t' : T' \\
%\boxed{\Gamma \vdash this : V\ ,\ canSubst(qual(V), env(T_c))} \\
%\boxed{env(owningMethodOrCtor(y)) <: env(T_c)} \\
%\boxed { qual(V) <: env(L_c) } \\
\midrule
\Gamma \vdash {\bf val}\ y =\ {\bf new}\ T_q^c\ \{\overline{d}\}\ ;\ t' : T_r' \\
%\Gamma \vdash {\bf val}\ y =\ {\bf new}\ p.L_c\ \{\overline{d}\}\ ;\ t' : T' \\
\end{array}
\end{equation*}

\vspace{0.4cm}

For an explanation of the other conditions on new object construction,
see the DOT paper~\cite{dot}.

\section{Method Declaration Subsumption} \label{dot-meth-decl-sub}

The method declaration subsumption rule~\ref{equ-mdecl-sub}
defines the conditions under which one method declaration is considered
a subtype of another method declaration.
For this rule, I add the condition \mbox{$rcv(m')<:rcv(m)$},
which allows the judgment \mbox{$m<:m'$} only if
the receiver qualifier of $m'$ is a supertype of the
receiver qualifier of~$m$.
As with other parameter qualifiers, receiver qualifier relationships are contravariant
with method subsumption relationships.

\begin{equation*}\tag{MDECL-$<:$}\label{equ-mdecl-sub}
\begin{array}{c}
\Gamma \vdash S_u' <: S_q\ ,\ T_r <: T_w' \\
\boxed{rcv(m') <: rcv(m)} \\
\midrule
\Gamma \vdash (m : S_q \rightarrow T_r) <: (\boxed{m'} : S_u' \rightarrow T_w') \\
\end{array}
\end{equation*}

\vspace{0.4cm}

DOT also defines rules for subsumption of type declarations and value (field) declarations,
but these subsumption judgments do not require any additional restrictions.



\begin{comment}
\section{Well-formedness of Type Selections} \label{dot-wf-type-sel}

Type selections in DOT are path-dependent.
DOT's well-formedness rules for type selection, \ref{equ-wf-type-sel} and \ref{equ-wf-type-sel-2},
are modified to restrict selections of concrete types (classes) where
the qualifier on the path used for the selection is not compatible with
the class's environment reference qualifier.
The condition \mbox{$q<:env(L)$} enforces the necessary restriction,
where $q$ is the path qualifier and $L$ is the selected type.
Abstract types cannot be constructed, so the \mbox{$isAbstract$} predicate is introduced
to avoid unnecessarily restricting abstract type selections.
%~-- a path-dependent type selection is prohibited only where $L$ refers to
%a concrete class, and the path qualifier $q$ is not compatible with $L$'s environment qualifier.

\begin{equation*}\tag{TSEL-WF$_1$}\label{equ-wf-type-sel}
\begin{array}{c}
\Gamma \vdash \boxed{p:V_q\ ,}\  p \ni L: S_r..U_w\ ,\ S_r\ {\bf wf}\ ,\ U_w\ {\bf wf} \\
\boxed{ isAbstract(L) \lor q <: env(L) } \\
\midrule
\Gamma \vdash p.L\ {\bf wf} \\
\end{array}
\end{equation*}

\vspace{0.4cm}

\begin{equation*}\tag{TSEL-WF$_2$}\label{equ-wf-type-sel-2}
\begin{array}{c}
\Gamma \vdash \boxed{p:V_q\ ,}\  p \ni L: \bot..U_w \\
\boxed{ isAbstract(L) \lor q <: env(L) } \\
\midrule
\Gamma \vdash p.L\ {\bf wf} \\
\end{array}
\end{equation*}

\vspace{0.4cm}

The added restrictions in rules~\ref{equ-wf-type-sel} and~\ref{equ-wf-type-sel-2}
are perhaps a bit more restrictive than necessary.
Preservation of reference immutability guarantees only requires restriction of
type selections that are ultimately used in new object constructions.
However, the extra modifications required to formally allow more permissive
type-selection rules in DOT may not be worth the added complexity.

% Environment-reference consistency requirements for inner classes?

\section{Well-formedness of Method Declarations} 
% Environment-reference consistency requirement?

\end{comment}

\section{Well-formedness of Field Declarations} 

A condition is added to rule~\ref{equ-vdecl-wf} to prevent
the declaration of fields with type qualifiers that are not allowed on fields.
The field can be declared only if the qualifier $q$ on type \mbox{$T_q$}
is unmodified by the \mbox{$fieldAdj$} function.

\begin{equation*}\tag{VDECL-WF}\label{equ-vdecl-wf}
\begin{array}{c}
	\Gamma \vdash T_q\ {\bf wf} \\
	\boxed{q = fieldAdj(q)} \\
	\midrule
	\Gamma \vdash l: T_q\ {\bf wf} \\
\end{array}
\end{equation*}

\vspace{0.4cm}



\section{Type Qualifiers Do Not Introduce Type Unsoundness} \label{dot-qual-typesafe}

The addition of type-qualifier-related modifications to DOT cannot compromise
the type safety of DOT.
That is: if all programs accepted by DOT are type-safe, then all programs
accepted by the modified DOT are type-safe.
The arguments for the foregoing statement are discussed below.

\subsubsection{Associating types with qualifiers does not change the meaning
of any DOT program.}
In the absence of any other modifications, adding a qualifier~$q$ to every type
$T$ does not change the meaning of any program.
Although the types $T_q$ and $T_r$ become distinguishable if their qualifiers
$q$ and $r$ are distinguishable, in the absence of other modifications
no rule in DOT relies on the distinction between $T_q$ and $T_r$.
Therefore, in the absence of other modifications, any $T_q$ is freely interchangable
with any $T_r$ for any given type $T$ in any program $P$ without changing $P$'s meaning.

\subsubsection{No modification described in this chapter allows acceptance of a program
that would be rejected by the original DOT.}
The modifications to DOT rules fall into two categories:
first, added constraints on accepted judgments (i.e., extra conditions above the bar on rules),
and second, viewpoint adaptation of result-type qualifiers.
Modifications of the first category can eliminate judgments that would have
been accepted by the unmodified DOT calculus, but adding extra constraints
on judgments cannot cause a judgment to be accepted where the unmodified DOT calculus
would have rejected it.
Therefore, no modifications of the first category can cause acceptance of a program
that would have been rejected by the original DOT.

Modifications of the second category do not change the form or underlying types
of any judgment. The only changes caused by modifications of the second category
are to the qualifiers on those types.
Since one qualifier can be substituted for another without changing program meaning
with respect to the original DOT, then if the original DOT accepts a judgment
containing a qualified type $T_q$, it also accepts that same judgment where
any other qualifier $r$ is substituted for $q$.
Viewpoint adaptation simply substitues one qualifier for another without changing
the underlying type.
Therefore, no modifications of the second category can cause acceptance
of a program that would have been rejected by the original DOT.

\begin{comment}
\subsubsection{No qualifier-related functions or operations are non-terminating.}

The viewpoint adaptation operator~$\triangleright$ and all auxiliary functions
terminate in constant time except for $env$.
Although $env$ is defined recursively, it always terminates for an input
program of finite size.

I assume here that obtaining the environment qualifier of any class label
\mbox{($env(L^c)$)} is a finite-time operation.
\mbox{$env(p.L^c)$} ...
\end{comment}

\section{Concerns Not Addressed by DOT} \label{dot-unaddressed}

DOT intentionally does not attempt to model inheritance
(that is, the construction of a nominal class hierarchy).
The authors of DOT considered inheritance an unnecessary complication
with respect to proving type safety~\cite{dot}.
The only inheritance-related correctness consideration in the present work
is the covariance of environment references (see section~\ref{sec-env-covariance}),
which must be checked at compile time.

Furthermore, the described modification of DOT does not model environment-reference
consistency requirements (see section~\ref{sec-env-consistency}).
These consistency requirements involve checking that receiver-reference
and environment-reference qualifiers are consistent with the environment
reference qualifiers of enclosing classes.
It seemed to me that the inclusion of environment-reference consistency checking
would unnecessarily complicate the formalism;
however, consistency checking must still be included in the implementation.


\chapter{Default Qualifiers} \label{chap-defaults}
	This chapter discusses default qualifiers for various language features.


\section{Qualifiers for Top-Level Classes}
TBD.

\section{Qualifiers for Nested Classes}
TBD.

\section{Qualifiers for Ordinary Methods}
TBD.

\section{Qualifiers for Fields and Accessor Methods}
TBD.

\section{Qualifiers for Bounded Abstract Types}
TBD.



\chapter{Programmer-facing Annotations} \label{chap-annotations}
	This chapter discusses programmer-facing annotations,
which are the interface between the programmer and the underlying
qualifier system.

\section{The @pure Annotation}
TBD.


\chapter{Related Work} \label{chap-related-work}
TBD.

\chapter{Evaluation} \label{chap-evaluation}
\section{Code Idioms}
TBD. {\em Iterator pattern, construct immutable circular list.}
\section{Performance \& Memory Usage}
TBD.
\section{Annotating the Standard Library}
TBD.

\chapter{Future Work} \label{chap-future-work}
\section{Observational Exposure}
TBD.

\chapter{Conclusion} \label{chap-conclusion}
TBD.



\bibliographystyle{plain}
\renewcommand*{\bibname}{References}
\bibliography{DotMod2015}
\end{document}