\section{DOTMod Syntax}

DOTMod is a modification of the DOT calculus.
Additions to DOT syntax are shown in boxes.
These additions include qualifiers for arbitrary types~(\mbox{$T@q$}),
receiver-type qualifiers for methods~(\mbox{$(@q)...$}),
and annotations that indicate whether named classes may contain methods
that mutate outer classes~(\mbox{$(@O)...$}).

The syntactical elements for concrete types have been removed.
In lieu of concrete types, DOTMod rules form judgements about
{\em constructible types} (discussed below). Constructible types
are a subset of concrete types.

\begin{figure}[htbp]
	\begin{equation*}
	\begin{array}{ll|ll}
		x,y,z & \text{Variable} &
			L::= & \text{Type label} \\
		l & \text{Value label} &
			\qquad \boxed{(@O)} L_c & \quad \text{class label} \\
		m & \text{Method label} &
			\qquad L_a & \quad \text{abstract type label} \\
		v::= & \text{Value} &
			S,T,U,V,W::= & \text{Type} \\
		\qquad x & \qquad \text{variable} &
			\qquad p.L & \quad \text{type selection} \\
		t::= & \text{Term} &
			\qquad T\ \{ z \Rightarrow \overline{D} \} & \quad \text{type refinement} \\
		\qquad v & \qquad \text{value} &
			\qquad T \land T & \quad \text{intersection type} \\
		\qquad {\bf val}\ x = {\bf new}\ c; t & \qquad \text{new instance} &
			\qquad T \lor T & \quad \text{union type} \\
		\qquad t.l & \qquad \text{field selection} & \qquad \top & \quad \text{top type} \\
		\qquad t.m(t) & \qquad \text{method invocation} &
			\qquad \bot & \quad \text{bottom type} \\
		p::= & \text{Path} &
			\qquad \boxed{T@q} & \quad \text{qualified type} \\
		\qquad x & \qquad \text{variable} &
			D::= & \text{Declaration} \\
			%\cancel{S_c, T_c ::=} & \cancel{\text{Concrete type}} \\
		\qquad p.l & \qquad \text{selection} &
			\qquad L:S..U & \quad \text{type declaration} \\
			%\multicolumn{2}{l}{
			%\qquad \cancel{p.L_c\ |
			%	\ T_c\ \{ z \Rightarrow \overline{D} \}\ |
			%	\ T_c \land T_c\ |
			%	\ \top}
			%} \\
		c::=T\ \{\overline{d}\} & \text{Constructor} &
			\qquad l:T & \quad \text{value declaration} \\
		d::= & \text{Initialization} &
			\qquad \boxed{(@q)}m:S \rightarrow U & \quad \text{method declaration} \\
		\qquad l=v & \qquad \text{field initialization} &
			\multicolumn{2}{l}{
				\fbox{
					\begin{minipage}{0.43\linewidth}
						$q,r::= \qquad \qquad \ \text{Type qualifier}$
						
						$\ \ \ \ mutable\ |\ roparam\ |\ rothis\ |\ readonly$

					\end{minipage}
				}
				} \\
		\qquad m(x)=t & \qquad \text{method initialization} &
			\multicolumn{2}{l}{
				\fbox{
					\begin{minipage}{0.43\linewidth}
						$O::= \qquad \qquad \ \ \ \text{Outer mutation qualifier}$
						
						$\ \ \ \ ModOuter\ |\ NoModOuter$

					\end{minipage}
				}
				} \\
	%\multicolumn{2}{l|}{
	%			\fbox{
	%				\begin{minipage}{0.47\linewidth}
	%					$B,A::= Q..R \qquad \quad \text{Qualifier Bounds}$
	%				\end{minipage}
	%			}
	%			} &
		s::=\overline{x \mapsto c} & \text{Store} &
			\Gamma ::= \overline{x:T} & \text{Environment} \\
	\end{array}
	\end{equation*}
	\caption{DOTMod Syntax}
	\label{fig-dot-syntax}
\end{figure}


DOTMod notations and rules are the same as in DOT
unless otherwise noted.

\newpage

%\section{New Rules in DOTMod}

\subsubsection{Qualifier Lattice}

\begin{equation*}\tag{\mbox{Qual-$<:$-1}}\label{equ-}
\begin{array}{c}
\midrule
mutable <: q \\
\end{array}
\end{equation*}

\vspace{0.4cm}

\begin{equation*}\tag{\mbox{Qual-$<:$-2}}\label{equ-}
\begin{array}{c}
\midrule
q <: readonly \\
\end{array}
\end{equation*}

\vspace{0.4cm}

\begin{equation*}\tag{\mbox{Qual-$<:$-3}}\label{equ-}
\begin{array}{c}
\midrule
q <: q \\
\end{array}
\end{equation*}

\vspace{0.4cm}

\begin{equation*}\tag{\mbox{Qual-$\land$}}\label{equ-qual-meet}
\begin{array}{llll}
q \land q'  &=& q & \text{if $q <: q'$} \\
            &=& q' & \text{if $q' <: q$} \\
            &=& mutable & \text{otherwise} \\
\end{array}
\end{equation*}

\vspace{0.4cm}

\begin{equation*}\tag{\mbox{Qual-$\lor$}}\label{equ-qual-join}
\begin{array}{llll}
q \lor q'  &=& q' & \text{if $q <: q'$} \\
            &=& q & \text{if $q' <: q$} \\
            &=& readonly & \text{otherwise} \\
\end{array}
\end{equation*}

\vspace{0.4cm}


\newpage

\subsubsection{Types to Qualifiers}

% The implementation will insert qualifier annotations on every type that
%  does not have an annotation.
%  See section on defaults.

$LowerQual$ and $UpperQual$ are mappings from types to qualifiers.
$LowerQual$ and $UpperQual$ are not defined for every type;
however, in practice we have the compiler assume a default qualifier
wherever $LowerQual$ or $UpperQual$ is not defined.
Qualifier defaults are discussed in another chapter.

\begin{equation*}\tag{Qual-Annot}\label{equ-qual-annot}
\boxed{
\begin{array}{c}
\Gamma \vdash T@q\ {\bf wf} \\
\midrule
\Gamma \vdash LowerQual(T@q) = q\ ,\ UpperQual(T@q) = q \\
\end{array}
}
\end{equation*}

\vspace{0.4cm}

\begin{equation*}\tag{Qual-TSEL-1}\label{equ-qual-tsel-1}
\boxed{
\begin{array}{c}
\Gamma \vdash p \ni L:S..U \\
\Gamma \vdash LowerQual(S) = q \\
\midrule
\Gamma \vdash LowerQual(p.L) = q \\
\end{array}
}
\end{equation*}

\vspace{0.4cm}

\begin{equation*}\tag{Qual-TSEL-2}\label{equ-qual-tsel-2}
\boxed{
\begin{array}{c}
\Gamma \vdash p \ni L:S..U \\
\Gamma \vdash UpperQual(U) = q \\
\midrule
\Gamma \vdash UpperQual(p.L) = q \\
\end{array}
}
\end{equation*}

\vspace{0.4cm}

\begin{equation*}\tag{Qual-RFN-1}\label{equ-qual-rfn-1}
\boxed{
\begin{array}{c}
\Gamma \vdash T\ \{ z \Rightarrow \overline{D} \}\ {\bf wf} \\
\Gamma \vdash LowerQual(T) = q \\
\midrule
\Gamma \vdash LowerQual(T\ \{ z \Rightarrow \overline{D} \}) = q \\
\end{array}
}
\end{equation*}

\vspace{0.4cm}

\begin{equation*}\tag{Qual-RFN-2}\label{equ-qual-rfn-2}
\boxed{
\begin{array}{c}
\Gamma \vdash T\ \{ z \Rightarrow \overline{D} \}\ {\bf wf} \\
\Gamma \vdash UpperQual(T) = q \\
\midrule
\Gamma \vdash UpperQual(T\ \{ z \Rightarrow \overline{D} \}) = q \\
\end{array}
}
\end{equation*}

\vspace{0.4cm}

\begin{equation*}\tag{\mbox{Qual-$\land$-1}}\label{equ-qual-meet-1}
\boxed{
\begin{array}{c}
\Gamma \vdash T \land T'\ {\bf wf} \\
\Gamma \vdash LowerQual(T) = q\ ,\ LowerQual(T') = q' \\
\midrule
\Gamma \vdash LowerQual(T \land T') = q \land q' \\
\end{array}
}
\end{equation*}

\vspace{0.4cm}

\begin{equation*}\tag{\mbox{Qual-$\land$}-2}\label{equ-qual-meet-2}
\boxed{
\begin{array}{c}
\Gamma \vdash T \land T'\ {\bf wf} \\
\Gamma \vdash UpperQual(T) = q\ ,\ UpperQual(T') = q' \\
\midrule
\Gamma \vdash UpperQual(T \land T') = q \land q' \\
\end{array}
}
\end{equation*}

\vspace{0.4cm}

\begin{equation*}\tag{\mbox{Qual-$\lor$-1}}\label{equ-qual-join-1}
\boxed{
\begin{array}{c}
\Gamma \vdash T \lor T'\ {\bf wf} \\
\Gamma \vdash LowerQual(T) = q\ ,\ LowerQual(T') = q' \\
\midrule
\Gamma \vdash LowerQual(T \lor T') = q \lor q' \\
\end{array}
}
\end{equation*}

\vspace{0.4cm}

\begin{equation*}\tag{\mbox{Qual-$\lor$}-2}\label{equ-qual-join-2}
\boxed{
\begin{array}{c}
\Gamma \vdash T \lor T'\ {\bf wf} \\
\Gamma \vdash UpperQual(T) = q\ ,\ UpperQual(T') = q' \\
\midrule
\Gamma \vdash UpperQual(T \lor T') = q \lor q' \\
\end{array}
}
\end{equation*}

\vspace{0.4cm}



\newpage

\subsubsection{Subtyping}

I define two subtype relationships \mbox{$<:_0$} and $<:$.
\mbox{$<:_0$} is the original subtype relationship in DOT,
with the addition of rules \mbox{$<:_0$-Q} and \mbox{Q-$<:_0$}.
These rules ignore type qualifiers.

\mbox{$<:$} also contains all subtyping rules in DOT,
but with the addition of rules \mbox{$<:$-Q} and \mbox{Q-$<:$}.


\begin{equation*}\tag{\mbox{$<:_0$-Q}}\label{equ-lt0-q}
\boxed{
\begin{array}{c}
\Gamma \vdash T <:_0 T' \\
\midrule
\Gamma \vdash T <:_0 T' @ q \\
\end{array}
}
\end{equation*}

\vspace{0.4cm}

\begin{equation*}\tag{\mbox{Q-$<:_0$}}\label{equ-q-lt0}
\boxed{
\begin{array}{c}
\Gamma \vdash T <:_0 T' \\
\midrule
\Gamma \vdash T @ q <:_0 T' \\
\end{array}
}
\end{equation*}

\vspace{0.4cm}

\begin{equation*}\tag{\mbox{$<:$-Q}}\label{equ-lt-q}
\boxed{
\begin{array}{c}
\Gamma \vdash T <:_0 T'\ ,\ UpperQual(T) <: q \\
\midrule
\Gamma \vdash T <: T' @ q \\
\end{array}
}
\end{equation*}

\vspace{0.4cm}

\begin{equation*}\tag{\mbox{Q-$<:$}}\label{equ-q-lt}
\boxed{
\begin{array}{c}
\Gamma \vdash T <:_0 T'\ ,\ q <: LowerQual(T') \\
\midrule
\Gamma \vdash T @ q <: T' \\
\end{array}
}
\end{equation*}

\vspace{0.4cm}



\newpage

\subsubsection{Viewpoint Adaptation}

\begin{equation*}\tag{VPField-1}\label{equ-vpfield-1}
\boxed{
\begin{array}{c}
\Gamma \vdash UpperQual(V) = q \\
\Gamma \vdash UpperQual(T) = rothis \\
\midrule
\Gamma \vdash VPField(V,T) = T @ q \\
\end{array}
}
\end{equation*}

\vspace{0.4cm}

\begin{equation*}\tag{VPField-2}\label{equ-vpfield-2}
\boxed{
\begin{array}{c}
\Gamma \vdash UpperQual(T) = readonly \\
\midrule
\Gamma \vdash VPField(V,T) = T \\
\end{array}
}
\end{equation*}

\vspace{0.4cm}

\begin{equation*}\tag{VPMeth-1}\label{equ-vpmeth-1}
\boxed{
\begin{array}{c}
\Gamma \vdash UpperQual(V) = q \\
\Gamma \vdash UpperQual(T) = rothis \\
\midrule
\Gamma \vdash VPMeth(V,S,T) = T @ q \\
\end{array}
}
\end{equation*}

\vspace{0.4cm}

\begin{equation*}\tag{VPMeth-2}\label{equ-vpmeth-2}
\boxed{
\begin{array}{c}
\Gamma \vdash UpperQual(S) = q \\
\Gamma \vdash UpperQual(T) = roparam \\
\midrule
\Gamma \vdash VPMeth(V,S,T) = T @ q \\
\end{array}
}
\end{equation*}

\vspace{0.4cm}

\begin{equation*}\tag{VPMeth-3}\label{equ-vpmeth-3}
\boxed{
\begin{array}{c}
q \in \{ mutable, readonly \} \\
\Gamma \vdash UpperQual(T) = q \\
\midrule
\Gamma \vdash VPMeth(V,S,T) = T \\
\end{array}
}
\end{equation*}

\vspace{0.4cm}


\subsubsection{Receiver Checking}


\begin{equation*}\tag{QualCompat-1}\label{equ-qual-compat-1}
\boxed{
\begin{array}{c}
q \in \{ rothis, roparam, readonly \} \\
\midrule
\Gamma \vdash QualCompat(V,q) \\
\end{array}
}
\end{equation*}

\vspace{0.4cm}

\begin{equation*}\tag{QualCompat-2}\label{equ-qual-compat-2}
\boxed{
\begin{array}{c}
\Gamma \vdash UpperQual(V) = mutable \\
\midrule
\Gamma \vdash QualCompat(V,mutable) \\
\end{array}
}
\end{equation*}

\vspace{0.4cm}




\subsubsection{Constructibility}

We introduce judgements of the form \mbox{$T\ {\bf constr}$}
to indicate whether a type $T$ is {\em constructible}.

The DOT calculus syntactically differentiated between concrete and
abstract types, where concrete types could be constructed,
but abstract types could not.
However, in DOTMod,
not every concrete type is constructible.

If a class label is annotated \mbox{$@ModOuter$},
but is reached through a non-mutable path, then
the resulting type is not constructible.

Semantically, every access to a member of an outer class is performed through
a hidden field that refers to an instance of that outer class.
\mbox{$@ModOuter$} means that the outer-class reference field is $rothis$.
\mbox{$@NoModOuter$} means that the field either does not exist,
or is $readonly$.
Top-level classes use outer-class references to access global state.

\begin{equation*}\tag{Constr}\label{equ-constr}
\boxed{
\begin{array}{c}
\Gamma \vdash p \ni L:S..U\ ,\ L = (@NoModOuter) L_c \\
\midrule
\Gamma \vdash p.L\ {\bf constr} \\
\end{array}
}
\end{equation*}

\vspace{0.4cm}

\begin{equation*}\tag{Constr}\label{equ-constr}
\boxed{
\begin{array}{c}
\Gamma \vdash p: T\ ,\ p \ni L:S..U\ ,\ L = (@ModOuter) L_c \\
\Gamma \vdash UpperQual(T) = mutable \\
\midrule
\Gamma \vdash p.L\ {\bf constr} \\
\end{array}
}
\end{equation*}

\vspace{0.4cm}

\begin{equation*}\tag{Constr}\label{equ-}
\boxed{
\begin{array}{c}
\Gamma \vdash T\ {\bf constr} \\
\midrule
\Gamma \vdash T \{ z \Rightarrow \overline{D}\}\ {\bf constr} \\
\end{array}
}
\end{equation*}

\vspace{0.4cm}

\begin{equation*}\tag{Constr}\label{equ-}
\boxed{
\begin{array}{c}
\Gamma \vdash T\ {\bf constr}\ ,\ T'\ {\bf constr} \\
\midrule
\Gamma \vdash T \land T'\ {\bf constr} \\
\end{array}
}
\end{equation*}

\vspace{0.4cm}

\begin{equation*}\tag{Constr}\label{equ-}
\boxed{
\begin{array}{c}
\midrule
\Gamma \vdash \top\ {\bf constr} \\
\end{array}
}
\end{equation*}

\vspace{0.4cm}



% Constructability and inheritance
% Constructability and type refinement



\section{Rules Modified in DOTMod}

Most rules in DOT are unchanged in DOTMod.
The rules that are changed are discussed in this section.
Parts of rules newly added to (or changed in) DOTMod are in boxes.

\subsubsection{Type Assignment}

\begin{equation*}\tag{SEL}\label{equ-sel}
\begin{array}{c}
\Gamma \vdash t : V\ ,\ t \ni l : T \\
\boxed{\Gamma \vdash VPField(V, T) = U} \\
\midrule
\Gamma \vdash t.l : \boxed{U} \\
\end{array}
\end{equation*}

\vspace{0.4cm}

\begin{equation*}\tag{MSEL}\label{equ-msel}
\begin{array}{c}
\Gamma \vdash t: V\ ,\ t': S'\ ,\ t \ni (@q)m: S \rightarrow T \\
\boxed{\Gamma \vdash VPMeth(V,S',T) = U} \\
\boxed{\Gamma \vdash QualCompat(V,q)} \\
\boxed{\Gamma \vdash S' <:_0 S\ ,\ LowerQual(S) = q'\ ,\ QualCompat(S',q')} \\
\midrule
\Gamma \vdash t.m(t'): \boxed{U} \\
\end{array}
\end{equation*}

\vspace{0.4cm}

\begin{equation*}\tag{NEW}\label{equ-new}
\begin{array}{c}
y \notin fn(T') \\
\Gamma \vdash T\ {\bf wfe}\ ,\ \boxed{T\ {\bf constr}}\ ,\ T \prec_y \overline{L:S..U},\overline{D} \\
\Gamma,y:T \vdash \overline{S <: U}\ ,\ \overline{d}:\overline{D}\ ,\ t':T' \\
\midrule
\Gamma \vdash {\bf val}\ y = {\bf new}\ T \{ \overline{d} \}\ ;\ t':T' \\
\end{array}
\end{equation*}

\vspace{0.4cm}


\newpage


\subsubsection{Declaration Subsumption}



\begin{equation*}\tag{\mbox{MDECL-$<:$}}\label{equ-mdecl-lt}
\begin{array}{c}
\Gamma \vdash \boxed{q' <: q}\ ,\ S' <: S\ ,\ T <: T' \\
\midrule
\Gamma \vdash ( (@q)m: S \rightarrow T) <: ( (@q')m': S' \rightarrow T') \\
\end{array}
\end{equation*}

\vspace{0.4cm}


\subsubsection{Well-formedness}

A qualified type is well-formed if its underlying type is well-formed.

\begin{equation*}\tag{\mbox{TQ-WF}}\label{equ-tq-wf}
\boxed{
\begin{array}{c}
\Gamma \vdash T\ {\bf wf} \\
\midrule
\Gamma \vdash T@q\ {\bf wf} \\
\end{array}
}
\end{equation*}

\vspace{0.4cm}

Fields cannot be {\em mutable} or {\em roparam}.

\begin{equation*}\tag{\mbox{VDECL-WF}}\label{equ-vdecl-wf}
\begin{array}{c}
\boxed{q \in \{ rothis, readonly \}\ ,\ \Gamma \vdash UpperQual(T) = q}\\
\Gamma \vdash T\ {\bf wf} \\
\midrule
\Gamma \vdash l:T\ {\bf wf} \\
\end{array}
\end{equation*}

\vspace{0.4cm}

Method receivers cannot be {\em roparam}, and
method parameters cannot be {\em rothis}.

\begin{equation*}\tag{\mbox{MDECL-WF}}\label{equ-mdecl-wf}
\begin{array}{c}
\boxed{q \ne roparam\ ,\ q' \ne rothis\ ,\ \Gamma \vdash UpperQual(S) = q'}\\
\Gamma \vdash S\ {\bf wf}\ ,\ U\ {\bf wf} \\
\midrule
\Gamma \vdash (@q)\ m:S \rightarrow U\ {\bf wf} \\
\end{array}
\end{equation*}

\vspace{0.4cm}

\newpage
\subsubsection{Examples}

\begin{lstlisting}
class List {
  type T
  private var _elem: T
  def elem: T = this._elem
  def elem_=(e: T @readonly) = this._elem = e
  @NoModOuter class It {
    (@mutable) def next(): T = this.outer.elem
  }
}
\end{lstlisting}

\begin{lstlisting}
class MuList extends List {
  type T = String @mutable
  //hidden var _elem: T  // ERROR: @mutable fields not allowed
  //def elem: T
  //def elem_=(e: T @readonly)
  //@NoModOuter class It {
  //  (@mutable) def next(): T = this.outer.elem
  //}
}
var r = new MuList @readonly
var i = new r.It
i.next
\end{lstlisting}



\section{}

\begin{lstlisting}

\end{lstlisting}

\section{Defaults}

Field: {\em rothis} if unqualified or {\em mutable} concrete type

Getter method receiver: {\em rothis}

Getter method result: {\em rothis} if unqualified or {\em mutable} concrete type

Setter method receiver: {\em mutable}

Setter method parameter: {\em mutable} if unqualified or {\em rothis} concrete type

Ordinary method receiver: {\em mutable}

Ordinary method parameter: {\em mutable} if unqualified or {\em rothis} concrete type

Top-level class: {\em NoModOuter}

Nested class: {\em ModOuter}


\section{Polymorphic Purity}

Invariant: Pure functions do not mutate their environments.

Which implies {\em rothis} on the function class's {\em apply} method.

All function classes have {\em @NoModOuter}, since they do not mutate their environments
(although derived classes may mutate their environments).

(@NoModOuter) trait Function1[-T,+R] {
  @mutable def apply(t: T): R
}

(@NoModOuter) trait PureFunction1[-T,+R] extends Function1[T,R] {
  @rothis def apply(t: T): R
}

